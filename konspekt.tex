\documentclass[a4paper,11pt]{article}
\usepackage[utf8x]{inputenc}
\usepackage[T1]{fontenc}
\usepackage{polski}
\usepackage{hyphenat}
\usepackage{todonotes}

%opening
\title{Konspekt Pracy - Optymalizacja
wyboru środowiska realizacji baz danych dla potrzeb wysokowydajnych,
złożonych systemów informacyjnych}
\author{Adam Pohorecki}

\begin{document}
\bibliographystyle{plain}
\hyphenation{MySQL NoSQL Cassandra Twitter Facebook Google Amazon BigTable}

\maketitle

\section{Definicja Problemu i Cel Pracy}

\subsection*{Wstęp}

Popularyzacja dostępu do internetu przyczyniła się w ostatnich latach do 
powstania licznych serwisów, które mogą się pochwalić dziesiątkami, a nawet
setkami milionów odwiedzin dziennie. Twórcy serwisów takich jak Facebook,
Twitter czy Google zostali postawieni przed problemem stworzenia rozproszonej
architektury bazodanowej, która zapewni horyzontalną skalowalność przy
równoczesnym zachowaniu wysokiej dostępności. Ze względu na występujący jeszcze
kilka lat temu brak takich rozwiązań na rynku firmy te były zmuszone do
implementacji tego typu baz na własne potrzeby. Ostatnimi laty jednak sytuacja
zaczęła ulegać zmianie: produkty wytworzone na wewnętrzny użytek zostają
upubliczniane na licencjach Open Source, dzięki czemu obecnie mamy już dostęp do
przynajmniej kilkunastu rozwiązań umożliwiających dostęp do peta-bajtów danych
rozproszonych na tysiącach serwerów.

\subsection*{NoSQL}

Powszechny i darmowy dostęp\todo{przykładowa uwaga}\ do produktów bazodanowych obiecujących wysoką
skalowalność zaowocował powstaniem ruchu określanego popularnie mianem NoSQL
(\emph{Not only SQL} - nie tylko SQL). Nazwa ta bierze się stąd, że
większość nowych produktów bazodanowych stworzonych z myślą o wysokiej
skalowalności nie dysponuje możliwością zadawania zapytań w języku SQL. Ruch ten
jest zainteresowany problematyką dużych i bardzo dużych serwisów internetowych,
które ze względów wydajnościowych nie stać na wykorzystywanie tradycyjnych
silników bazodanowych z transakcjami i łączeniem tabel w zapytaniach.

W literaturze trudno znaleźć nazwę problemu, który próbuje rozwiązać ruch
NoSQL. W artykule \cite{monash-db-hvsp} autor sugeruje nazwę HVSP (ang.
\emph{High Volume Simple Processing}). Sugerowane wyróżniki problemu to:
\begin{itemize}
 \item Wielu równocześnie korzystających z bazy użytkowników, dokonujących
zarówno odczytów, jak i zapisów.
 \item Operacje wykonywane przez użytkowników są nieskomplikowane, bez
transakcji, łączenia tabel, czy operacji grupujących.
\end{itemize}

Dotychczas popularnym rozwiązaniem było stosowanie zdenormalizowanych,
horyzontalnie dzielonych (ang. \emph{sharded}) i replikowanych baz (np. MySQL)
wspartych przez przechowywany w pamięci RAM serwerów cache (zazwyczaj
Memcached). Produkty NoSQL zazwyczaj idą o krok dalej całkowicie pozbywając się
transakcji i joinów, a często także sztywnego schematu bazy.

\subsection*{Cel pracy}
Praca dyplomowa, której konspekt niniejszym przedstawiam, ma na celu opisanie
dostępnych na rynku rozwiązań NoSQL, zwracając uwagę na szczególne właściwości
i możliwości zastosowania poszczególnych produktów, jak również ich
dojrzałości, tempa ich rozwoju, oceny aktywności ich użytkowników i ekosystemu
narzędzi z nimi związanych. Elementem tego porównania będzie także zbudowanie
frameworku do przeprowadzania testów na części zaprezentowanych produktów,
zróżnicowanych pod względem klas tych rozwiązań (osobno dla \emph{document
oriented} \emph{key-value stores}, \emph{table oriented}, \emph{graph
oriented}). Ponieważ każde narzędzie dysponuje własną nomenklaturą konieczne
będzie także wprowadzenie jednolitego nazewnictwa, które ułatwi zrozumienie,
oraz porównanie opisanych produktów.

Porównanie takie uważam za wartościowe, ze względu na to, iż będzie
najprawdopodobniej jednym z pierwszych obszernych porównań produktów NoSQL, a
możliwe że pierwszym takim napisanym w języku polskim.
  
\section{Studium Literatury}
W studium literatury zajmę się opisaniem teorii CAP\cite{brewers-conjecture}
(\emph{Consistency, Availability, Partition Tolerance}), zwanej również
teorią Brewer'a od nazwiska jej autora. Teoria ta twierdzi, że rozproszony
system nie jest w stanie zapewnić równocześnie wszystkich trzech gwarancji:
konsystencji danych widzianych przez węzły systemu, odporności całości systemu
na awarie poszczególnych jego węzłów oraz odporności systemu na utratę
połączenia pomiędzy poszczególnymi węzłami lub ich grupami. Przedstawię także
zasady działania Google BigTable\cite{google-bigtable} oraz Amazon
Dynamo\cite{amazon-dynamo}, które bardzo istotnie wpłynęły na architekturę
systemów NoSQL. Porównam także semantykę BASE (\emph{Basically Available,
Soft-state, Eventually-consistent}) z ACID (\emph{Atomicity, Consistency,
Isolation, Durability}).

\section{Opis dostępnych rozwiązań}
Przedstawiona zostanie klasyfikacja rozwiązań NoSQL (podział na document
oriented, table oriented, key-value stores, graph oriented i RDBMS based) oraz
dla każdej z grup przedstawię charakterystykę 2-3 produktów przynależących do
danej kategorii.

\section{Porównanie dostępnych rozwiązań}
W tej części dokonam opisu eksperymentów przeprowadzonych z użyciem wcześniej
opisanych produktów, oraz stworzonego na potrzeby pracy frameworka do ich
testowania. Przedstawiona zostanie także analiza wyników powyższych
eksperymentów.

\section{Zakończenie}
Praca zostanie zakończona prezentacją wniosków wynikających ze zdobytych
informacji na temat systemów NoSQL oraz przeprowadzonych eksperymentów.
Dołączona bibliografia zawierać będzie notatki dotyczące poszczególnych
pozycji, a w załączniku do pracy, dla wygody przyszłych czytelników,
dołączę komplet artykułów wymienionych w bibliografii.

\bibliography{bibliografia}

\end{document}
