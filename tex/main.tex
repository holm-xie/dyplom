\documentclass[12pt]{report}

%-------------------------------------------------------------------------------
% PAKIETY
%-------------------------------------------------------------------------------

% obsluga jezyka polskiego
\usepackage[utf8x]{inputenc}
\usepackage[OT4]{fontenc}
\usepackage{polski}

% dzielenie wyrazow
\usepackage{hyphenat}

% dla /todo
\usepackage{todonotes}

% skorowidz
\usepackage{makeidx}

\usepackage[dvips, bookmarks, colorlinks=false, pdfauthor={Adam Pohorecki}, pdfsubject={Praca Magisterska}, pdfkeywords={master thesis, nosql, scalability}]{hyperref}

% page headers
\usepackage{fancyhdr}
\setlength{\headheight}{15.2pt}

\pagestyle{fancyplain}
\renewcommand{\chaptermark}[1]{\markboth{#1}{}}
 
\lhead{\fancyplain{}{\textit{\rightmark}}}
\chead{}
\rhead{}
\lfoot{}
\cfoot{\fancyplain{}{\thepage}}
\rfoot{}

\makeindex

\bibliographystyle{apalike}

%-------------------------------------------------------------------------------
% Dane pracy
%-------------------------------------------------------------------------------

\author{Adam Pohorecki}
\title{Optymalizacja wyboru środowiska realizacji baz danych dla potrzeb wysokowydajnych systemów informacyjnych}

% ograniczenia co do tego co ma byc zalaczone w  wygenerowanym pdf
%\includeonly{chapters/definicja_problemu/definicja_problemu}
%\includeonly{chapters/studium_literatury/studium_literatury}
%\includeonly{chapters/opis_rozwiazan/opis_rozwiazan}
%\includeonly{chapters/porownanie_rozwiazan/porownanie_rozwiazan}
%\includeonly{chapters/wnioski/wnioski}

%-------------------------------------------------------------------------------
% Treść pracy
%-------------------------------------------------------------------------------

\begin{document}

% slowa, ktore nie powinny byc dzielone na koncu linii
\hyphenation{
Amazon
BigTable
Cassandra 
Facebook
Google
HBase
MySQL 
NoSQL 
SimpleDB
Twitter 
}

% moje komendy
\newcommand{\id}[1]{\index{#1}}  
\newcommand{\wi}[1]{#1\index{#1}}  
\newcommand{\wwi}[1]{\emph{#1}\index{#1}}  
\newcommand{\mwi}[1]{\textbf{#1}\index{#1}}
\newcommand{\ii}[1]{\textit{#1}}
\newcommand{\myfigure}[3]{\begin{figure}[!ht]\centering\includegraphics[width=0.75\textwidth]{#1}\caption{#2}\label{#3}\end{figure}}


%-------------------------------------------------------------------------------
% Część początkowa
%-------------------------------------------------------------------------------

\maketitle

\tableofcontents

%-------------------------------------------------------------------------------
% Główna treść pracy
%-------------------------------------------------------------------------------

% rozdzialy pracy
\chapter{Wstęp}

\section*{Streszczenie}
Wstęp do pracy, opisujący jej cel i strukturę.

\chapter{Definicja Problemu}

\section*{Streszczenie}
W poniższym rozdziale zostanie zaprezentowany problem, który usiłują rozwiązać systemy bazodanowe opisane w niniejszej pracy.
Przedstawione zostaną także sposoby radzenia sobie z tym problemem przy zastosowaniu relacyjnych baz danych.

\section{Wstęp}
Popularyzacja dostępu do internetu przyczyniła się w ostatnich latach do  powstania licznych serwisów, które mogą się pochwalić dziesiątkami, a nawet setkami milionów odwiedzin dziennie. 
Twórcy serwisów takich jak Facebook, Twitter czy Google zostali postawieni przed problemem stworzenia rozproszonej architektury bazodanowej, która zapewni horyzontalną skalowalność przy równoczesnym zachowaniu wysokiej dostępności. 
Ze względu na występujący jeszcze kilka lat temu brak takich rozwiązań na rynku firmy te były zmuszone do implementacji tego typu baz na własne potrzeby. 
Ostatnimi laty jednak sytuacja zaczęła ulegać zmianie: produkty wytworzone na wewnętrzny użytek zostają upubliczniane na licencjach Open Source, dzięki czemu obecnie mamy już dostęp do przynajmniej kilkunastu rozwiązań umożliwiających dostęp do peta-bajtów danych rozproszonych na tysiącach serwerów.

\section{Skalowalność}
Niniejsza praca stawia przed sobą zadanie wprowadzenia czytelnika w świat wysoce skalowalnych systemów bazodanowych.
Aby jednak tego dokonać musimy najpierw przedstawić, czym jest skalowalność w rozumieniu autora i jakiego typu skalowalnością będziemy się zajmować.

Wyobraźmy sobie aplikację internetową, na przykład stronę społecznościową taką jak Nasza Klasa czy Facebook.
Aplikacje takie jak ta początkowo mają tysiące użytkowników, a gdy odniosą sukces, liczba ta może wzrosnąć do dziesiątek czy nawet setek milionów.
Ponadto aplikacje te podlegają ciągłemu rozwojowi przez wiele lat, co bardzo często pociąga za sobą konieczność wprowadzania zmian w schemacie bazy, a także niejednokrotnie dokonywania migracji danych.

System bazodanowy jest skalowalny w rozumieniu tej pracy, jeżeli koszty jego utrzymania wzrastają co najwyżej liniowo wraz ze wzrostem liczby użytkowników aplikacji. 
Warunkiem koniecznym jest także to, aby czas potrzebny na wykonanie operacji na rzecz użytkownika aplikacji (zapisy, odczyty), nie degradował się znacznie wraz ze wzrostem ilości danych, którymi system zarządza.

\section{Problemy z tradycyjnymi rozwiązaniami}
Dotychczas najczęściej stosowanym rozwiązaniem problemów skalowalności było zastosowanie bazy danych takiej jak MySQL połączonej z Memcached jako przechowywany w pamięci operacyjnej, rozproszony cache\cite{highscalability-mysql-end-of-an-era}.
Opiszmy pokrótce, jak takie skalowanie aplikacji w uproszczeniu wygląda.

\subsection{Replikacja}
Początkowo cała baza danych mieści się na jednej maszynie i jest w stanie poradzić sobie ze wszystkimi operacjami odczytu i zapisu jakich dokonuje aplikacja.
Wraz ze wzrostem liczby użytkowników serwer zaczyna docierać do granic swojej przepustowości. 
Ponieważ aplikacje internetowe wykonują zazwyczaj dużo więcej odczytów niż zapisów, dodajemy stopniowo kilka kolejnych serwerów replikowanych w konfiguracji master-slave, zwiększając w ten sposób liczbę równoczesnych odczytów, jednak nie zwiększając (a nawet zmniejszając) liczby możliwych zapisów.
Alternatywą jest dodawanie kolejnych serwerów w konfiguracji master-master, która jest zazwyczaj trudniejsza do poprawnego skonfigurowania, ale za to w przypadku gdy jeden z serwerów master zawiedzie, inny z łatwością przejmie jego rolę.

\missingfigure{replikacje master-slave}

Warto jednak zwrócić uwagę, na to jak szybko takie rozwiązanie się degraduje.
Załóżmy, że początkowo serwer \emph{master} jest obciążony w 50\% zapisami.
Oznacza to, że każdy z węzłów \emph{slave} musi także poświęcić 50\% zasobów na radzenie sobie z replikacją zapisów\footnote{Zakładam, że wszystkie serwery są identyczne}, a resztę może poświęcić na przetwarzanie odczytów.
Wraz ze wzrostem obciążenia serwera \emph{master}, serwery \emph{slave} muszą coraz więcej zasobów poświęcać na replikację zapisów, co zmniejsza ogólną przepustowość systemu.
Ponadto problemem może być także to, że niektóre systemy baz danych, tak jak, np. MySQL dokonują replikacji asynchronicznie.
Powoduje to, że po dokonaniu operacji zapisu, odczyt może zwrócić wartość niezgodą z zapisaną.
Wymaga to dodatkowej pracy ze strony programisty\cite{zaitsev-scaling-mysql}.

\subsection{Partycjonowanie}
Kiedy oczywiste staje się, że nasz serwer nie jest w stanie sobie poradzić z obciążeniem wynikającym z zapisów, mamy dwie drogi do wyboru: wymienić maszynę na mocniejszą, albo zacząć przetrzymywać tylko część danych na pojedynczym serwerze.
Pierwsza opcja, nazywana \emph{skalowaniem wzwyż}, ma pewne zalety: nie wymaga zmian w aplikacji i jej nie komplikuje.
Wadą tego rozwiązania jest jednak cena w porównaniu do możliwości.
Cytując za artykułem na blogu \emph{Coding Horror}\cite{codinghorror-scaling-up-vs-out}: za cenę stu tysięcy dolarów możemy albo kupić jeden potężny serwer z 32 procesorami, 512GB RAM i 4TB przestrzeni dyskowej, albo 83 mniejsze serwery posiadające łącznie 332 rdzenie, 664GB RAM i 40,5TB przestrzeni dyskowej.
Oczywiście nie można w ten sposób skalować aplikacji wzwyż w nieskończoność, ale warto pamiętać, że tylko nieliczne strony mogą konkurować rozmiarami z takimi gigantami jak Facebook.
Większość aplikacji nigdy nie wyrośnie ponad rozmiar, w którym superkomputer nie wystarcza już do utrzymania ich obciążenia.

\missingfigure{super komputer}

Z drugiej strony mamy partycjonowanie danych, czyli \emph{skalowanie aplikacji wszerz}.
W tym rozwiązaniu wraz ze wzrostem liczby użytkowników zamiast wymieniać serwer na większy i mocniejszy dodajemy kolejne maszyny.
Nie możemy jednak całości bazy danych przechowywać na każdym z serwerów - problemy z tym związane zostały opisane powyżej.
W związku z tym, dokonuje się podziału danych (partycjonowania) poziomego, pionowego albo obu równocześnie.

Kiedy mowa o pojedynczej bazie danych, horyzontalne partycjonowanie oznacza, że rzędy na podstawie jakiegoś kryterium trafiają do różnych tabel.
Dla przykładu: tabela użytkowników może być podzielona w taki sposób, że użytkownicy z Polski trafiają do tabeli \verb=polish_users=, natomiast użytkownicy z Niemiec do tabeli \verb=german_users=.
Tego rodzaju partycjonowania dokonuje się zazwyczaj w celu zwiększenia wydajności i niektóre bazy danych (w tym MySQL\footnote{http://dev.mysql.com/tech-resources/articles/performance-partitioning.html}) pozwalają na definiowanie partycjonowania w schemacie bazy danych.
W przypadku rozproszonych baz danych, mówimy zamiennie o partycjonowaniu horyzontalnym bądź shardowaniu (ang. \emph{sharding}).
Różni się ono od partycjonowania poziomego dla pojedynczej bazy tym, że dane zamiast trafiać do różnych tabel, trafiają do tej samej tabeli, ale na różne węzły bazy danych.

\missingfigure{shardownie}

Partycjonowanie pionowe (wertykalne) podobnie ma dwa znaczenia w zależności od kontekstu.
Dla pojedynczej bazy polega ono na podziale tabel na mniejsze (zawierające tylko część kolumn).
Przypomina to normalizację bazy, ale dzielimy także już znormalizowane tabele, np. po to aby aby część kolumn umieścić na innym dysku, albo aby oddzielić częściej odczytywane dane od rzadziej odczytywanych.
W rozproszonym systemie partycjonowanie pionowe oznacza, że nie połączone ze sobą tabele mogą być umieszczone na różnych węzłach bazy danych.
W ten sposób zapis do jednej tabeli nie obciąża dwóch, tylko jeden serwer.
Oczywiście bardzo często nie jest możliwe znalezienie tabel, które nie są za sobą połączone pośrednio lub bezpośrednio.
Dlatego często takie partycjonowanie wymaga zduplikowania części tabel na obu węzłach.

Opisane wyżej formy partycjonowania systemów rozproszonych najczęściej muszą być obsługiwane przez warstwę aplikacji.
Z tego względu praktycznie niemożliwe jest zapewnienie transakcyjności pomiędzy poszczególnymi węzłami.
Kosztowne i trudne są także operacje joinowania pomiędzy serwerami.

\subsection{Memcached}
\todo{tu skończyłem}

\section{NoSQL}
Powszechny i darmowy dostęp do produktów bazodanowych obiecujących wysoką skalowalność zaowocował powstaniem ruchu określanego popularnie mianem NoSQL (\emph{Not only SQL} - nie tylko SQL). 
Nazwa ta bierze się stąd, że większość nowych produktów bazodanowych stworzonych z myślą o wysokiej skalowalności nie dysponuje możliwością zadawania zapytań w języku SQL. 
Ruch ten jest zainteresowany problematyką dużych i bardzo dużych serwisów internetowych, które ze względów wydajnościowych nie stać na wykorzystywanie tradycyjnych silników bazodanowych z transakcjami i łączeniem tabel w zapytaniach.

W literaturze trudno znaleźć nazwę problemu, który próbuje rozwiązać ruch NoSQL. 
W artykule \cite{monash-db-hvsp} autor sugeruje nazwę HVSP (ang. \emph{High Volume Simple Processing}). 
Sugerowane wyróżniki problemu to:
\begin{itemize}
 \item Wielu równocześnie korzystających z bazy użytkowników, dokonujących zarówno odczytów, jak i zapisów.
 \item Operacje wykonywane przez użytkowników są nieskomplikowane, bez transakcji, łączenia tabel, czy operacji grupujących.
\end{itemize}

Dotychczas popularnym rozwiązaniem było stosowanie zdenormalizowanych, horyzontalnie dzielonych (ang. \emph{sharded}) i replikowanych baz (np. MySQL) wspartych przez przechowywany w pamięci RAM serwerów cache (zazwyczaj Memcached). 
Produkty NoSQL zazwyczaj idą o krok dalej całkowicie pozbywając się transakcji i joinów, a często także sztywnego schematu bazy.

\section{Cel pracy}
Praca dyplomowa, której konspekt niniejszym przedstawiam, ma na celu opisanie dostępnych na rynku rozwiązań NoSQL, zwracając uwagę na szczególne właściwości i możliwości zastosowania poszczególnych produktów, jak również ich dojrzałości, tempa ich rozwoju, oceny aktywności ich użytkowników i ekosystemu narzędzi z nimi związanych. 
Elementem tego porównania będzie także zbudowanie frameworku do przeprowadzania testów na części zaprezentowanych produktów, zróżnicowanych pod względem klas tych rozwiązań (osobno dla \emph{document oriented} \emph{key-value stores}, \emph{table oriented}, \emph{graph oriented}).
Ponieważ każde narzędzie dysponuje własną nomenklaturą konieczne będzie także wprowadzenie jednolitego nazewnictwa, które ułatwi zrozumienie, oraz porównanie opisanych produktów.

Porównanie takie uważam za wartościowe, ze względu na to, iż będzie najprawdopodobniej jednym z pierwszych obszernych porównań produktów NoSQL, a możliwe że pierwszym takim napisanym w języku polskim.

\chapter{Studium Literatury}

\section*{Streszczenie}

W studium literatury zajmę się opisaniem teorii CAP \cite{brewers-conjecture} (\emph{Consistency, Availability, Partition Tolerance}), zwanej również teorią Brewer'a od nazwiska jej autora. 
Teoria ta twierdzi, że rozproszony system nie jest w stanie zapewnić równocześnie wszystkich trzech gwarancji: konsystencji danych widzianych przez węzły systemu, odporności całości systemu na awarie poszczególnych jego węzłów oraz odporności systemu na utratę połączenia pomiędzy poszczególnymi węzłami lub ich grupami. 
Przedstawię także zasady działania Google BigTable \cite{google-bigtable} oraz Amazon Dynamo \cite{amazon-dynamo}, które bardzo istotnie wpłynęły na architekturę systemów NoSQL. 
Porównam także semantykę BASE (\emph{Basically Available, Soft-state, Eventually-consistent}) z ACID (\emph{Atomicity, Consistency, Isolation, Durability}).\todo{napisać dlaczego, uporządkować}

\section{Teoria CAP}

Teoria CAP (\emph{Consistency, Availability, Partition Tolerance}) zwana również teorią Brewera od nazwiska jej autora została po raz pierwszy zaprezentowana podczas prezentacji profesora Uniwersytetu Berkley Eryka Brewera 19 czerwca 2000r. na konferencji \emph{ACM Symposium on the Principles of Distributed Computing} \cite{podc-keynote}. 
W około dwa lata później, w 2002 roku, teoria ta (pod nazwą \emph{Brewer's Conjecture} - Domysł Brewera) została formalnie udowodniona przez Nancy Lynch oraz Setha Gilberta z MIT \cite{brewers-conjecture}.

Teoria CAP powstała jako efekt doświadczeń Brewera w firmie Inktomi oraz jego prac badawczych nad systemami rozproszonymi na Uniwersytecie w Berkley. 
Mówi ona, że z trzech pożądanych właściwości systemu rozproszonego: Konsystencji (ang. \emph{Consistency}), Wysokiej Dostępności (ang. \emph{Availability}) oraz Odporności na Podział Sieci (ang. \emph{Partition Tolerance}) możliwe jest zapewnienie co najwyżej dwóch z nich \cite{browne-cap-theorem}.

\subsection*{Konsystencja}

Termin ,,Konsystencja'' w Teorii CAP ma nieco inne znaczenie niż w ACID, gdzie oznacza on, iż zapisywane dane nie mogą złamać pewnych określonych reguł integralności. 
W Teorii CAP \emph{Consistency} jest dużo bardziej zbliżone do \emph{Atomicity} z ACID, oznacza ono bowiem, że gdy dokonamy operacji zapisu $x=x_0$ każdy kolejny odczyt $x$, niezależnie do którego węzła byłby skierowany, zwróci wartość $x_0$.

\subsection*{Wysoka Dostępność}

Wysoka dostępność jest najbardziej pożądaną właściwością z trzech wymienionych.
Oznacza ona, jak sama nazwa wskazuje, że system powinien udostępniać swoje usługi w pełni przez cały czas, wliczając w to awarie poszczególnych węzłów, aktualizacje oprogramowania czy awarie sieci. 
Bardziej formalnie: jeżeli operacja dotrze do nie ulegającego właśnie awarii węzła, to w pewnym skończonym czasie zwróci wynik do klienta.
Warto przy tym zwrócić uwagę (za  \cite{brewers-conjecture}), że dostępność zawodzi najczęściej właśnie wtedy, gdy jest najbardziej potrzebna, czyli w okresach największego obciążenia systemu.

\subsection*{Odporność na Podział Sieci}

W systemach rozproszonych, działających na tysiącach węzłów, często rozsianych w centrach obliczeniowych na wielu kontynentach utrata połączenia pomiędzy grupami węzłów jest oczekiwanym, codziennym problemem.
Podział sieci, rozumiany tu właśnie jako utrata połączenia między dowolnymi dwoma lub więcej węzłami systemu, przy zachowaniu połączenia tych węzłów z klientem, może nastąpić z wielu powodów: awarie switchy lub routerów, utrata połączenia między centrami obliczeniowymi, dokonywane naprawy.
W przypadku kiedy węzeł lub ich grupa zostanie całkowicie odcięty od klientów i innych węzłów systemu, możemy traktować te węzły tak samo jak gdyby na przykład odcięto im nagle zasilanie, dlatego definicja podziału takiego przypadku nie obejmuje.

Dla przykładu wyobraźmy sobie bazę danych replikowaną w systemie master-master.
Jeżeli połączenie między węzłami zostanie zerwane, modyfikacje dokonane na jednym z nich nie będą widoczne na drugim. 
Z kolei w sytuacji gdy mamy bazę danych z horyzontalnym podziałem danych (ang. \emph{sharded}), ponieważ każdy z serwerów zawiera informacje dotyczące tylko części danych i z góry wiadomo do którego z nich należy się zwrócić aby otrzymać informacje dotyczące dowolnego klucza, nawet w przypadku utraty połączenia między nimi, o ile klient ma dostęp do obu serwerów, ciągłość dostarczanych usług jest zapewniona.

\subsection*{Znaczenie Teorii}

Teoria CAP nabiera znaczenia w miarę wzrostu wielkości systemu. 
Gdy dysponujemy bazą rozsianą na kilku maszynach, narzut czasowy replikacji danych pomiędzy nimi jest akceptowalny dla większości zastosowań, nie musimy się też zbytnio martwić o podział sieci, gdyż zazwyczaj jest tak, że maszyny zlokalizowane w jednej szafie (ang. \emph{rack}) będą albo działać wszystkie, albo żadna. 
Kiedy jednak zajmujemy się usługami rozproszonymi na tysiącach węzłów, nawet gdybyśmy dysponowali 10000 maszynami o niezawodności MTBF 30 lat, każdego dnia następowałaby awaria którejś z nich \cite{google-lessons}. 
W przypadku tak dużych systemów, czas jakiego wymaga replikacja danych aby doprowadzić aby każdy węzeł widział ten sam stan, czy to jak system reaguje na podziały sieci nabiera o wiele większego znaczenia.

\subsubsection*{Podział systemów}

Teoria CAP mówi, że nie możemy zapewnić równocześnie wszystkich trzech gwarancji, dlatego skalowalne systemy muszą porzucić jedną z nich. 
Ze względu na to dzielimy je na:

\begin{enumerate}
 \item \emph{CA} - (Consistent, Available) te systemy mają problemy z podziałem sieci, wymagając zazwyczaj aby operacje dotyczące poszczególnych transakcji trafiały do pojedynczej grupy węzłów, które podlegają awarii ,,atomowo'' - albo wszystkie działają, albo żaden. 
 To podejście zazwyczaj wiąże się z problemami dla skalowalności.
 \item \emph{AP} - (Available, Partition-Tolerant) te systemy zapewniają największą odporność na awarie wynikające z rozproszonego środowiska, jednocześnie jednak stawiając twórców aplikacji przed trudnym zadaniem radzenia sobie z problemami wynikającymi z niespójności danych widzianych przez klientów bazy.
 \item \emph{CP} - (Consistent, Partition-Tolerant) te systemy w wypadku podziału oczekują na przywrócenie połączenia, ograniczając w ten sposób dostępność.
\end{enumerate}

W praktyce systemy omawiane w tej pracy zazwyczaj zadowalają się częściowym zapewnieniem wszystkich wymienionych gwarancji, przy czym wysoka dostępność odgrywa główną rolę, a poświęcana jest albo odporność na podziały, albo konsystencja. 
Dlatego systemy te należą albo do grupy CA (Google BigTable, HBase), albo do grupy AP (Amazon Dynamo, Riak).
\todo{Podział na CA, AP, CP - po lepszym poznaniu poszczególnych rozwiązań} 

\section{BASE}

BASE (\emph{Basically Available, Soft state, Eventual consistency}) to model konsystencji, który jest przeciwstawiany modelowi ACID (\emph{Atomicity, Consistency, Isolation, Durability}). 
Oba te modele dotyczą baz danych, zatem oba wymagają aby operacje klienta były trwałe (\emph{Durability}).
Różnica między nimi bierze się z podejścia do konsystencji.
W ACID system zawsze musi być w spójnym stanie, a dokonanie zmiany może wymagać szeregu operacji które doprowadzą system do tego stanu.
Operacje te ponadto zostają objęte transakcją i zostają aplikowane albo wszystkie, albo żadna.
W BASE system może być w stanie niespójnym z punktu widzenia aplikacji, a operacje które mają przywrócić tą spójność są wykonywane asynchronicznie.

Producenci relacyjnych baz danych od dawna już byli świadomi potrzeby partycjonowania danych na wiele węzłów.
Aby zapewnić semantykę ACID w kontekście rozproszonych transakcji stosuje się technikę 2PC (ang. \emph{2 Phase Commit}).
Protokół 2PC działa dwustopniowo:

\begin{enumerate}
 \item Najpierw koordynator transakcji żąda od wszystkich węzłów biorących udział w operacji aby wstępnie dokonały operacji commit dla transakcji i potwierdziły możliwość wykonania tej operacji.
 Jeżeli wszystkie węzły dokonały tego potwierdzenia, przechodzi się do drugiego kroku.
 \item W drugim kroku koordynator żąda od wszystkich zainteresowanych węzłów dokonania operacji commit.
 Jeżeli którakolwiek z baz zawetuje tą operację, wszystkie muszą wycofać transakcję.
\end{enumerate}

Problem, na jaki napotykamy w tym podejściu, to ograniczenie dostępności systemu (A w CAP).
Wystarczy aby jeden z węzłów systemu podległ awarii, aby cały system stał się niedostępny dla zapisów.
Dostępność w kontekście transakcji staje się iloczynem dostępności poszczególnych węzłów systemu.
Jeżeli mamy zatem węzły o indywidualnej dostępności 99.9\% to transakcja, która obejmuje trzy z nich będzie miała dostępność ok. 99.7\% - czyli o ok. 90 minut mniejszy \emph{uptime} w skali miesiąca \cite{base-an-acid-alternative}.

Jeżeli zatem ACID oferuje nam poziom konsystencji, który można byłoby określić mianem Strong Consistency, ale kosztem dostępności, to BASE oferuje w zamian wysoką dostępność kosztem konsystencji. 

\subsection{Eventual Consistency}

Eventual Consistency (w wolnym tłumaczeniu: konsystencja po pewnym czasie, ostatecznie) to słaba forma gwarancji konsystencji, która gwarantuje jedynie, że po pewnym, możliwym do przewidzenia czasie od momentu wykonania operacji, jej efekty będą widziane przez klientów systemu, niezależnie od tego, do którego z węzłów systemu się zwrócą z zapytaniem.
Okno czasowe między operacją a propagacją jej efektów do wszystkich zainteresowanych węzłów w systemie nazywamy oknem niespójności (ang. \emph{inconsistency window}).

Choć początkowo może się wydawać, że tego typu model sprawia duże trudności w implementacji aplikacji korzystających z baz danych, które go zapewniają, w rzeczywistości tego typu interakcje napotykamy każdego dnia.
Kiedy w systemie bankowym dokonujemy przelewu z konta na konto, pieniądze znikają z bilansu jednego z nich, ale na drugim pojawiają się z pewnym opóźnieniem.
Innym przykładem może być system DNS, gdzie zmiana jest propagowana w systemie stopniowo, często oczekując na przeterminowanie cache, ale po jakimś czasie jest zauważalna u każdego klienta.

Artykuł \cite{vogels-eventually-consistent} wprowadza ponadto kilka rodzajów Eventual Consistency:

\begin{enumerate}
 \item \emph{Causal Consistency} - Jeżeli proces A wykonał jakąś operację, a następnie zakomunikował ten fakt procesowi B, to proces B będzie widział zmienione dane w taki sam sposób jak proces A, a jego zapisy nie będą wchodzić w konflikt z tą operacją.
 Proces C, któremu ta informacja nie została przekazana, będzie podlegał normalnym regułom.
 \item \emph{Read-your-writes Consistency} - Proces dokonujący operacji w kolejnych operacjach zawsze widzi rezultaty tej operacji.
 \item \emph{Session Consistency} - To praktyczna realizacja poprzedniego rodzaju konsystencji.
 W tym przypadku proces komunikuje się z systemem w kontekście sesji, w ramach której ma zapewnioną gwarancję odczytu swoich operacji.
 W przypadku awarii i konieczności nawiązania nowej sesji, gwarancje te nie przechodzą na nowo nawiązaną sesję.
 \item \emph{Monotonic Write Consistency} - W tym przypadku system gwarantuje, że operacje zostaną wykonane w tej samej kolejności, w jakiej żądania ich wykonania zostały wysłane.
 Systemy które nie oferują tej gwarancji są bardzo trudne w użyciu.
\end{enumerate}

\subsubsection*{Konfiguracja Eventual Consistency}

Werner Vogels w artykule o Eventual Consistency \cite{vogels-eventually-consistent} wprowadził nomenklaturę stosowaną w konfiguracji konsystencji wielu systemów NoSQL (np. Cassandra, Riak).
Konfigurację tą opisują zazwyczaj trzy liczby:

\begin{enumerate}
 \item \emph{N} - liczba węzłów systemu na które zostanie zreplikowany pojedynczy rekord.
 Liczba ta jest zazwyczaj określana jako parametr konfiguracji systemu, lub podczas wydawania polecenia utworzenia ``tabeli'' (nazwanego zbioru rekordów).
 Niektóre systemy pozwalają na zmianę tej wartości w trakcie działania systemu (np. Riak), ale większość wymaga ponownego uruchomienia aplikacji w celu zastosowania tej zmiany.
 \item \emph{R} - liczba węzłów systemu, które muszą dokonać odczytu zanim wartość zostanie zwrócona do klienta.
 Czasem wartości zwrócone przez poszczególne węzły będą różne.
 Wtedy system  odpowiada albo za rozwiązanie konfliktu, albo za przekazanie wielu wersji klientowi.
 Parametr R jest najczęściej określany z osobna dla każdego polecenia odczytu.
 \item \emph{W} - liczba węzłów systemu, które muszą potwierdzić zapis aby operacja została zakończyła się sukcesem.
 Podobnie jak R jest to parametr przekazywany dla każdego zapytania.
\end{enumerate}

Jeżeli $R+W>N$, to mamy do czynienia z silną konsystencją (ang. \emph{Strong Consistency}) - każdy odczyt zwróci ostatnią zapisaną wartość.
Jeżeli $W = 0$, to zapis jest dokonywany w pełni asynchronicznie.

\subsubsection*{Znaczenie}

Eventual Consistency opisuje problem dotykający każdej rozproszonej bazy danych, niezależnie od tego czy jest to baza relacyjna, sieciowy system plików czy baza NoSQL.
Tradycyjne podejście do konsystencji w kontekście replikacji jest ograniczające.
Systemy bazodanowe najczęściej implementują rozwiązanie typu wszystko-albo-nic: operacja zapisu musi się powieść na wszystkich węzłach albo zostać cofnięta.
Podejście takie nie tylko ogranicza dostępność systemu, powoduje ono także ograniczenie możliwości decyzyjnych autorów aplikacji korzystających z tych systemów.
Nawet jeżeli system umożliwia asynchroniczną replikację, zazwyczaj jest to bardzo prymitywny mechanizm, który nie bierze pod uwagę wersjonowania rekordów i sprowadza bazę do konfiguracji R=1, W=1.

Z drugiej strony systemy takie jak Amazon Dynamo pozwalają swoim użytkownikom na pełną dowolność w konfiguracji mechanizmów persystencji.
Nawet przy $R+W>N$ mamy możliwość sterowania zachowaniem systemu: czy zapisy powinny być szybsze kosztem odczytów, czy na odwrót, czy też może gdzieś pomiędzy.
Wiele z tych systemów zapewnia \emph{Read-your-writes Consistency} co stanowi dodatkowe ułatwienie.
Możliwość konfiguracji parametrów R, W i N stopniowo staje się standardem wśród systemów NoSQL obsługujących partycjonowanie danych.

\section{Google MapReduce}

Google MapReduce \cite{google-mapreduce} jest biblioteką wykorzystywaną do przetwarzania dużych zbiorów danych w środowisku rozproszonym.
Użytkownik biblioteki specyfikuje dwie funkcje nazywane \emph{map} i \emph{reduce} oraz kilka innych parametrów konfiguracyjnych.
Następnie biblioteka dba o to aby dane wejściowe zostały podzielone, na poszczególnych rekordach została wykonana funkcja \emph{map}, a jej wyniki zostały zagregowane przy pomocy funkcji \emph{reduce}.

Operacje \emph{map} i \emph{reduce} są powszechnie spotykane w językach funkcyjnych, takich jak na przykład LISP.
\emph{Map} na wejściu otrzymuje parę $(k, v): k \in K_1, v \in V_1$ a na wyjściu emituje listę par $(k, v): k \in K_2, v \in V_2$.
\emph{Reduce} na wejściu otrzymuje parę $(k, (v_1, ..., v_n)): k \in K_2, v_1...v_n \in V_2$ i na wyjściu emituje $v: v \in V_2$.

\subsection*{Przykład}

Poniżej przedstawiam przykładowy kod funkcji \emph{map} i \emph{reduce} w języku Python dla problemu zliczania wystąpień słów w zbiorze tekstów. 
Jak widzimy funkcja \emph{map} przyjmuje pary (klucz, wartość) z przestrzeni (Nazwy Dokumentów, Treść Dokumentów), zwracając pary z innej przestrzeni (Słowa, Liczby Naturalne).
Funkcja \emph{reduce} w przykładzie przyjmuje pary gdzie kluczem jest słowo, natomiast wartością jest lista liczb naturalnych określająca liczby wystąpień tego słowa w różnych dokumentach (albo wielokrotnie w tym samym dokumencie).

\begin{verbatim}
def map(document_name, document_value):
  """ funkcja mapujaca dokumenty na pary (slowo, 1)  """
  for word in words(document_value):
    yield (word, 1) # emit word

def reduce(word, counts):
  """ 
  funkcja redukujaca, przyjmuje slowo 
  i liste (iterator) po liczbie jego wystapien
  zwrace sumaryczna liczbe wystapien danego slowa
  """
  sum = 0
  for value in counts:
    sum += value
    # sum += 1
    # tak tez mozna byloby zapisac, 
    # ale wtedy funkcja nie bylaby laczna     
  return sum 
\end{verbatim}


\subsection*{Opis działania}

Użytkownik tworzy aplikację, w której specyfikuje dwie funkcje \emph{map} i \emph{reduce}, oraz dwa parametry konfiguracyjne: \emph{M} i \emph{R}:

\begin{itemize}
 \item \emph{M} - określa na ile części ma zostać podzielony plik wejściowy.
 Zazwyczaj wybiera się taką liczbę aby wielkość plików wejściowych zawierała się między 16MB a 64MB.
 Ponieważ GFS dzieli pliki na kawałki (ang. \emph{chunks}) wielkości 64MB, jest dość istotne aby pliki wejściowe nie przekraczały tej wielkości, gdyż w przeciwnym przypadku mogłaby występować konieczność komunikacji sieciowej w celu odczytania danych z pliku wejściowego.
 \emph{M} określa ponadto liczbę zadań \emph{map}.
 \item \emph{R} - określa liczbę zadań \emph{reduce}.
\end{itemize}

\myfigure{chapters/studium_literatury/map-reduce.png}{Google MapReduce}{fig:map-reduce}

\begin{enumerate}
 \item Biblioteka MapReduce dzieli plik wejściowy na \emph{M} części.
 \item Program użytkownika zostaje wysłany i uruchomiony na maszynach klastra.
 Jedna z tych maszyn przyjmuje specjalną rolę \emph{master}, pozostałe zaś mają rolę \emph{worker}\todo{może powinienem zamienić master na zarządca/nadzorca a worker na pracownik?}.
 \item \emph{master} przypisuje poszczególnym \emph{workerom} po jednym zadaniu do wykonania.
 Kolejne są przydzielane w miarę jak węzły kończą przydzieloną im pracę.
 Zadanie \emph{map} zostanie przydzielone w pierwszej kolejności maszynie która jest równocześnie \emph{chunkserverem} przechowującym odpowiedni plik wejściowy.
 Pozwala to uniknąć komunikacji sieciowej w celu odczytania pliku.
 \item \emph{Worker} przetwarza plik wejściowy rekord po rekordzie wywołując funkcję \emph{map} i zapisując jej wynik w pamięci.
 \item Co pewien czas dane zapisane w pamięci są zrzucane na dysk do plików lokalnych.
 W tym procesie dane są rozdzielane do \emph{R} plików poprzez funkcję partycjonującą, domyślnie $hash(key) mod R$.
 Lokacje tych plików są przekazywane do węzła \emph{master}, który z kolei jest odpowiedzialny za przekazanie ich do węzła wykonującego operację \emph{reduce} na odpowiednim fragmencie danych.
 \item Węzeł wykonujący operację \emph{reduce} po otrzymaniu takiego powiadomienia pobiera odpowiednie pliki bezpośrednio od węzła, który je przechowuje.
 Po otrzymaniu wszystkich potrzebnych plików, \emph{worker} sortuje otrzymane dane po kluczu, tak aby wartości dla danego klucza sąsiadowały ze sobą w pliku.
 \item Po posortowaniu plików wejściowych węzeł iteruje po kluczach i dla każdego z nich przekazuje klucz oraz listę wszystkich przypisanych mu wartości do funkcji \emph{reduce}, zapisując następnie jej wynik w pliku wyjściowym.
 \item Kiedy wszystkie operacje \emph{reduce} zakończą się, \emph{master} budzi program użytkownika i wywołanie funkcji \emph{MapReduce} kończy się.
\end{enumerate}

Wynikiem operacji jest \emph{R} plików wynikowych.
W większości przypadków konsumentem tych danych są inne operacje MapReduce, bądź aplikacje rozproszone, więc nie ma potrzeby łączenia tych plików w jedną całość.

\subsection*{Optymalizacje}

W artykule \cite{google-mapreduce} zostało opisanych kilka istotnych optymalizacji i ulepszeń:

\begin{enumerate}
 \item Operacja \emph{map} na pliku wejściowym jest wykonywana na tym samym serwerze, który przechowuje ten plik.
 \item Operacja \emph{map} może zwrócić bardzo wiele wartości dla danego klucza pośredniego.
 Z tego względu biblioteka wprowadza pojęcie funkcji łączącej (ang. \emph{combiner}).
 Funkcja ta dokonuje wstępnej redukcji przed wysłaniem wartości przez sieć do węzła wykonującego operację \emph{reduce}.
 Najczęściej stosuje się w tym miejscu tą samą funkcję co w operacji \emph{reduce}, ale aby to było możliwe, funkcja ta musi być łączna i przemienna, co czasem wymaga wprowadzenia pewnych zmian.
 Dla przykładu: przedstawiona wcześniej funkcja map emitująca wszystkie słowa w danym dokumencie emituje powtarzające się słowa wielokrotnie.
 Poprzez zsumowanie wystąpień przed przesłaniem znacznie zmniejszamy ilość danych do wysłania.
 \item Funkcja partycjonująca może być wyspecyfikowana przez użytkownika, dzięki czemu potencjalnie możliwe jest takie jej określenie, aby dane powiązane ze sobą znalazły się w jednym pliku wynikowym (aczkolwiek kosztem ryzyka nierównomiernego podziału kluczy między partycje).
 \item Dzięki sortowaniu kluczy pośrednich, pliki wynikowe są łatwiejsze do przetwarzania, umożliwiając na przykład wyszukiwanie binarne.
 \item Kiedy zbliża się koniec operacji, zadania \emph{reduce} są zlecane do wykonania przez dodatkowe węzły.
 Dzięki temu uszkodzone, nadmiernie obciążone przez inne procesy albo wadliwie skonfigurowane maszyny nie powodują nadmiernego wydłużenia całości operacji MapReduce.
\end{enumerate}

\subsection*{Odporność na awarie}

Ponieważ biblioteka MapReduce służy do wykonywania operacji w rozproszonym środowisku, często nawet na tysiącach węzłów, konieczne jest aby była ona odporna na awarie części z węzłów.
Rozróżniamy dwa typy awarii: awaria węzła \emph{master} i awarie węzłów typu \emph{worker}.

\subsubsection*{Awaria węzła master}

Ponieważ \emph{master} przechowuje między innymi informacje o zrealizowanych zadaniach i lokalizacji plików wynikowych.
\emph{Master} musi działać aby operacja MapReduce się zakończyła powodzeniem, ale możliwe jest aby jego struktury danych były zapisywane w GFS, albo replikowane do zapasowego węzła.
W opisanym systemie awaria węzła master zawsze kończy się niepowodzeniem całości operacji MapReduce i koniecznością jej ponownego uruchomienia.
Jest to dopuszczalne ponieważ czas trwania operacji jest liczony w minutach, więc awaria tego konkretnego węzła jest mało prawdopodobna (w odróżnieniu od np. BigTable, który jest systemem, który działa bez przerwy).

\subsubsection*{Awaria węzła slave}

\emph{Master} regularnie komunikuje się z węzłami \emph{slave} w celu ustalenia ich stanu.
W przypadku gdy węzeł przestaje odpowiadać jest uznawany za ,,martwy'', w związku z czym \emph{master} oznacza zadanie aktualnie przez niego wykonywane jako przeznaczone to przydziału, tak samo wszystkie wykonane przez niego zadania \emph{map}.
Zadania \emph{map} muszą być wykonane ponownie, ponieważ ich wyniki zostały zapisane w lokalnych plikach i przez to są niedostępne do odczytu.
Zadania \emph{reduce} zapisują swoje wyniki w GFS, w związku z tym nie muszą być powtarzane. 

\subsection*{Znaczenie}

Większość systemów opisanych w niniejszej pracy nie dysponuje zaawansowanymi metodami wykonywania zapytań, a w szczególności zapytań agregujących czy zliczających takich jak funkcje COUNT, SUM, AVG i operator GROUP BY w relacyjnych bazach danych.
Operacje tego typu są szczególnie trudne w kontekście systemów rozproszonych, gdzie wykonanie zapytań tego typu w trybie on-line przy zachowaniu odpowiedniego czasu odpowiedzi jest często wręcz niemożliwe.
Dlatego najczęściej wykonywanie tego typu operacji jest dokonywane co pewien czas, lub zlecane przy zapisie danych, a jego wyniki są przechowywane w systemie.

W przeważającej większości obliczenia te są wykonywane przy zastosowaniu algorytmu MapReduce.
Wiele systemów, tak jak MongoDB czy Riak, zapewnia taką funkcjonalność, inne wymagają zastosowania zewnętrznych narzędzi.
Wierną implementacją Google MapReduce jest Hadoop, który opiszemy nieco bliżej przy okazji Hbase.
Bardzo ciekawe podejście do MapReduce prezentuje CouchDB, która przechowuje wyniki funkcji map, dzięki czemu umożliwia dokonywanie zapytań opartych o MapReduce w trybie on-line.
CouchDB zostanie opisane bliżej w kolejnym rozdziale.

Algorytm MapReduce ma duże znaczenie w systemach NoSQL, gdyż stanowi surogat dla wykonywania skomplikowanych zapytań.
Stanowi on jedno z podstawowych narzędzi dla praktyki prostych odczytów, ale złożonych zapisów.

\section{Amazon Dynamo}

Amazon Dynamo to baza typu klucz-wartość używana w największym na świecie sklepie internetowym \emph{amazon.com}.
Baza została opisana w artykule z 2007 roku \cite{amazon-dynamo} i od tego czasu opisany w nim system doczekał się już dwóch implementacji open-source: Dynomite i Riak, oraz stał się inspiracją dla innych baz takich jak Cassandra, czy rozszerzenia CouchDB pozwalającego na rozpraszanie tej bazy.

Amazon Dynamo skaluje się bardzo dobrze - do setek węzłów, co dzięki zastosowaniu architektury opartej na usługach w zupełności wystarcza nawet takiemu gigantowi jak Amazon.
Interesującym aspektem architektury tego sklepu internetowego jest to, że poszczególne usługi muszą oferować gwarancje na czas wykonania (ang. \emph{Service Level Agreements}).
Przykładowym kontraktem tego typu jest gwarancja, że usługa zwróci odpowiedź w ciągu 300ms dla 99,9\% zapytań przy obciążeniu 500 zapytań na sekundę.
Konsekwencją istnienia takich gwarancji jest to, że Dynamo musi udostępniać wiele możliwości konfiguracji, tak aby usługi mogły dostosować właściwości bazy na potrzeby oferowanego kontraktu.

Jednym z wymagań dla systemu obsługującego sklep internetowy jest to aby użytkownik był w stanie zawsze dodać przedmioty do koszyka czy złożyć zamówienie.
W związku z tym Amazon Dynamo jest systemem w którym dostępność (A w CAP) odgrywa kluczową rolę, a odporność na podział sieci (P w CAP) jest jej uzupełnieniem.
Wydawało by się, że konsystencja ma duże znaczenie w przypadku sklepu internetowego - nie można przecież sprzedać towaru, którego się nie ma, a przecież w przypadku sklepu z którego korzystają miliony użytkowników równocześnie warunki wyścigu (ang. \emph{race conditions}) muszą występować nagminnie.
W amazon.com dopuszczalne jest aby dwóch klientów zamówiło ostatnią z książek w magazynie, a kiedy okaże się że jeden z nich jej nie może otrzymać sklep kontaktuje się z nim proponując na przykład dłuższy okres dostawy.
Nawet jeżeli nie wszyscy użytkownicy się zgodzą z taką sytuacją, to i tak przekłada się to na wyższe przychody.
Pokazuje to jak Eventual Consistency może z powodzeniem być stosowane w sytuacjach, gdzie normalnie kładzie się duży nacisk na spójność danych, nie powodując przy tym ujmy dla systemu.

\subsection*{Główne założenia architektury}

Autorzy Amazon Dynamo określili takie założenia co do architektury:

\begin{itemize}
 \item system jest zawsze w stanie przyjąć operację zapisu
 \item w związku z replikacją danych i naciskiem na wysoką dostępność, w systemie może wystąpić wiele wersji rekordu
 \item rozwiązywanie konfliktów między wersjami następuje przy odczycie i jest dokonywane przez aplikację, nie przez bazę (ale aplikacja może z tego zrezygnować i wybrać prosty mechanizm, np. ostatni zapis wygrywa)
 \item system musi się łatwo skalować wszerz poprzez dodawanie kolejnych węzłów i nie powinno to mieć dużego wpływu ani na administrację systemem ani na sam system
 \item system musi się charakteryzować symetrią: każdy węzeł ma taki sam zestaw obowiązków jak wszystkie inne
 \item system musi być zdecentralizowany: żaden węzeł nie jest wyróżniony i wszelkie operacje są wykonywane korzystając z mechanizmów \emph{peer-to-peer}
 \item system musi radzić sobie z heterogenicznością środowiska: narzut pracy na pojedynczy węzeł powinien uwzględniać jego możliwości w porównaniu do innych węzłów
\end{itemize}

\subsection*{Stosowane techniki}

Amazon Dynamo łączy w sobie wiele ciekawych technik rozwiązujących problemy takie jak partycjonowanie, wersjonowanie rekordów czy wykrywanie awarii.
W niniejszym rozdziale postaram się je czytelnikowi przybliżyć.

\subsubsection*{Partycjonowanie - Consistent Hashing} 

W systemie takim jak Dynamo, który ma za zadanie skalować się do setek węzłów, horyzontalne partycjonowanie jest nieuniknione.
W Amazon Dynamo za podział na partycje odpowiada zmodyfikowany algorytm Consistent Hashing.
W następnych kilku paragrafach opiszę najpierw podstawowy algorytm, a następnie kolejne jego modyfikacje.

Consistent Hashing to algorytm który dla dowolnego klucza określa który węzeł systemu go przechowuje.
Ponieważ klucze są dowolnymi ciągami bajtów, zawsze operując na nich używamy funkcji mieszającej (ang. \emph{hashing function}).
Przeciwdziedzinę tej funkcji mieszającej traktujemy jako pierścień gdzie najmniejsza i największa możliwa wartość niejako stykają się ze sobą, podobnie jak ma to na przykład miejsce w przypadku długości geograficznych czy godzin na tarczy zegara.
W podstawowej wersji systemu każdy węzeł systemu losuje jedną wartość na tym pierścieniu.
Aby określić któremu węzłowi odpowiada dany klucz, obliczamy dla niego wartość funkcji mieszającej, a następnie znajdujemy węzeł, którego ,,pozycja'' na pierścieniu jest najbliższą zgodnie z ruchem wskazówek zegara po wartości wyliczonej.
W takim schemacie każdemu węzłowi odpowiada jeden, ciągły fragment pierścienia (patrz rysunek \ref{fig:consistent-hashing-01}).

\myfigure{chapters/studium_literatury/consistent-hashing-01.png}{Podstawowe Consistent Hashing}{fig:consistent-hashing-01}

Podstawowy wariant algorytmu ma wiele wad.
Problemem jest na przykład potencjalnie bardzo nierównomierna dystrybucja kluczy między węzłami.
Nie ma też żadnego wsparcia dla heterogeniczności środowiska - nie możemy ,,mocniejszemu'' serwerowi przekazać większego zakresu kluczy.
W związku z tymi ograniczeniami, wprowadza się tak zwane ,,wirtualne węzły'': pojedynczemu fizycznemu węzłowi przypisuje się wiele pozycji na pierścieniu (zazwyczaj liczba tych pozycji jest znacznie większa od ogólnej liczby węzłów).
Dzięki tej modyfikacji (o ile liczba wirtualnych węzłów jest wystarczająco duża) dystrybucja kluczy będzie znacznie bardziej sprawiedliwa, możemy też kontrolować obciążenie poszczególnych węzłów poprzez przyporządkowanie im mniejszej lub większej liczby pozycji na pierścieniu (patrz rysunek \ref{fig:consistent-hashing-02}).

\myfigure{chapters/studium_literatury/consistent-hashing-02.png}{Consistent Hashing z wirtualnymi węzłami}{fig:consistent-hashing-02}

Opisana powyżej wersja algorytmu radzi sobie zbyt dobrze z dodawaniem i usuwaniem węzłów niż wersja podstawowa, jednak w rzeczywistości nie sprawdza się zbyt dobrze.
Przy dodaniu węzła do systemu, klucze znajdujące się do tej pory pod opieką innych węzłów muszą być przekazane nowemu węzłowi.
Aby tego dokonać, konieczne jest dokonanie przeglądu wszystkich kluczy w dzielonym zakresie, konieczne jest także przeliczenie na nowo i synchronizacja drzew Merkle (vide \emph{Merkel Trees} na stronie \pageref{merkle-trees}). 
Przegląd taki obciąża system i musi być dokonywany w tle aby nie złamać gwarancji dawanych przez korzystającą z Dynamo usługę.
Aby uniknąć tego problemu należy rozdzielić mechanizm partycjonowania danych (w jaki sposób są tworzone przedziały) od tego które węzły nimi zarządzają.

W tym wariancie algorytmu przestrzeń kluczy zostaje podzielona z góry na określoną liczbę przedziałów S.
Najczęściej liczba ta będzie znacznie większa niż liczba węzłów systemu N ($S >> N$).
Następnie każdy z przedziałów przydzielamy losowo jednemu z węzłów tak aby każdy miał ich po $S/N$\footnote{oczywiście ten mechanizm może być dostosowany do wymagania heterogeniczności, tak aby jedne węzły otrzymywały więcej przedziałów niż inne}.
Dzięki temu rekordy przynależące do różnych z przedziałów mogą być przechowywane w odrębnych plikach, które z kolei mogą być przesyłane w ramach potrzeby do przejmujących nad przedziałem kontrolę węzłów bez zbędnego przeglądania danych na dysku.
Rysunek \ref{fig:consistent-hashing-03} przedstawia opisaną powyżej wersję algorytmu.

\myfigure{chapters/studium_literatury/consistent-hashing-03.png}{Consistent Hashing z rozdziałem partycjonowania i pozycjonowania rekordów}{fig:consistent-hashing-03}

\subsubsection*{Replikacja}

Mechanizm opisany w poprzednim rozdziale przydziela przedziałów kluczy do węzłów, które są \emph{koordynatorami} tych przedziałów.
Oczywiście nie wystarcza aby każdy przedział był zapisany tylko na dysku koordynatora - awaria pojedynczego węzła powodowałaby wtedy (potencjalnie nieodwracalne) straty danych, a przynajmniej ich czasową niedostępność.
W Amazon Dynamo dane są replikowane do N fizycznych węzłów, gdzie N\footnote{
Uważny czytelnik zwróci uwagę na to, że ta sama nomenklatura była wykorzystana w artykule o Eventual Consistency.
W rzeczy samej, autor tego artykułu Werner Vogels (CTO amazon.com) figuruje na także na liście autorów artykułu o Amazon Dynamo.
Obie prace są ze sobą blisko związane i dobrze uzupełniają się.
}
jest parametrem określanym w konfiguracji bazy.
Najczęściej stosowaną wartością N jest 3.

Klucze są zapisywane na dysku koordynatora, oraz na dyskach N-1 kolejnych węzłów.
Kolejne węzły to węzły, które są koordynatorami kolejnych w kierunku ruchu wskazówek zegara przedziałów pierścienia.
Ponieważ może zajść sytuacja, że koordynatorem kolejnych przedziałów jest ten sam węzeł, poruszamy się wzdłuż pierścienia aż znajdziemy N różnych węzłów.
Lista węzłów odpowiedzialnych za przechowywanie klucza jest nazywana \emph{listą preferencyjną}.
Lista ta zazwyczaj zawiera więcej niż N adresów, na wypadek awarii jednego lub więcej z N podstawowych węzłów.

\subsubsection*{Wersjonowanie - Zegary Wektorowe}

Amazon Dynamo jest systemem, w którym dane są replikowane asynchronicznie i który zachowuje dostępność nawet w wypadku podziałów sieci, czy awarii poszczególnych węzłów.
Konsekwencją tego podejścia jest możliwość występowania wielu wersji rekordu równocześnie w systemie, a w przypadku długotrwałych podziałów sieci możliwe jest nawet istnienie równoległych ,,drzew historii'' dla tego rekordu.

Podobnie jak wcześniej i tu widzimy wyraźny wpływ tego, jak wymagania biznesowe wpłynęły na wybór algorytmów i architektury Dynamo.
Jednym z rozdzajów danych które są przechowywane w tym systemie jest koszyk użytkownika.
Ze względów finansowych jest istotne aby użytkownik nie tylko mógł zawsze dodać nowe przedmioty do swojego koszyka, ale także aby informacja o dodaniu tych przedmiotów nigdy nie została utracona.
Gdyby zatem do wersjonowania rekordów w systemie użyto znaczników czasowych, nie byłoby możliwe określenie logicznej kolejności modyfikacji danego rekordu.
Wykorzystanie startegii ,,nowszy wygrywa'' do rozwiązywania ewentualnych konfliktów mogłoby grozić utratą zamówień z koszyka, a zatem negatywnymi konsekwencjami finansowymi dla firmy.

Autorzy Amazon Dynamo zdecydowali się na wykorzystanie mechanizmu nazywanego zegarami wektorowymi (ang. \emph{vector clocks}).
Zegar wektorowy jest to zbiór par, gdzie pierwszym elementem pary jest identyfikator węzła\footnote{Riak, opisany w rozdziale \ref{sec:riak}, używa identyfikatorów klientów jako pierwszych elementów pary. Użycie identyfikatorów serwerów wiąże się bowiem z ryzykiem utraty danych. Problem ten jest opisany bliżej w \cite{basho-vector-clocks-hard}.}, a drugim elementem jest liczba naturalna.
Przy każdej modyfikacji rekordu serwer dokonujący tej modyfikacji zwiększa o 1 liczbę przypisaną do swojego identyfikatora w zegarze wektorowym (albo wpisuje swój identyfikator z wartością 1, jeżeli go wcześniej nie było).
Jeżeli wszystkie elementy zegara A mają wszystkie wartości odpowiednich elementów mniejsze bądź równe wartościom wektora B, to mówimy że A jest przodkiem wektora B i w przypadku napotkania takich dwóch wersji B zastępuje A.
Jeżeli natomiast ani A nie jest przodkiem B ani vice versa, to mówimy że te dwie wersje rekordu znajdują się w konflikcie, który z kolei musi zostać rozwiązany przez aplikację (klienta).

Zegar wektorowy rekordu o długim czasie życia w systemie może osiągnąć potencjalnie nieograniczoną długość.
Aby temu zapobiec do pary identyfikator węzła-liczba, dodaje się jeszcze trzeci element: znacznik czasu (ang. \emph{timestamp}), który zapisuje kiedy ostatni raz dany węzeł modyfikował wartość rekordu.
Długość zegara można dzięki temu kontrolować poprzez usuwanie najstarszych elementów, kiedy długość zegara przekroczy pewien próg (10 dla Dynamo).
Teoretycznie utrata tych informacji grozi wystąpieniem konieczności rozwiązywania konfliktów między wersjami które w rzeczywistości są w relacji następstwa, ale zdarzenie takie jest wysoce nieprawdopodobne i nie wiąże się z dużymi trudnościami dla aplikacji.

Bardzo dobry opis zegarów wektorowych czytelnik może znaleźć w artykułach na blogu firmy Basho, która to firma zajmuje się rozwojem bazy Riak: \cite{basho-vector-clocks-easy} opisuje tą technikę z punktu widzenia klienta, zaś \cite{basho-vector-clocks-hard} opisuje ją z punktu widzenia implementacji serwera.

\subsubsection*{Operacje odczytu i zapisu}

Jako baza typu klucz-wartość, Amazon Dynamo udostępnia użytkownikowi dwie podstawowe operacje: \emph{put} i \emph{get}.
Operacja \emph{get} przyjmuje między innymi parametr R, określający minimalną liczbę węzłów które muszą dokonać operacji odczytu aby zakończyła się ona sukcesem.
Operacja \emph{set} jako jeden z parametrów przyjmuje z kolei wartość W, określającą ile węzłów musi potwierdzić zapis wartości aby operacja się powiodła.

Amazon Dynamo używa protokołu HTTP w komunikacji między klientem i serwerem.
Zaletą tego podejścia jest brak konieczności linkowania aplikacji z kodem służącym do komunikacji z Amazon Dynamo, wadą jednak jest to, że klient który traktuje bazę jako ,,czarną skrzynkę'' jest wolniejszy.
Jeżeli klient nie jest świadomy istnienia list preferencyjnych i informacji w nich zawartych, operacja odczytu przebiega następująco: 
\begin{enumerate}
 \item Klient wysyła zapytanie, które jest obsługiwane przez \emph{load balancer}.
 \item \emph{Load balancer} może wybrać węzeł, do którego ma przekazać zapytanie, albo poprzez inspekcję przekazywanego zapytania i wybór węzła na podstawie listy preferencyjnej, albo losowo z pośród wszystkich węzłów systemu.
 \item Jeżeli węzeł, do którego trafiło polecenie zapisu nie znajduje się na jednym z N pierwszych miejsc listy preferencyjnej, przekazuje on zapytanie dalej do pierwszego, nie podlegającego aktualnie awarii węzła na tej liście.
 Nie dzieje się tak w przypadku poleceń odczytu, które mogą być koordynowane przez dowolny węzeł, gdyż nie muszą modyfikować zegarów wektorowych powiązanych z rekordami.
 \item Węzeł ten wykonuje żądaną operację, oraz przekazuje ją do wykonania do N-1 innych węzłów znajdujących się na szczycie listy preferencyjnej.
 \item Kiedy węzeł otrzyma potwierdzenie od $R-1$ bądź $W-1$ z tych węzłów, operacja kończy się powodzeniem.
 Jeżeli w systemie występują wersje rekordu, które znajdują się w konflikcie, wszystkie one zostaną przekazane do klienta.
 Nieaktualne wersje rekordów zostaną wykryte i automatycznie uaktualnione w sposób przeźroczysty dla klienta.
\end{enumerate}

Jak widać w powyższym schemacie, klient znający listy preferencyjne systemu jest w stanie całkowicie pominąć \emph{load balancer} i ewentualne przekazanie zapytania do jednego z węzłów na liście preferencyjnej klucza. 
W przypadku gdy Dynamo posługuje się znacznikami czasu do wersjonowania rekordów, klient może przejąć na siebie także obowiązki koordynatora zapytań.
W przypadku wykorzystania zegarów wektorowych nie jest to możliwe, gdyż w zegarze wektorowym w Dynamo jest zapisywany identyfikator węzła systemu, nie klienta (tak jak to ma miejsce w Riak).
Koordynacja zapytań przez klienta zmniejsza opóźnienia (ang. \emph{latency}) w systemie o ponad połowę, zarówno dla wartości średnich, jak i dla 99.9\% zapytań.

\subsubsection*{Obsługa awarii - Hinted Handoff}

W poprzednich rozdziałach opisane zostały listy preferencyjne oraz mechanizm koordynacji operacji zapisu i odczytu.
Przypomnijmy, że Amazon Dynamo na liście preferencyjnej zawiera więcej węzłów, niż liczba N podana przez użytkownika, a w razie awarii części węzłów przechowujących dany zakres kluczy, klient będzie kierował swoje zapytania do węzłów znajdujących się dalej na liście preferencyjnej, pomijając te, które uległy awarii.
Informacja o tym, że rekord miał trafić do innego węzła, ale ze względu na jego awarię trafił do węzła aktualnego jest zapisywana w metadanych tego rekordu.
Rekordy takie są przechowywane osobno od reszty rekordów i regularnie jest dokonywane sprawdzenie, czy można już taki rekord przekazać do węzła, na który miał on trafić oryginalnie.
Technikę tę nazywamy \emph{Hinted Handoff}, czyli w wolnym tłumaczeniu przekazywanie z sugestiami.
W Amazon Dynamo operacje dołączenia, czy usunięcia fizycznego węzła są jawne, stąd też przyjmujemy że węzły stają się niedostępne jedynie tymczasowo, w przypadku permanentnej awarii węzeł powinien zostać jawnie usunięty.

\subsubsection*{Wykrywanie różnic - Merkle Trees}
\label{merkle-trees}

W Amazon Dynamo każdy węzeł jest odpowiedzialny za przechowywanie pewnej liczby zakresów kluczy.
Węzły odpowiedzialne za określony zakres kluczy nazywamy jego replikami.
Repliki mogą zawierać różne wersje rekordów i aby zmniejszyć ryzyko utraty danych należy co jakiś czas dokonywać synchronizacji danych między nimi.
W tym celu stosuje się drzewa Merkle (ang. \emph{Merkle Trees}).
Drzewo takie jako liście posiada sumy kontrolne wartości dla wszystkich kluczy danego przedziału, wszystkie pozostałe węzły drzewa zawierają sumę kontrolną ze swoich dzieci.
Dzięki temu, że dwa drzewa o tej samej wartości w korzeniu są identyczne, możliwe jest porównanie zakresów kluczy przechowywane przez różne repliki w bardzo efektywny sposób, znacznie ograniczając ilość danych do przesłania.

\subsubsection*{Optymalizacje}

Podobnie jak wiele innych wysokowydajnych baz danych\footnote{MongoDB, Riak, Redis}, w celu zmniejszenia opóźnień w Amazon Dynamo zapisy mogą być przechowywane tymczasowo w pamięci i zrzucane na dysk w tle co jakiś czas.
Aby zapewnić trwałość danych Dynamo dba o to aby przynajmniej jeden węzeł z N zapisywał dane od razu na dysku.
Ponieważ jednak W jest zazwyczaj mniejsze od N, nie zmniejsza to szybkości działania.

Inną wartą uwagi optymalizacją, jest taka konstrukcja list preferencyjnych aby każdy zakres kluczy był przechowywany na węzłach znajdujących się w co najmniej dwóch centrach danych.
Dzięki temu w razie awarii całego centrum danych system jest w stanie zachować dostępność.

\subsubsection*{Znaczenie}

Amazon Dynamo jest bardzo ciekawym przykładem rozproszonego systemu bazodanowego.
Jest jednym z nielicznych systemów, w których wszystkie węzły są sobie równe.
Jest też warty uwagi ze względu na niezwykle wysokie gwarancje dostępności jakie oferuje i jakie ma to konsekwencje dla konsystencji.
Dynamo doczekało się naśladowców w postaci baz Riak i Dynomite, oraz zainspirowała twórców Apache Cassandra i programistów firmy Cloudant tworzących rozproszoną wersję CouchDB.
Artykuł ,,Dynamo: amazon's highly available key-value store'' \cite{amazon-dynamo} jest ponadto jednym z najciekawszych i najbardziej przystępnych artykułów dotyczących zagadnień związanych ze skalowalnymi bazami danych znanych autorowi tej pracy i warty polecenia wszystkim osobom zainteresowanym ruchem NoSQL.

\section{Google BigTable}

W kolejnym rozdziale opiszę Google BigTable \cite{google-bigtable} - bazę stosowaną w Google, przewyższającą Amazon Dynamo o rząd wielkości pod względem możliwej liczby węzłów, opartą na rozproszonym systemie plików GFS.
Google BigTable jest przykładem bazy CA (\emph{Consistent, Avaliable}).
W rozdziale opisującym dostępne systemy opiszę Hbase - bazę silnie wzorowaną na BigTable, ale opartą na Hadoop Core zamiast GFS, oraz Apache Cassandra, która łączy w sobie cechy Amazon Dynamo i Google BigTable.

\chapter{Klasyfikacja rozwiązań}

\section*{Streszczenie}

Jednym z założeń ruchu NoSQL jest zwrócenie uwagi użytkowników systemów bazodanowych na fakt, że dla różnych aplikacji i do różnych zastosowań mogą pasować zupełnie różne systemy.
Ogromna liczba nierelacyjnych baz danych\footnote{Strona http://nosql-database.org/ wymienia ponad stu takich baz.} sprawia, że niemożliwym jest opisanie wszystkich z nich w niniejszej pracy.
Z tego względu przedstawionych zostanie jedynie kilka najlepiej znanych i najbardziej popularnych rozwiązań.

Istnieje wiele możliwych sposobów klasyfikacji systemów NoSQL.
Na potrzeby tej pracy przyjęto najczęściej stosowany system, który dzieli bazy ze względu na wykorzystywany model danych.

\section{Podział ze względu na model danych}

\begin{description}
 \item[Bazy typu klucz-wartość]
 Systemy tego typu modelują dane jako tablicę asocjacyjną.
 W takiej tablicy unikalnemu kluczowi jest przyporządkowana wartość.
 Zazwyczaj ani klucz, ani jego wartość nie są interpretowane przez system.
 Bazy tego typu udostępniają zazwyczaj jedynie prosty interfejs pozwalający na odczytanie lub ustawienie wartości klucza bądź jego usunięcie; bardziej skomplikowane zapytania są zazwyczaj niemożliwe.
 
 \item[Bazy kolumnowe]
 Systemy te wzorują swój model danych na Google BigTable (patrz strona \pageref{google-bigtable-model-danych}).
 W bazie kolumnowej unikalnemu kluczowi odpowiada wiele kolumn.
 Kolumny są grupowane w rodziny - zazwyczaj wszystkie kolumny należące do tej samej rodziny są jednego typu.
 W przypadku Apache Cassandra kilka rodzin kolumn może być zebranych w super-rodzinę.
 Kolumny przynależące do jednej rodziny z reguły zawierają dane jednego typu, gdyż są one kompresowane razem.
 Bazy te zazwyczaj wymagają zdefiniowania struktury bazy z dokładnością do rodzin, czy super-rodzin kolumn.
 Jedynie kolumny, którym jest przypisana jakaś wartość dla danego klucza zajmują miejsce w pamięci, dzięki temu różnych kolumn może być bardzo wiele i często służą na przykład do tworzenia list wartości.
 Bazy kolumnowe pozwalają jedynie na wykonywanie zapytań o pojedynczy klucz lub ich zakres, ale za to oferują szeroki zakres możliwości limitowania zwracanych kolumn.

 \item[Bazy dokumentowe]
 W systemie dokumentowym unikalnemu kluczowi przypisany jest tak zwany dokument.
 Dokument jest to drzewo, którego liśćmi są wartości prymitywne (np. ciąg znaków, liczba, wartość logiczna, pusty obiekt NULL, itp.) a węzłami wartości złożone (tablica, tablica asocjacyjna).
 Dokumenty są zazwyczaj zgrupowane w kolekcje, po których można iterować.
 Dokumenty należące do tej samej kolekcji nie muszą mieć tych samych pól czy struktury.
 W odróżnieniu od systemu typu klucz-wartość przechowującego dane w formacie JSON, systemy dokumentowe są w stanie interpretować wartości zawarte w dokumentach i umożliwiają użytkownikowi efektywne wyszukiwanie rekordów na podstawie tych wartości przy pomocy różnych rodzajów indeksów.
 W bazie dokumentowej nie występują zazwyczaj relacje między dokumentami, ale możliwe jest zagnieżdżanie dokumentów, co ułatwia modelowanie relacji jeden-do-wielu.
 
 \item[Bazy grafowe]
 Głównym wyróżnikiem grafowych baz danych jest to, że baza grafowa pozwala w złożoności O(1) uzyskać informacje o relacjach danego węzła z innymi węzłami.
 W takim systemie można uzyskać informacje zarówno o krawędziach wchodzących jak i wychodzących.
 Zarówno węzły, jak i krawędzie grafu są opisane właściwościami.
 Podobnie jak w przypadku baz dokumentowych węzły czy krawędzie nie mają z góry narzuconej struktury i mogą się różnić liczbą, typem i nazwami właściwości.
 Podstawowym mechanizmem zapytań w bazie grafowej są rożne mechanizmy trawersowania grafu biorąc pod uwagę właściwości węzłów i krawędzi i poruszając się po krawędziach wchodzących i wychodzących.
 Drugorzędnym mechanizmem jest wyszukiwanie węzłów i krawędzi grafu na podstawie ich właściwości.
 Możliwość wykonywania takich zapytań nie jest jednak konieczna w bazie grafowej i niektóre produkty pozostawiają obsługę takich zapytań innym systemom. 

\end{description}

\section{Opisane aspekty rozwiązań}

Ponieważ dokładne przedstawienie dowolnego systemu bazodanowego na kilku stronach A4 jest zadaniem nie do wykonania, zamierzeniem autora tej pracy było wybranie najbardziej istotnych i cennych dla czytelnika informacji.
Pominięte zostały przede wszystkim szczegóły dotyczące API poszczególnych systemów oraz łatwo dostępne informacje, takie jak instrukcja instalacji systemu.
Opisane zostały natomiast istotne cechy systemów, które pozwalają czytelnikowi lepiej zrozumieć mechanizm działania aplikacji.
Każda baza została opisana według następującego schematu:

\begin{description}
 \item[Wstęp] 
 Każdą sekcję poprzedzono wstępem przybliżającym w kilku zdaniach opisywaną bazę. 
 
 \item[Protokół komunikacji]
 Opisuje dostępne protokoły komunikacji z bazą i wsparcie dla różnych języków programowania.

 \item[Replikacja]
 Opisuje dostępne mechanizmy replikacji systemu i możliwości ich konfiguracji.

 \item[Partycjonowanie]
 W przypadku baz obsługujących horyzontalne partycjonowanie danych opisany zostanie ten mechanizm i jego możliwości konfiguracji.
 W przypadku baz nie dających takiej możliwości opisane zostaną narzędzia ją zapewniające.

 \item[Persystencja]
 Opisany zostanie mechanizm trwałego zapisu danych oraz różne opcje konfiguracji systemu z tym związane i w przypadku systemów, które obsługują wymienne moduły persystencji, zostaną one opisane.

 \item[Wersjonowanie]
 W przypadku systemów które oferują taką opcję, zostanie opisany mechanizm wersjonowania rekordów i rozwiązywania konfliktów.

 \item[Wyszukiwanie]
 Opisane zostaną ogólnie opcje tworzenia zapytań do systemu i indeksowania danych.

 \item[Unikalne cechy]
 Przedstawione zostaną cechy systemu, które wyróżniają go na tle konkurencji.

 \item[Typowe zastosowania]
 Opisane zostaną typowe zastosowania i znane wdrożenia danego systemu.

 \item[Przeciwwskazania]
 Opisuje typy zastosowań, z którym system może mieć problemy, szczególnie takie, które mogą nie występować przy prostej ewaluacji.

 \item[Dokumentacja i wsparcie]
 Opisany zostanie poziom aktywności społeczności związanej z daną bazą, możliwości komercyjnego wsparcia, subiektywna ocena jakości dokumentacji oraz w przypadku produktów rozwijanych przez konkretnych dostawców, zostaną oni przedstawieni.

 \item[Pomocne odnośniki]
 Każdy opis zostanie zakończony garścią odnośników pomocnych w zapoznawaniu się z systemem.

\end{description} 
\chapter{Bazy Klucz-Wartość}

\section*{Streszczenie}
W tym rozdziale opisane zostaną dwie bazy klucz-wartość: Redis i Riak.
\todo{Rozszerzyć wstęp}

\section{Redis}
\label{sec:redis}

\subsection*{Wstęp} 

Redis jest jedną z najbardziej popularnych baz NoSQL.
Swoją popularność zawdzięcza on niezwykle bogatemu jak na bazę typu klucz-wartość API, oraz bardzo dużej wydajności\footnote{Wraz z instalacją systemu dostępne jest narzędzie redis-benchmark, które to dla komputera klasy PC pokazało wydajność ponad 140 tysięcy operacji GET i SET na sekundę.}.
Bardzo często system ten jest wykorzystywany w aplikacjach internetowych równolegle z relacyjnymi bazami danych, najczęściej jako kolejka zadań albo w zastępstwie Memcached jako cache oraz baza, w której przechowywana jest sesja użytkownika.

Mimo, że jest on klasyfikowany jako baza typu klucz-wartość, Redis nie jest horyzontalnie skalowalny.
Nie jest to zazwyczaj bardzo dużym problemem, ponieważ Redis jest w większości przypadków wystarczająco wydajny aby jeden węzeł był w stanie obsłużyć wszystkie zapytania generowane przez aplikację.

\subsection*{Protokół komunikacji}

Redis używa bardzo prostego, tekstowego protokołu komunikacji.
Istnieje możliwość autentykacji, ale tylko przy pomocy hasła zapisanego w konfiguracji serwera.
Nie ma możliwości tworzenia wielu użytkowników i określania ich uprawnień.
Ponieważ hasło jest przesyłane otwartym tekstem, a połączenia nie są w żaden sposób szyfrowane, Redis nie należy do najbezpieczniejszych rozwiązań na rynku.

Dzięki prostemu protokołowi, biblioteki do komunikacji z tym systemem są dostępne praktycznie dla każdego języka programowania.
Dla niektórych języków programowania dostępne są także biblioteki wyższego poziomu, które pełnią rolę bardzo podobną do bibliotek ORM (Object-Relational Mapping).
Biblioteki takie jak Ohm dla języka Ruby pozwalają na mapowanie obiektów na strukturę kluczy w Redis, włącznie z indeksowaniem wartości niektórych pól, co umożliwia wyszukiwanie rekordów po czym innym niż wartość klucza. 

\subsection*{Replikacja}

Redis posiada mechanizm replikacji w trybie master-slave.
Replikacja jest asynchroniczna, co oznacza, że dane odczytane z serwerów slave mogą być czasem nieaktualne.
Często Redis jest konfigurowany w taki sposób, aby serwer master w ogóle nie dokonywał zapisów na dysk twardy, tylko pozostawiał tą rolę serwerowi lub serwerom slave.
Rozwiązanie takie pozwala na zwiększenie wydajności.

\subsection*{Partycjonowanie}

Redis nie obsługuje partycjonowania w wersji 2.0.
Niektóre biblioteki pozwalają wprawdzie na implementację partycjonowania po stronie klienta, ale są to bardzo prymitywne rozwiązania, które nie mogą w żaden sposób konkurować z partycjonowaniem po stronie serwera, jakie jest zaimplementowane np. w omawianym w kolejnym rozdziale systemie Riak.

W przypadku systemów typu klucz-wartość samodzielna implementacja partycjonowania i replikacji kluczy jest zazwyczaj najprostsza z pośród wszystkich typów nierelacyjnych baz danych.
Redis jednak obsługuje złożone typy danych, których wielkość dla pojedynczego klucza może potencjalnie przekroczyć wielkość dostępnej pamięci RAM, a których nie da się w żaden prosty sposób podzielić na wiele węzłów.
Dlatego dopóki nie zostanie zaimplementowany \emph{Redis Cluster}, systemowi temu będzie daleko do konkurencji pod względem skalowalności.

\subsubsection*{Redis Cluster}

W planach jest implementacja tak zwanego \emph{Redis Cluster}, który pozwoli na horyzontalne skalowanie systemu poprzez replikację i partycjonowanie danych.
Redis Cluster będzie systemem typu CA (Consistent-Available).
W systemie będą występować cztery rodzaje węzłów:

\begin{description}
 \item[Data Node] (węzeł danych) - węzły przechowujące dane.
 Skalowanie systemu odbywa się poprzez zwiększanie liczby tych węzłów.

 \item[Configuration Node] (węzeł konfiguracyjny) - węzeł przechowujący metadane o całym systemie, takie jak listy \emph{Data Node} i \emph{Proxy Node}.
 W razie awarii tego węzła, system wprawdzie nie zaprzestaje normalnej pracy, ale nie jest w stanie obsłużyć sytuacji wyjątkowych takich jak awaria innego węzła.
 Ten węzeł tylko przechowuje dane konfiguracyjne, węzłem odpowiedzialnym za ich obsługę jest \emph{Handling Node}.

 \item[Proxy Node] (węzeł proxy) - węzły odpowiedzialne za koordynację zapytań w systemie.
 Zapytania w systemie są zawsze kierowane do węzłów Proxy, które przekazują je następnie do jednego (w przypadku odczytów) lub wszystkich (w przypadku operacji zmieniających dane) węzłów, na których odpowiednie rekordy są zapisane.
 Awarie są wykrywane przez \emph{Proxy Node} i informacja o nich jest zapisywana w węźle konfiguracyjnym.

 \item[Handling Node] (węzeł zarządzający) - to klient, który obsługuje dane zapisane w węźle konfiguracyjnym.
 Zajmuje się on zmianą przydziału zakresów kluczy do węzłów w razie dodania lub usunięcia węzła, lub jego awarii.
 Najczęściej \emph{Handling Node} i \emph{Configuration Node} powinny być umieszczone na tym samym fizycznym węźle.
\end{description}

\subsection*{Persystencja}

\subsubsection*{Snapshotting}

Redis zawdzięcza swoją szybkość temu, że zarówno operacje zapisu, jak i operacje odczytu nie muszą operować na danych zapisanych na dysku twardym, tylko na danych w pamięci RAM.
Domyślnie Redis pracuje w trybie zapisywania zrzutów aktualnego stanu (ang. \emph{snapshotting}) co pewien czas w zależności od liczby wykonanych operacji.
Typową konfiguracją w tej opcji jest zapis danych na dysk co 60 sekund jeżeli zostało dokonanych przynajmniej 100 zmian, albo do 1000 sekund, jeżeli została wykonana co najmniej jedna zmiana.

\subsubsection*{Append Only File}

Alternatywą dla periodycznego zrzucania stanu bazy na dysk jest \emph{Append Only File} (plik tylko do dopisywania).
Technika ta polega na dopisywaniu zmian w bazie na koniec pliku, przypominającego dziennik transakcji.
Przy starcie systemu plik ten jest odczytywany i zmiany w nim zawarte są aplikowane po kolei aby przywrócić stan bazy.
Oczywiście plik ten rośnie z czasem, co powoduje także wydłużenie czasu startu systemu, dlatego dostępna jest komenda służąca do przepisania pliku z postaci dziennika do postaci zrzutu bazy - pozwala to zaoszczędzić miejsce na dysku i znacznie przyspieszyć start systemu.

W przypadku \emph{Append Only File} domyślnie dane są zapisywane na dysk co sekundę przy pomocy wywołania systemowego \verb+fsync()+, które gwarantuje że dane zostaną zapisane na dysku.
Alternatywnie możliwe jest wywołanie \verb+fsync()+ po każdej modyfikującej operacji, ale obija się to tak bardzo negatywnie na wydajności, że nie jest to zalecana konfiguracja.
Można także skonfigurować Redis tak, aby nigdy sam nie dokonywał operacji \verb+fsync()+ i powierzył odpowiedzialność za regularne zapisywanie danych na dysk systemowi operacyjnemu, co zwiększa wydajność kosztem bezpieczeństwa danych.

\subsection*{Wersjonowanie}

Redis nie posiada mechanizmu wersjonowania rekordów - dla klucza jest przechowywana tylko jego aktualna wartość i nie ma identyfikatorów wersji.
System ten jednak obsługuje transakcje (operacja \verb+MULTI+) oraz komendę \verb+WATCH+, które pozwalają na uniknięcie przypadkowego nadpisania rekordu.

\subsubsection*{Zmiana rekordu bez użycia MULTI i WATCH}

\begin{enumerate}
 \item Klienci A oraz B odczytują wartość klucza (\verb+GET nazwa_klucza+)
 \item Klienci A oraz B przetwarzają otrzymaną wartość klucza i zmienia ją
 \item Klient A zapisuje nową wartość dla klucza (\verb+SET nazwa_klucza nowa_wartosc1+)
 \item Klient B zapisuje nową wartość dla klucza (\verb+SET nazwa_klucza nowa_wartosc2+) nadpisując tym samym zmiany dokonane przez klienta A.
\end{enumerate}

\subsubsection*{Zmiana rekordu z użyciem transakcji}

\begin{enumerate}
 \item Klienci A oraz B oznaczają klucz jako obserwowany (\verb+WATCH nazwa_klucza+)
 \item Klienci A oraz B odczytują wartość klucza (\verb+GET nazwa_klucza+)
 \item Klienci A oraz B przetwarzają otrzymaną wartość klucza i zmienia ją
 \item Klienci A oraz B rozpoczynają transakcję w celu zapisania nowej wartości (\verb+MULTI+)
 \item Klient A zapisuje nową wartość dla klucza (\verb+SET nazwa_klucza nowa_wartosc1+)
 \item Klient B zapisuje nową wartość dla klucza (\verb+SET nazwa_klucza nowa_wartosc2+)
 \item Klient A kończy transakcję (\verb+EXEC+) - transakcja kończy się powodzeniem
 \item Klient B kończy transakcję (\verb+EXEC+) - transakcja kończy się niepowodzeniem, ponieważ klient A zmodyfikował obserwowany klucz.
 Zmiana dokonana przez klienta B jest wycofana.
 \item Klient B musi wykonać całą operację od początku, odczytując tym samym wartość zapisaną przez klienta A.
\end{enumerate}


\subsection*{Wyszukiwanie}

Redis implementuje operacje na kilku różnych strukturach danych.
Wyróżnia go to spośród innych systemów typu klucz-wartość, sprawiając że jest on tym samym o wiele łatwiejszy w użyciu i wymaga mniej pracy od programisty.

Podobnie jak inne systemy typu klucz-wartość Redis nie udostępnia możliwości indeksowania zapisanych wartości, ale samodzielne tworzenie takich indeksów jest nieco ułatwione dzięki istnieniu takich struktur danych jak zbiór czy lista.

\subsubsection*{Obsługiwane struktury danych}

\begin{description}
 \item[Ciąg znaków] - jest to podstawowa struktura danych.
 Dostępne są między innymi operacje odczytujące i zapisujące wartość dla klucza, oraz (nietypowo) operacje pozwalające na dopisanie ciągu znaków na koniec klucza oraz na odczytanie jedynie jego fragmentu.

 \item[Lista] - pozwala na zapisanie sekwencji wartości.
 Możliwe jest atomowe dodanie wartości na początek bądź koniec listy oraz atomowe pobranie wartości z początku lub końca.
 Dostępna jest także blokująca wersja pobrania wartości.
 Listy są bardzo przydatne do implementacji kolejek zadań.
 Stosuje się je także do łączenia wielu rekordów w kolekcje (zapisując ich klucze na liście), ale warto pamiętać, że stronicowanie takiego zbioru ma złożoność liniową do wielkości listy, więc powinno się go raczej unikać.

 \item[Zbiór] - umożliwia wykonywania operacji na zbiorach (sprawdzenie członkostwa elementu, suma, przecięcie, różnica).
 Możliwe jest pobranie jednego, losowego elementu, albo wszystkich elementów zbioru.

 \item[Posortowany Zbiór] - umożliwia zapisanie posortowanej sekwencji wartości.
 Możliwe jest wykonanie przecięcia i sumy, ale nie różnicy posortowanych zbiorów.
 Możliwe jest także, podobnie jak dla list, stronicowanie zbioru.

 \item[Tablica mieszająca] - często rekordy w Redisie są zapisywane w taki sposób, że kluczowi identyfikującemu rekord przypisujemy wartość, którą jest obiekt, zserializowany, np. do formatu JSON.
 Jest to akceptowalne, jeżeli zawsze chcemy odczytywać rekordy w całości, ale jeżeli rekord jest duży, a potrzebne jest nam tylko kilka jego pól, to lepiej każde z nich zapisać pod osobnym kluczem.
 Tablica mieszająca jest strukturą, która optymalizuje właśnie to zastosowanie.
\end{description}

\subsection*{Unikalne cechy}

\subsubsection*{Publish-Subscribe}

Unikalne cechy

\subsubsection*{Pamięć Wirtualna}

Redis jest przeznaczony do pracy ze zbiorami danych, które w całości mieszczą się w pamięci operacyjnej serwera.
Ponieważ jednak w rzeczywistości często występują zbiory danych, których nie da się zmieścić w pamięci RAM, ale których niektóre fragmenty są odczytywane i zapisywane dużo częściej niż inne, Redis implementuje własną wersję pamięci wirtualnej.

Bardzo często pada pytanie dlaczego Redis implementuje coś, co jest już zaimplementowane w systemie operacyjnym.
Szeroko odpowiada na to artykuł na blogu autora tej bazy \cite{antirez-redis-vm}.
Jedną z zalet własnej implementacji jest to, że pozwala to na stronicowanie na poziomie rekordów, podczas gdy system operacyjny operuje na poziomie stron pamięci, które mogą zawierać zarówno często, jak i rzadko używane rekordy.
Inną zaletą jest to, że rekord zapisany na dysku nie musi mieć identycznej postaci jak w pamięci.
Poprzez nie zapisywanie metadanych i różnych wskaźników, zapisywane rekordy mogą zajmować na dysku 10 razy mniej miejsca niż w pamięci operacyjnej.
Zaletą nie wymienioną przez autora jest to, że kiedy korzystamy z wirtualnej pamięci, ustalony zostaje maksymalny rozmiar pamięci zajmowanej przez system\footnote{Opcja ta może być włączona także bez włączania pamięci wirtualnej, ale włączenie jej w takiej sytuacji może prowadzić do zrywania połączeń z klientami a nawet utraty danych.}.
Dzięki temu Redis może koegzystować na jednej maszynie z innymi aplikacjami takimi jak serwer aplikacyjny i nie konkurować z nimi o RAM.

\subsection*{Typowe zastosowania}

Typowe zastosowania

\subsection*{Przeciwwskazania}

Przeciwwskazania

\subsection*{Dokumentacja i wsparcie}

Dokumentacja i wsparcie

\subsection*{Pomocne odnośniki}
 
Pomocne odnośniki

\section{Riak}
\label{sec:riak}

\subsection*{Wstęp} 

Wstęp

\subsection*{Protokół komunikacji}

Protokół komunikacji

\subsection*{Replikacja}

Replikacja

\subsection*{Partycjonowanie}

Partycjonowanie

\subsection*{Persystencja}

Persystencja

\subsection*{Wersjonowanie}

Wersjonowanie

\subsection*{Wyszukiwanie}

Wyszukiwanie

\subsection*{Unikalne cechy}

Unikalne cechy

\subsection*{Typowe zastosowania}

Typowe zastosowania

\subsection*{Przeciwwskazania}

Przeciwwskazania

\subsection*{Dokumentacja i wsparcie}

Dokumentacja i wsparcie

\subsection*{Pomocne odnośniki}
 
Pomocne odnośniki
\chapter{Bazy Kolumnowe}

\section*{Streszczenie}
W tym rozdziale opisane zostały dwie bazy kolumnowe: HBase i Apache Cassandra.
Mimo podobnego modelu danych, bazy te znacznie różnią się pod wieloma względami.
HBase powstało jako implementacja open source systemu opisanego w artykule o Google BigTable i mimo ciągłego rozwoju i dodawania coraz nowszych funkcjonalności te dwa systemy są do siebie wciąż bardzo zbliżone.
Z drugiej strony Apache Cassandra, który to system został zaprojektowany między innymi przez jednego z autorów Amazon Dynamo, mimo że czerpie swój model danych z Google BigTable, na poziomie architektury znacznie bardziej przypomina system firmy Amazon niż Google, a co za tym idzie podobnie jak Riak należy do grupy systemów oferujących bardzo duże gwarancje dostępności.

\section{Apache Cassandra}
\label{sec:cassandra}

\subsection*{Wstęp}

Apache Cassandra powstała na potrzeby jednej z największych stron internetowych na świecie: sieci społecznościowej Facebook. 
Zadaniem tego systemu było zastąpienie wcześniejszej architektury opartej o MySQL do wyszukiwania w skrzynkach wiadomości użytkowników.
Jednym z architektów nowego systemu był Avinash Lakshman, który wcześniej był współautorem Amazon Dynamo.
W 2008 roku Facebook zdecydował się na udostępnienie swojego produktu na licencji open source.
Przez pierwszy rok jedynie programiści firmy Facebook mieli możliwość zmiany kodu aplikacji, co w połączeniu z tym, że odmawiali oni przyjmowania zmian autorstwa innych programistów, powodowało duże napięcia w społeczności powiązanej z tym projektem i podniesienie się licznych głosów nawołujących do podziału projektu (ang. \emph{fork}).
W marcu 2009 Facebook przekazał prawa do Cassandra fundacji Apache, która okazała się dużo bardziej sprawna w zarządzaniu tym projektem\cite{evans-cassandra}.
W tym samym roku ukazała się także publikacja\footnote{Czytając ten artykuł warto jednak pamiętać, że odnosi się on do systemu w postaci w jakiej został on stworzony na potrzeby firmy Facebook, a nie do systemu, który ostatecznie został upubliczniony i nosi teraz nazwę Apache Cassandra. Główną różnicą między nimi jest to, że integracja z systemem Zookeeper nie została upubliczniona i nie jest częścią Apache Cassandra, a co za tym idzie różnią się one tym, że dostępna publicznie wersja systemu nie ma wyróżnionych węzłów i nie zależy od innych systemów.} omawiająca architekturę projektu autorstwa jego twórców \cite{cassandra-paper}.

Apache Cassandra stanowi bardzo ciekawe połączenie modelu danych zaczerpniętego z Google BigTable z modelem replikacji i partycjonowania (a co za tym idzie konsystencji) zaczerpniętym z Amazon Dynamo.
Podobnie jak opisany wcześniej Riak, Cassandra nie posiada wyróżnionych węzłów, ale w odróżnieniu od tej bazy typu klucz-wartość posiada ona bardziej zaawansowany model danych.

\subsection*{Protokół komunikacji}

Apache Cassandra posiada API wykorzystujące technologię RPC Apache Thrift\footnote{Apache Thrift został stworzony na potrzeby firmy Facebook. Jest on podobny do technologii CORBA.} do komunikacji, dzięki czemu oferuje wsparcie dla 12 języków programowania.
Dla najpopularniejszych języków programowania takich jak Java, Python czy Ruby istnieją także biblioteki wyższego poziomu, często wzorowane na bibliotekach ORM.

Cassandra pozwala na rozszerzanie podstawowego zestawu funkcjonalności systemu (na przykład zestawu komparatorów używanego do sortowania) poprzez implementację własnych klas w języku Java (w tym języku została napisana ta baza).
Dzięki temu, że istnieje wiele języków programowania, które są kompilowane do kodu pośredniego maszyny wirtualnej Java (JVM), rozszerzenia te można pisać także w wielu innych językach.

\subsection*{Replikacja}

Apache Cassandra czerpie swój model replikacji z Amazon Dynamo.
Podobnie jak w przypadku tamtej bazy użytkownik specyfikuje parametry R, W i N, określające odpowiednio ile węzłów musi odpowiedzieć na zapytanie aby odczyt się powiódł, ile musi potwierdzić zapis, oraz na ile węzłów jest replikowany dany klucz.

Bardzo ciekawą właściwością tej bazy jest to, że strategia wyboru węzłów, na które zostanie replikowany dany przedział kluczy może być konfigurowana.
Poza domyślną strategią, która podobnie jak w Amazon Dynamo replikuje klucze na N-1 kolejnych węzłów, dostępne są także dwie dodatkowe strategie: ,,Rack Aware Strategy'' oraz ,,Data Center Shard Strategy''.
Pierwsza z nich sprawia, że dla każdego przedziału drugi węzeł na liście preferencyjnej\footnote{patrz strona \pageref{sec:dynamo-replikacja}.} będzie węzłem z innego centrum obliczeniowego, a kolejne będą z innej szafy (ang. \emph{rack}) niż koordynator.
Druga z dodatkowych strategii pozwala na zdefiniowanie jak repliki mają zostać podzielone między centrami obliczeniowymi w przypadku gdy jest ich więcej niż dwa.
Dla przykładu w przypadku gdy dysponujemy trzema centrami obliczeniowymi, a każdy klucz jest replikowany na sześć węzłów, możemy zdefiniować że trzy z nich będą węzłami z pierwszego centrum, dwa z drugiego i jeden z trzeciego.

Podobnie jak Riak, Cassandra pozwala na specyfikowanie parametrów R i W dla zapytań nie tylko jako liczby, ale także jako wartości symboliczne, np. $ALL$ (wszystkie), czy $QUORUM$ (większość).
Dodatkowo dostępna jest wartość $DCQUORUM$, która działa w połączeniu ze strategiami opisanymi powyżej.
Pozwala ona na wyspecyfikowanie, że jedynie węzły z lokalnego centrum obliczeniowego mają być brane pod uwagę przy określaniu liczby węzłów potrzebnej do osiągnięcia kworum, dzięki czemu można zapewnić konsystencję na poziomie centrum danych oraz uniknąć oczekiwania na odpowiedź odległych węzłów.

\subsection*{Partycjonowanie}

Cassandra wykorzystuje algorytm \emph{Consistent Hashing}\footnote{patrz strona \pageref{sec:dynamo-consistent-hashing}.} w wersji podstawowej.
Każdemu węzłowi jest przyporządkowany jeden przedział kluczy, których jest koordynatorem, a ponadto każdy klucz jest replikowany na N-1 kolejnych węzłów.

Algorytm podziału kluczy na przedziały jest konfigurowalny.
Domyślnym (i polecanym) algorytmem jest podział ,,losowy'', czyli tak samo jak w Amazon Dynamo na podstawie funkcji mieszającej MD5, dzięki czemu klucze powinny być rozłożone równomiernie pomiędzy węzłami.
Pozostałe dwa dostępne algorytmy nie używają funkcji mieszającej, tylko porównują wartość klucza bezpośrednio z wartościami granicznymi przypisanymi poszczególnym węzłom, dzięki czemu węzły o kolejnych kluczach trafią do tego samego węzła (podobnie jak to ma miejsce w Google BigTable).
Umożliwia to wyszukiwanie zakresów kluczy na podstawie ich wartości.
Niestety to podejście może powodować bardzo nierównomierny rozkład obciążenia w klastrze, a co za tym idzie doprowadzić do niestabilnego działania systemu.
Na szczęście wykorzystując dodatkową rodzinę kolumn (patrz niżej) można własnoręcznie stworzyć indeks pozwalający na dokonywanie tego typu zapytań bez narażania się na ryzyka związanie ze zmianą algorytmu partycjonowania.

\subsection*{Persystencja}

Cassandra, podobnie jak Google BigTable\footnote{patrz strona \pageref{sec:bigtable-architektura-serwera-tabletow}.} zapisuje dane najpierw do logu transakcji, który potem może służyć do ewentualnego odtworzenia stanu bazy w razie awarii, a następnie do struktury danych w pamięci (\emph{memtable}).
Z pamięci co pewien czas dane są zrzucane na dysk tworząc pliki SSTable (Cassandra wykorzystuje tą samą nomenklaturę co Google BigTable).
Ponieważ odczyty często muszą przeglądnąć więcej niż jeden plik SSTable aby znaleźć najnowszą wersję rekordu, konieczne jest łączenie tych plików co pewien czas w większe.

Użytkownik bazy ma do dyspozycji dwie opcje konfiguracji wpływające na trwałość danych.
Możliwe jest albo wykonywanie operacji \verb+fsync()+ przed potwierdzeniem każdej operacji zapisu, albo co pewien określony czas.
W pierwszej z tych konfiguracji zaleca się aby dziennik transakcji znajdował się na innym dysku niż reszta plików, aby uniknąć opóźnienia spowodowanego przesunięciem głowicy przy każdym zapisie.

\subsection*{Wersjonowanie}

Cassandra, w odróżnieniu od Amazon Dynamo i Riak nie dysponuje mechanizmem zegarów wektorowych.
W Cassandrze każda wartość (a właściwie para klucz-nazwa kolumny) ma przypisaną 64-bitową liczbę całkowitą będącą znacznikiem czasowym ostatniej modyfikacji.
W przypadku gdy system wykryje istnienie więcej niż jednej wartości dla danej kolumny, automatycznie jest wybierana najnowsza wartość.

Brak wsparcia dla mechanizmu zegarów wektorowych wydaje się być problematyczny w systemie, który podobnie jak Amazon Dynamo czy Riak należy do rodziny systemów AP w rozumieniu teorii CAP.
Warto jednak zwrócić uwagę, że Dynamo i Riak są systemami typu klucz-wartość i tam wersjonowane są całe rekordy, a zatem jeżeli rekord zostanie zmodyfikowany przez dwóch klientów równocześnie, to nawet jeżeli ci klienci zmodyfikowali całkiem różne ,,pola''\footnote{W systemach typu klucz wartość najczęściej przypisujemy kluczowi wartość będącą zserializowanym obiektem, np. w formacie JSON. Przez pole rekordu rozumiemy tu właśnie pole tego zserializowanego obiektu.} rekordu, konflikt musi zostać rozwiązany przez aplikację.
W przypadku Cassandry każde takie ,,pole'' ma swój znacznik wersji, a co za tym idzie wiele przypadków konfliktów napotykanych w systemach typu klucz-wartość nie ma miejsca w bazie kolumnowej.
W większości pozostałych przypadków można uniknąć powstawania konfliktów odpowiednio modyfikując strukturę danych.

\subsection*{Wyszukiwanie}

\subsubsection*{Model Danych}

Model danych w Apache Cassandra bardzo przypomina ten, który został wcześniej opisany przy okazji Google BigTable\footnote{patrz strona \pageref{google-bigtable-model-danych}.}, ale z pewnymi istotnymi różnicami.
Niestety jedną z tych różnic jest wykorzystanie tej samej nomenklatury, ale w różnych znaczeniach, co może być mylące dla osób które znają model danych bazy Google, dlatego poniżej przedstawiona zostanie lista terminów dotyczących Apache Cassandra wraz z odniesieniami do odpowiadających im terminów dotyczących BigTable.
Bardzo dobre wprowadzenie do modelu danych Apache Cassandra można znaleźć w artykule na blogu Arina Sarkissiana \cite{arin-wtf-is-a-supercolumn}.

\begin{description}
 \item[Keyspace] - dosłownie ,,przestrzeń kluczy''.
 Pozwala na grupowanie rodzin kolumn tworząc pewien rodzaj ,,przestrzeni nazw'' i pozwalając na ustawienie różnych zmiennych konfiguracyjnych.
 Pod wieloma względami przypomina pojęcie bazy, znane z relacyjnych systemów zarządzania bazami danych, takich jak MySQL.
 Nie posiada odpowiednika w BigTable.

 \item[Column Family] - rodzina kolumn.
 Jest to zbiór rzędów, z których każdy posiada klucz, który go identyfikuje oraz dowolną liczbę kolumn.
 Nie istnieje żaden schemat, który wymuszałby aby rzędy tej samej rodziny kolumn miały tą samą strukturę - każdy może mieć zupełnie inny zestaw kolumn.
 Co więcej, kolumny należące do danej rodziny mogą być wyszukiwane po nazwie, stronicowane a ponadto są posortowane przy użyciu komparatora wybranego przez użytkownika.
 Te właściwości sprawiają, że bardzo często dane są przechowywane w nazwach kolumn, a wartości są ignorowane.
 W BigTable odpowiednikiem rodziny kolumn jest tabela.

 \item[Column] - kolumna.
 Jest to trójka (nazwa, wartość, czas ostatniej modyfikacji).
 Nazwa i wartość kolumny nie może przekroczyć rozmiaru dwóch gigabajtów.
 W odróżnieniu od Google BigTable (gdzie jest przechowywanych wiele wersji wartości), Cassandra przechowuje tylko najnowszą wersję, a znacznik czasowy jest używany tylko przy rozstrzyganiu konfliktów.

 \item[Super Column] - super kolumna.
 Jest to struktura pozwalająca na grupowanie wielu kolumn.
 W odróżnieniu od rodziny kolumn, super kolumny nie posiadają kluczy.
 Odpowiednikiem super kolumn w BigTable są rodziny kolumn.

 \item[Super Column Family] - rodzina super kolumn.
 Zwykła rodzina kolumn nie pozwala na to aby jej rzędy składały się z kolumn i super kolumn, dopuszczalne są tylko kolumny.
 Rodzina super kolumn z kolei pozwala jedynie aby jej rzędy składały się z super kolumn.
 Stanowi to różnicę w stosunku do BigTable, gdzie rząd tabeli mógł składać się z kolumn zarówno pogrupowanych w rodziny jak i nie.
\end{description}

Do wersji 0.7 Apache Cassandra nie była możliwa zmiana listy rodzin kolumn i przestrzeni kluczy bez ponownego uruchomienia serwera, ale najnowsze wydanie tej bazy wprowadza taką możliwość.

\subsubsection*{Wyszukiwanie rekordów}

Apache Cassandra posiada rozbudowane API pozwalające na filtrowanie i stronicowanie kolumn w ramach pojedynczego rekordu.
Przy wykorzystaniu odpowiedniego mechanizmu partycjonowania, możliwe jest także wyszukiwanie rekordów zakresami po kluczu głównym.

Od wersji 0.7 możliwe jest już tworzenie indeksów drugiego poziomu, co pozwala na wyszukiwanie rekordów po wartości kolumny.
Wcześniej, aby uzyskać taką samą funkcjonalność należało samodzielnie zaimplementować taki indeks poprzez dodanie kolejnej rodziny kolumn, gdzie kluczem byłaby wartość, po której chcemy wyszukać, a nazwami kolumn byłyby identyfikatory wyszukiwanych rekordów.

\subsubsection*{MapReduce}

Cassandra nie posiada własnego, wbudowanego frameworku MapReduce.
Posiada natomiast integrację z systemem Hadoop, dla którego może służyć zarówno jako zbiór danych wejściowych, jak i danych wyjściowych (od wersji 0.7).

\subsection*{Unikalne cechy}

Cassandra łączy w sobie model danych, który jest znacznie bogatszy i łatwiejszy w użyciu niż spotykane w systemach typu klucz-wartość, z technikami znanymi z Amazon Dynamo, które pozwalają na osiągnięcie bardzo wysokiej dostępności systemu.
Dzięki temu, że użytkownikowi jest pozostawiony wybór między konsystencją a dostępnością w obliczu awarii czy podziałów sieci, możliwe jest także stosunkowo proste dopasowanie zachowania systemu do potrzeb aplikacji z niego korzystającej.

Wart uwagi jest fakt, że Apache Cassandra jako jeden z bardzo nielicznych systemów pozwala na taką konfigurację, aby mógł bezproblemowo pracować będąc rozproszonym między wieloma centrami obliczeniowymi, co stawia go w czołówce systemów oferujących najwyższe gwarancje dostępności.

\subsection*{Typowe zastosowania}

Jeszcze zanim ten system został upubliczniony, dowiódł on swojej przydatności w firmie Facebook, gdzie umożliwił wyszukiwanie w miliardach wiadomości, które użytkownicy tego serwisu wysyłają do siebie codziennie.
Nie jest zatem przypadkiem, że baza ta została później zaadoptowana przez innych wielkich graczy na rynku aplikacji internetowych: Twitter, Digg i Reddit.
Zainteresowanie jakim Cassandra cieszy się pośród inżynierów największych aplikacji internetowych, świadczy o tym, że system ten należy do grupy najlepiej skalowalnych baz danych dostępnych aktualnie na rynku.

Do tej pory jednak Cassandra nie zdobyła dużej popularności wśród twórców mniejszych aplikacji.
Jest to najprawdopodobniej związane z tym, że tworzenie aplikacji opartych o ten system jest dość trudne i wymaga znacznie większych nakładów pracy niż w przypadku dokumentowych baz danych, czy nawet baz klucz-wartość takich jak Riak, które mogą się pochwalić wbudowanym frameworkiem MapReduce. 
Pomiędzy wersjami 0.6 i 0.7\footnote{W momencie pisania pracy data wydania wersji 0.7 nie jest jeszcze znana, ale dostępna jest już wersja \emph{Release Candidate} 4.} zostało jednak wprowadzonych wiele zmian (na czele z indeksami drugiego poziomu i możliwością zmiany schematu bez ponownego uruchamiania serwera), które sprawiają, że system ten jest znacznie bardziej przyjazny dla użytkownika niż jeszcze przed rokiem, a zatem jest bardzo prawdopodobne, że nabierze on tym samym większej popularności.

Podobnie jak Riak, Cassandra jest systemem, który powinien dobrze radzić sobie ,,w chmurze''.
Posiada on nawet wbudowane mechanizmy pozwalające na określenie topologii sieci w środowisku Amazon EC2 na potrzeby strategii replikacji.

\subsection*{Przeciwwskazania}

Apache Cassandra jest systemem, w którym odczyty trwają zazwyczaj dłużej niż zapisy, dlatego ten system lepiej się sprawdza w aplikacjach, które wymagają dużej liczby zapisów i modyfikacji.
Znanym ograniczeniem tego systemu jest to, że nie można w nim przechowywać dużych plików, ze względu na to, że protokół Thrift nie udostępnia opcji strumieniowania danych.
Brak transakcji i zegarów wektorowych sprawia także, że problematyczne może być stworzenie w oparciu o ten system aplikacji, w której różni klienci często dokonywaliby zmian tych samych rekordów. 

\subsection*{Dokumentacja i wsparcie}

W internecie jest dostępnych wiele źródeł informacji na temat Apache Cassandra.
Wiele użytecznych informacji znajduje się na wiki projektu, jednak struktura tej strony jest nieprzejrzysta i utrudnia dotarcie do interesujących danych.

Projekt ten jest aktywnie rozwijany i wyraźnie nabiera co raz większej popularności.

Od dość niedawna możliwe jest też uzyskanie płatnego wsparcia oraz szkoleń dzięki firmie Riptano.

\subsection*{Pomocne odnośniki}

Poniżej zamieszczono kilka odnośników do stron WWW związanych z Apache Cassandra:

\begin{description}
 \item [http://cassandra.apache.org/] - strona domowa projektu
 \item [http://wiki.apache.org/cassandra/] - strona wiki z dokumentacją
 \item [http://www.riptano.com/docs/0.6/index] - alternatywna dokumentacja dla wersji 0.6 oferowana przez firmę Riptano
 \item [http://arin.me/blog/wtf-is-a-supercolumn-cassandra-data-model] - bardzo dobry artykuł opisujący model danych systemu
 \item [http://www.parleys.com/\#st=5\&id=1866] - prezentacja wideo oferująca bardzo dobre wprowadzenie do projektu
\end{description}

\section{HBase}
\label{sec:hbase}

\subsection*{Wstęp}

Apache HBase to system wzorowany na Google BigTable.
Podobnie jak tamten system HBase zależy od dwóch innych systemów: Apache Hadoop i Apache ZooKeeper, pierwszy z których zapewnia replikację (podobnie jak GFS), a drugi przechowuje konfigurację i pozwala na wybór węzła master (tak jak Google Chubby).
HBase powstało początkowo jako rozszerzenie do projektu Hadoop, ale z czasem zostało wydzielone jako osobny projekt, który początkowo przyjął tą samą numerację wersji i harmonogram wydań co projekt-rodzic, ale od najnowszej, jeszcze nie wydanej wersji 0.90, także ta więź zostanie zerwana.

Apache Hadoop, na którym opiera się HBase, to implementacja Google MapReduce w Javie, jednak ponieważ tamten framework zależy od GFS, który nie jest publicznie dostępny, to Hadoop posiada własną implementację rozproszonego systemu plików.

\subsection*{Protokół komunikacji}

Istnieją trzy różne protokoły komunikacji z HBase: Java, Thrift oraz REST (dzięki nakładce o nazwie Stargate).
Najczęściej aktualizowanym i w związku z tym oferującym najwięcej możliwości i najszybciej reagującym na zmiany jest API w Javie, ale pozostałe dwa nie są daleko w tyle.
HBase jest wykorzystywane głównie w połączeniu z aplikacjami pisanymi w języku Java (prawdopodobnie głównie ze względu na ścisłe powiązanie z Apache Hadoop), ale dostępne są też biblioteki dla innych języków programowania.

Podobnie jak większość systemów NoSQL, HBase nie posiada żadnych bardziej zaawansowanych mechanizmów kontroli dostępu.
Jest jednak prawdopodobne, że w przyszłych wersjach taka funkcjonalność zostanie dodana dzięki koprocesorom, czyli funkcjonalności, która ma pojawić się w HBase 0.92 (patrz dalej).
Istnieje już nawet implementacja on nazwie \emph{Secure HBase}, która to umożliwia\footnote{http://hbaseblog.com/2010/10/11/secure-hbase-access-controls/}.

\subsection*{Replikacja}

Podobnie jak Google BigTable, HBase pozostawia replikację rozproszonemu systemowi plików, na którym zapisuje dane.
Teoretycznie, HBase może być wykorzystane z praktycznie dowolnym rozproszonym systemem plików, ale w rzeczywistości system ten jest wykorzystywany tylko w połączeniu z HDFS (\emph{Hadoop Distributed File System}), oraz o wiele rzadziej z KFS (dawniej \emph{Kosmos File System}, teraz \emph{CloudStore}).

\subsubsection*{HDFS}

HDFS jest systemem plików przypominającym swoją architekturą Google File System.
Została ona opisana bliżej w artykule z 2007 roku \cite{hdfs-architecture}, który wprawdzie daje dobry pogląd o tym jak działa ten system, ale jest niestety na tyle stary, że nie jest już autorytatywnym źródłem na ten temat.

Podobnie jak GFS występują w tym systemie dwa typy węzłów: NameNode (jeden w klastrze, odpowiednik węzła Master) oraz DataNode (wszystkie pozostałe, odpowiedniki \emph{chunk-server}).
Tak samo jak GFS, HDFS jest stworzony z myślą przede wszystkim o dużych i bardzo dużych plikach, które podobnie jak w systemie Google są przechowywane podzielone na fragmenty o domyślnej wielkości 64MB, które to fragmenty są także jednostką replikacji.

Największą różnicą między HDFS a GFS jest to, że w systemie Apache awaria węzła NameNode powoduje awarię całego systemu i konieczność jego ponownego uruchomienia.
Jest to o tyle istotne, że węzeł NameNode HDFS pozostaje jedyną ,,piętą Achillesową''\footnote{czyt. \emph{Single Point of Failure}} HBase, które wprawdzie także posiada węzeł Master, ale od niedawna jego awaria już nie powoduje awarii całego systemu.

Inną istotną różnicą między tymi systemami jest to, że ponieważ HDFS powstał na potrzeby narzędzia MapReduce (Hadoop), które z natury nie potrzebuje modyfikować plików po tym jak zostaną one już zapisane, to system ten nie posiada żadnej funkcjonalności przypominającej tą znaną z GFS, która pozwala na dopisywanie ,,rekordów'' na koniec plików przez wielu klientów równocześnie.
Początkowo HDFS był całkowicie pozbawiony możliwości zmieniania zawartości plików, a same pliki były dostępne do odczytu dopiero gdy zostały poprawnie zamknięte\footnote{W związku z tym, HBase jeszcze do niedawna traciło dane w przypadku awarii węzła zapisującego, ponieważ nie dało się ich odczytać z nie zamkniętego poprawnie pliku. W najnowszej wersji (niestabilnej) błąd ten już został naprawiony i da się odzyskiwać dane zapisane w ten sposób.}, później jednak została dodana operacja \verb+append()+, która pozwala na dopisywanie na koniec pliku, ale ponieważ okazało się że nie działa ona całkowicie poprawnie, operacja ta jest dostępna dopiero po ustawieniu odpowiedniej opcji w konfiguracji.

\subsection*{Partycjonowanie}

Apache HBase wykorzystuje ten sam sposób partycjonowania danych co Google BigTable.
Podobnie jak w tamtym systemie, tabele są podzielone na fragmenty o konfigurowalnej wielkości (domyślnie 256MB) nazywane regionami (w BigTable tabletami).
Tak samo jak w BigTable, rzędy są sortowane, ale w odróżnieniu od Apache Cassandra nie jest możliwa zmiana algorytmu sortowania.

Węzły systemu HBase dzielą się na węzeł master oraz serwery regionów (ang. \emph{Region Server}), które są odpowiednikami serwerów tabletów z BigTable.
Każdy serwer regionów (tak jak w Apache Cassandra i BigTable) zapisuje wszystkie zmiany w dzienniku (\verb+HLog+), a następnie odwzorowuje je w pamięci (\verb+MemStore+).
Struktura danych w pamięci jest regularnie zapisywana na dysk do plików (\verb+HFile+), które formatem przypominają pliki SSTable znane z BigTable, a te z kolei co pewien czas są łączone aby uniknąć wyszukiwania w dużej liczbie plików przy odczycie.
Architektura ta jest dokładniej opisana w artykule na blogu Larsa George - jednego z programistów tworzących HBase \cite{george-hbase-storage}, z którego zaczerpnięty został Rysunek \ref{fig:hbase-files}. 

\myfigure{chapters/bazy_kolumnowe/hbase-files.png}{Architektura HBase}{fig:hbase-files}{Źródło: \cite{george-hbase-storage}}

W starszych wersjach HBase, w przypadku awarii węzła master cały system musiał być uruchomiony ponownie, ale od wersji 0.20 została wprowadzona integracja z odpowiednikiem Google Chubby - Apache ZooKeeper, co pozwoliło na usunięcie tego problemu.

\subsection*{Persystencja}

Za trwałość danych w HBase odpowiada wybrany przez użytkownika system plików.
Jednym z głównych problemów HBase, który został rozwiązany dopiero w nowej, niestabilnej wersji jest wspomniany już problem utraty danych w przypadku awarii serwera regionów, która sprawi, że plik ten nie zostanie poprawnie zamknięty.
Pomijając ten problem, ponieważ każda operacja zmiany wymaga aby dane zostały poprawnie zreplikowane na skonfigurowaną przez użytkownika\footnote{stopień replikacji} liczbę węzłów, to dane zapisane w HBase są zapisywane w sposób trwały w porównaniu z innymi systemami.

\subsection*{Wersjonowanie}

Podobnie jak w Google BigTable, dane są wersjonowane przy pomocy znaczników czasowych (albo innych wartości liczbowych podanych przez użytkownika) i przechowywane są wszystkie wersje wartości, a nie tylko najnowsza jak w Apache Cassandra.

Ponieważ Apache HBase jest systemem CP w rozumieniu Teorii CAP, to przechowywanie wielu wersji każdej wartości ma mniejsze znaczenie niż w systemach typu AP, gdzie konflikty są bardziej prawdopodobne.
Znaczenie to jest tym mniejsze, że HBase posiada możliwość zakładania blokad na rzędy, dzięki czemu możliwe jest zaimplementowanie transakcji w obrębie pojedynczego rekordu (opcjonalnie można też włączyć możliwość blokowania wielu rzędów równocześnie, ale może to się wiązać z negatywnymi skutkami dla wydajności).

\subsection*{Wyszukiwanie}

\subsubsection*{Model Danych}

Model danych HBase jest praktycznie identyczny do modelu danych BigTable\footnote{patrz strona \pageref{google-bigtable-model-danych}.}, oraz w odróżnieniu od Apache Cassandra posługuje się tą samą nomenklaturą nie zmieniając znaczenia poszczególnych terminów.

W HBase jednostką najwyższego poziomu jest \emph{tabela}, która jest posortowanym zbiorem rzędów, z których każdy jest identyfikowany przez klucz, który jest ciągiem bajtów.
Sortowanie odbywa się po kluczu, w porządku leksykalnym, bajtowo.
Każdy rząd może składać się z dowolnej liczby kolumn oraz rodzin kolumn, które z kolei także mogą zawierać kolumny.
Oczywiście tak jak we wszystkich bazach kolumnowych, ,,puste'' wartości nie zajmują miejsca na dysku, a dowolne dwa rzędy mogą się całkowicie różnić zestawem kolumn.

W odróżnieniu od Apache Cassandra nie ma potrzeby deklarowania z góry tabel, mogą one być tworzone dynamicznie, w trakcie działania aplikacji i bez potrzeby restartowania czegokolwiek.

\subsubsection*{Wyszukiwanie rekordów}

Jak przystało na kolumnową bazę, HBase posiada bardzo ograniczone API.
Możliwe jest pobieranie rzędów używając klucza albo zakresu kluczy, a ponadto na poziomie pojedynczego wiersza limitowanie pobranych kolumn (także rodzin kolumn) oraz liczby i zakresu wersji wartości.

Funkcjonalność oferowana przez HBase w zakresie wyszukiwania nie należy do rozbudowanych, stąd też większość ciężaru spoczywa w tym przypadku na użytkowniku, który musi własnoręcznie implementować indeksy umożliwiające znajdowanie rekordów.

\subsection*{Unikalne cechy}

\subsubsection*{Integracja z Apache Hadoop}

Apache Hadoop jest projektem stworzonym z myślą o przetwarzaniu gigantycznych zbiorów danych, a Apache HBase z myślą o przechowywaniu takich zbiorów.
Funkcjonalność oferowana przez ten framework MapReduce stanowi bardzo istotne rozszerzenie dla HBase, ponieważ umożliwia dokonywanie różnego rodzaju obliczeń i transformacji danych, których dane wejściowe i wyjściowe mogą być w niej zapisywane, a co za tym idzie ułatwia tworzenie indeksów, zmaterializowanych widoków, a nawet import i eksport danych.

\subsubsection*{Coprocessor API}

Jedną z ciekawszych funkcjonalności, które będą dostępne w przyszłym wydaniu Apache HBase są koprocesory (ang. \emph{coprocessor}).
Mogą one być porównane do \emph{triggerów} w relacyjnych bazach danych.

Koprocesorem w HBase jest klasa (napisana w Javie), której metody są wywoływane w przestrzeni adresowej serwera w przypadku zajścia różnych zdarzeń.
Do zdarzeń tych należą między innymi zdarzenia związane z tworzeniem i łączeniem plików HFile, oraz zdarzenia, które mają miejsce przy zapisie, odczycie itp. pojedynczych wierszy.
W przyszłości planowane jest także zastosowanie koprocesorów w celu optymalizacji przetwarzania danych na serwerze, w celu zmniejszenia narzutu komunikacyjnego, jaki ma miejsce przy wykorzystaniu Hadoop do wykonywania operacji Map-Reduce.

\subsection*{Typowe zastosowania}

HBase, jako silnie konsystentna baza danych oferuje prostszy model programowania niż na przykład Apache Cassandra, ale z drugiej strony ma także mniejsze możliwości jeżeli chodzi o wyszukiwanie rekordów i filtrowanie kolumn, co sprawia, że nie jest wcale o wiele łatwiejszym systemem w użyciu.
Największą zaletą tej bazy jest bardzo dobra integracja z Apache Hadoop, która czyni ją kuszącą opcją dla aplikacji, które przechowują ogromne ilości danych przetwarzanych na różne sposoby, czyli na przykład wyszukiwarek internetowych.

Do najbardziej znanych aplikacji korzystających z HBase należy StumbleUpon - spersonalizowany silnik rekomendacji różnego rodzaju treści online.
HBase i Hadoop są także wykorzystywane przez wyszukiwarkę internetową firmy Microsoft - Bing\footnote{http://www.theregister.co.uk/2009/05/07/microsoft\_search\_built\_on\_open\_source/}.

\subsection*{Przeciwwskazania}

Apache HBase w swojej obecnej ,,stabilnej'' wersji (0.20.6) nie jest jeszcze bazą wystarczająco stabilną, aby można ją było polecić do zastosowania w środowisku produkcyjnym.
Szczególnie problematyczne jest tu ryzyko utraty danych w przypadku awarii węzła.

Nie wskazane (choć oczywiście możliwe) jest też uruchamianie HBase na platformie Amazon EC2 ze względu na istnienie pojedynczego punktu awarii, a mianowicie węzła NameNode HDFS.

W internecie można znaleźć informacje, że HBase jest zbyt wolne aby nadawać się dla aplikacji on-line\footnote{http://www.metabrew.com/article/anti-rdbms-a-list-of-distributed-key-value-stores/}, jednak przykład StumbleUpon pokazuje, że nie jest to prawdą.

\subsection*{Dokumentacja i wsparcie}

HBase, w odróżnieniu od większości systemów NoSQL nie jest tworzone głównie przez programistów jednej firmy, ale przez osoby należące do wielu różnych instytucji.
Zaletą takiego układu jest to, że łatwiej jest uzyskać darmową pomoc od ludzi, którzy pracują nad tą aplikacją ponieważ ich firmy nie zarabiają na sprzedawaniu tych usług.
Wadą jest jednak to, że brakuje firmy, która oferowałaby wyspecjalizowane usługi, szkolenia i gwarantowane wsparcie dla tego produktu.

Od września 2009 HBase zostało włączone do pakietu produktów opartych o Apache Hadoop oferowanego przez firmę Cloudera, co przynajmniej częściowo mityguje tą wadę.

Dokumentacja projektu, podobnie jak w przypadku Apache Cassandra jest dość chaotyczna i dostępna jedynie w postaci wiki.
W internecie jest dostępnych stosunkowo niewiele wideo-prezentacji opisujących tą bazę, ale za to można znaleźć dość dużo informacji o tym systemie na różnych blogach.

\subsection*{Pomocne odnośniki}

Poniżej zamieszczono kilka odnośników do stron WWW związanych z Apache HBase:

\begin{description}
 \item [http://hbase.apache.org/] - strona domowa projektu
 \item [http://wiki.apache.org/hadoop/Hbase] - strona wiki z dokumentacją
 \item [http://www.parleys.com/parleysserver/indexing/presentation.form?id=1859] - prezentacja wideo oferująca bardzo dobre wprowadzenie do projektu
 \item [http://nosqltapes.com/video/ryan-rawson-on-hbase-at-stumbleupon] - wywiad z jednym z twórców bazy, który opowiada między innymi o zastosowaniach HBase w firmie StumbleUpon
\end{description}

\chapter{Bazy Dokumentowe}

\section*{Streszczenie}
W tym rozdziale opisane zostaną dwie bazy dokumentowe: CouchDB i MongoDB.

\section{CouchDB}
\label{sec:couchdb}

\subsection*{Wstęp}

Apache CouchDB to dokumentowa baza danych oferująca szereg nie spotykanych w innych systemach funkcjonalności, takich jak framework MapReduce, który pozwala na tworzenie indeksów drugiego poziomu, czy bardzo rozbudowaną funkcjonalność replikacji, która pozwala na tworzenie systemów, w których klient może przechowywać lokalnie jedynie ten fragment bazy, który go interesuje (umożliwiając tym samym pracę nawet w warunkach utraty połączenia z siecią) i pozostawić funkcjonalność synchronizacji danych CouchDB.

System ten jest jednym z najdłużej dostępnych na rynku - baza ta została udostępniona już w 2005 roku.
O dojrzałości i popularności tego projektu świadczy ogromna ilość dostępnej dokumentacji oraz aktywna społeczność.

Apache CouchDB jest systemem napisanym w języku Erlang i dostępnym na licencji Open Source (Apache License 2.0).

\subsection*{Protokół komunikacji}

Głównym interfejsem komunikacji z CouchDB jest REST po HTTP.
Z jednej strony tego typu interfejs jest zaletą, gdyż jest to prosty protokół, obsługiwany z łatwością w każdym praktycznie języku programowania, a także zastosowanie HTTP umożliwia uzupełnienie CouchDB poprzez zastosowanie serwerów proxy czy load balancerów, których dla tego protokołu jest bardzo wiele.
Zastosowanie tego protokołu umożliwia wręcz tworzenie aplikacji internetowych, które nie wymagają serwera aplikacyjnego - wystarczy klient w języku JavaScript wykonywany w przeglądarce i baza.
Wadą wykorzystania HTTP jako protokołu komunikacji jest to, że w porównaniu z binarnymi protokołami transmisji jest dość wolny, gdyż wymaga większej przepustowości i więcej czasu na serializację i deserializację danych.

Alternatywą dla HTTP jest biblioteka Hovercraft, która pozwala na komunikację z CouchDB przy użyciu mechanizmów dostępnych w języku Erlang.
Biblioteka ta oferuje znacznie lepszą wydajność, ale implementuje tylko część API CouchDB (pozwala na dokonywanie operacji na pojedynczych dokumentach i bazach, ale nie na wyszukiwanie przy pomocy widoków\footnote{patrz dalej.}).
Biblioteka ta ponadto nie jest częścią projektu CouchDB, a zatem istnieje duże ryzyko że jej rozwój nie nadąży za rozwojem bazy.

Opisane dalej BigCouch posiada warstwę komunikacji poniżej protokołu HTTP, która daje dostęp do wszystkich funkcjonalności systemu.
Biblioteka ta wprawdzie nie jest dostępna osobno, ani nie posiada dostępnej publicznie dokumentacji, ale ponieważ jest integralną częścią tego projektu, wydaje się że aplikacje korzystające z BigCouch i napisane w języku Erlang, powinny móc bezpiecznie z niej korzystać.

Dla CouchDB dostępne są biblioteki dla praktycznie wszystkich popularnych języków programowania, jak również biblioteki wysokiego poziomu.
Istnieją także implementacje CouchDB w innych językach niż Erlang, dzięki czemu CouchDB może być instalowane na urządzeniach mobilnych a nawet, wykorzystując mechanizmy HTML5, w przeglądarce internetowej.

\subsection*{Replikacja}

\missingfigure{Replikacja CouchDB}

Jednym z najciekawszych możliwych zastosowań CouchDB jest tworzenie aplikacji rozproszonych, w których baza danych jest zainstalowana na tym samym komputerze co aplikacja kliencka i synchronizuje się z bazą na centralnym serwerze, dzięki czemu można zapewnić działanie takiego systemu nawet w przypadku braku połączenia z siecią, a także ograniczyć opóźnienia postrzegane przez użytkownika.
Aby umożliwić tworzenie aplikacji według tego wzorca, konieczne jest aby baza umożliwiała dwukierunkową replikację jedynie części zbioru danych, najlepiej w połączeniu z systemem kontroli dostępu, tak aby użytkownicy nie mogli nawzajem pobierać swoich danych.
Ponadto potrzebne jest aby system pozwalał na utrzymywanie wielu równoległych połączeń o małej aktywności równocześnie, ponieważ w takiej rozproszonej architekturze każdy klient nawiązuje połączenie z centralnym serwerem na cały czas trwania sesji.

System replikacji w CouchDB posiada wszystkie wymienione powyżej cechy.
Dane mogą być replikowane zarówno jedno-, jak i dwukierunkowo, w sposób ciągły albo jednorazowy.
Dostęp do danych może być ograniczony albo przez ustawienie uprawnień dostępu do konkretnej bazy, albo przy pomocy specjalnych funkcji filtrujących, które pozwalają na wybranie, które dane mogą zostać przesłane.
W przypadku gdy użytkownik powinien mieć uprawnienia do odczytu, ale nie do zapisu można użyć funkcji walidujących zapisywane dokumenty.

Omawiając system uprawnień w CouchDB należy pamiętać jednak, że o ile regulowanie dostępu do konkretnych dokumentów - niezależnie czy chodzi o uprawnienia zapisu czy odczytu - jest nieskomplikowane, to widoki\footnote{Indeksy drugiego poziomu, generowane przy pomocy funkcji map i reduce, których wyniki są zapisywane w B-drzewie.} nie przechowują żadnych informacji o uprawnieniach, tak że jeżeli ograniczony zostanie dostęp do odczytu pewnych danych istnieje ryzyko, że część tych danych będzie dostępna w widokach.
W kontekście replikacji oznacza to, że centralny serwer nie powinien umożliwiać niezaufanym klientom dostępu do widoków.

Ponieważ CouchDB zostało zaimplementowane w Erlangu, pojedynczy węzeł jest w stanie obsłużyć nawet ponad 20 000 równoczesnych połączeń\footnote{http://blog.mattwoodward.com/massive-couchdb-brain-dump}.

Przy okazji replikacji warto także wspomnieć, że poza ,,kanoniczną'' implementacją CouchDB w Erlangu, istnieją także inne implementacje, których zadaniem jest zastąpienie tego systemu w innych środowiskach.
Do takich implementacji należy między innymi CouchOne for Android, czyli implementacja CouchDB dla telefonów z systemem operacyjnym Android, czy BrowserCouch, czyli implementacja CouchDB w języku JavaScript, która wykorzystuje mechanizm lokalnego przechowywania danych (ang. \emph{Local Storage}) wprowadzony przez HTML5 i obsługiwany przez niektóre nowe przeglądarki internetowe.

\subsection*{Partycjonowanie}

CouchDB nie posiada funkcjonalności automatycznego partycjonowania danych, ale architektura tego systemu, która sprawia że dodanie takiej funkcjonalności nie jest bardzo trudne, pozwoliła firmom zainteresowanym horyzontalnym skalowaniem CouchDB na zaimplementowanie takiej funkcjonalności.

Dostępne są aktualnie dwa rozwiązania na licencji Open Source, które umożliwiają horyzontalne skalowanie CouchDB: Lounge, stworzone przez Meebo.com, oraz BigCouch stworzone przez firmę Cloudant.
BBC, które na swoje potrzeby także zaimplementowało takie narzędzie, niestety nigdy go nie udostępniło publicznie.	

\subsubsection*{BigCouch}

Firma Cloudant oferuje usługi hostingowe CouchDB, między innymi ,,CouchDB w Chmurze'' (\emph{CouchDB Cloud}), która to usługa daje klientowi dostęp do bazy hostowanej i zarządzanej przez firmę Cloudant na ich serwerach.
Na potrzeby tej usługi stworzono system o nazwie BigCouch, który pozwala na rozpraszanie CouchDB na wiele węzłów w klastrze. 
BigCouch ułatwił swoim twórcom zarządzanie systemem, pozwalając na to by wszyscy użytkownicy współdzielili jedną jego instancję.
W porównaniu do rozwiązania gdzie na każdych kilku użytkowników przypada osobna, replikowana instancja systemu, redukuje to znacznie koszty administracji, podnosząc równocześnie elastyczność i niezawodność tego rozwiązania.
W sierpniu 2010 Cloudant udostępniło BigCouch na licencji Apache 2.0. 

BigCouch jest rozszerzoną wersją Apache CouchDB, tak samo jak ta baza zaimplementowaną w języku Erlang i w stu procentach kompatybilną\footnote{Oznacza to, że każde zapytanie, które jest poprawnym zapytaniem do oryginalnego CouchDB, jest także poprawnym zapytaniem do BigCouch.} z oryginałem.
BigCouch rozszerza CouchDB poprzez dodanie partycjonowania danych opartego o algorytm Consistent Hashing, znany już z pewnością czytelnikowi z Amazon Dynamo, Apache Cassandra i Riak.

Podobnie jak we wszystkie systemy wzorowane na Amazon Dynamo, także BigCouch pozwala na konfigurowanie trzech parametrów R, W i N (odpowiednio: liczba węzłów które muszą zwrócić wartość przy odczycie, liczba węzłów które muszą potwierdzić zapis i na ile węzłów musi być zreplikowany dokument w bazie).
Ponadto tworząc bazę można podać też parametr Q, który określa liczbę przedziałów, na jakie zostanie podzielony zbiór kluczy.

BigCouch jest młodym projektem, dlatego brakuje mu części funkcjonalności, które zostały opisane w artykule o Amazon Dynamo.

Najbardziej znaczącym brakiem jest niezaimplementowany mechanizm zmniejszania entropii (w Amazon Dynamo zapewniany przez Merkle Trees\footnote{patrz strona \pageref{merkle-trees}.}).
Brak takiego mechanizmu sprawia, że w przypadku zmiany przyporządkowania przedziałów kluczy\footnote{W nomenklaturze BigCouch: \emph{shards}.} do węzłów, cały przedział może potencjalnie nigdy nie zostać przekazany do nowo dodanego węzła, ponieważ dokumenty są przekazywane między węzłami tylko dzięki mechanizmowi \emph{Read Repair} (czyli kiedy przy odczycie okazuje się, że nie wszystkie węzły odpowiedzialne za dany przedział posiadają tą samą wersję żądanego dokumentu).
Dodatkowym efektem tego braku, jest to że BigCouch nie implementuje także mechanizmu Hinted Handoff\footnote{patrz strona \pageref{sec:dynamo-hinted-handoff}.}, co potencjalnie jeszcze bardziej zmniejsza dostępność systemu.

BigCouch nie implementuje mechanizmu zegarów wektorowych, ale dzięki wbudowanemu w CouchDB mechanizmowi wersjonowania i rozwiązywania konfliktów (patrz dalej), nie jest to niezbędne.

Przydział przedziałów kluczy do węzłów nie jest rozpowszechniany tak jak w innych systemach przy pomocy protokołu plotkowania (ang. \emph{gossip protocol}).
Zarówno lista aktywnych węzłów w systemie, jak i przydziały kluczy do węzłów są przechowywane w specjalnej bazie jako dokumenty i replikowane przy zastosowaniu normalnych mechanizmów CouchDB. 

BigCouch nie posiada żadnych wbudowanych mechanizmów pozwalających na zmianę przydziału przedziałów kluczy przy dodaniu albo usunięciu węzła z klastra.
Ponieważ jednak każdy taki przedział wewnętrznie jest zapisywany w osobnej bazie, a same przydziały są normalnymi dokumentami, to stworzenie własnych skryptów, które dokonają replikacji odpowiednich przedziałów a potem przypiszą te przedziały do nowo dodanego węzła jest proste i nie wymaga dogłębnej znajomości BigCouch czy nawet języka Erlang.

\subsubsection*{Lounge}

CouchDB Lounge jest starszym rozwiązaniem pozwalającym na horyzontalne skalowanie CouchDB.
Składa się z dwóch aplikacji: modułu serwera NGINX pozwalającego na obsługę zapytań dotyczących pojedynczych dokumentów i aplikacji napisanej w Pythonie/Twisted, która pozwala na agregację wyników działania widoków.

Lounge pod wieloma względami przypomina BigCouch.
Głównymi różnicami jest to, że w Lounge użytkownik nie ma kontroli nad parametrami R i W, oraz że konfiguracja listy węzłów i liczby przedziałów odbywa się przy pomocy plików konfiguracyjnych zamiast być trzymana w bazie.

\subsection*{Persystencja}

CouchDB przechowuje dane w strukturze $B^+-tree$ - rozłożystym drzewie, umożliwiającym szybkie pobieranie rekordów, nawet gdy baza zawiera ich miliony czy miliardy.
$B^+-tree$ ma bardzo małą wysokość - poniżej 10 nawet dla milionów rekordów.
Wyszukanie rekordu w takim drzewie wymaga przejścia ścieżki od korzenia do liścia (wewnętrzne węzły drzewa nie zawierają rekordów, jedynie klucze).
Ponieważ w praktyce węzły pośrednie są buforowane w pamięci, bardzo często wystarczy jedno wyszukanie na dysku aby odczytać dokument w CouchDB.

Struktury danych CouchDB są zapisywane na dysku jedynie przez dopisywanie na koniec pliku.
Dzięki temu, w razie awarii, po ponownym uruchomieniu systemu, nie jest konieczne wykonywanie czasochłonnego sprawdzania plików z danymi - system jest gotów do działania praktycznie natychmiast.
Warto także zwrócić uwagę, że plik ten nie jest dziennikiem transakcji (jak np. w Google BigTable) - jest to zapisane na dysku $B^+-tree$.

Wadą dopisywania na koniec pliku jest to, że pliki z danymi CouchDB potrafią urosnąć do bardzo dużych rozmiarów w stosunku do liczby dokumentów w bazie.
Aby temu zaradzić możliwe jest upakowanie tych plików (ang. \emph{compaction}), ale operacja ta w przypadku bardzo obciążonych serwerów, gdy dodawanie zmian do nowego pliku nie nadąża za dodawaniem ich do oryginalnego pliku, może się nigdy nie skończyć.

Domyślnie CouchDB nie wykonuje operacji \verb+fsync()+ przy każdym zapisie, a jedynie co sekundę.
Możliwe jest też dokonywanie tej operacji przy każdej zmianie, ale ponieważ CouchDB wykonuje wtedy dwie operacje \verb+fsync()+ przy każdym dopisaniu na koniec pliku, to pojedyncze operacje trwają wtedy bardzo długo (ale przepustowość systemu na tym nie cierpi, więc gdy operacje te są dokonywane asynchronicznie, lub jeżeli są dokonywane przez wielu klientów na raz, to ogólna liczba operacji na sekundę się nie zmienia).
Dodatkowo, możliwe jest też wykonywanie operacji \verb+fsync()+ explicite, oraz wyłącznie lub włączanie jej wykonania na potrzeby pojedynczej operacji. 

\subsection*{Wersjonowanie}

W CouchDB każdy dokument posiada numer wersji (\verb+_rev+).
Każdy numer wersji składa się z liczby, która zwiększa się o jeden za każdym razem gdy dokument zostanie zmieniony, oraz z ciągu znaków, który jest wartością funkcji mieszającej MD5 liczoną z treści dokumentu, jego załączników i flag).

Aktualizując dokument w CouchDB należy podać jego aktualny numer wersji, jeżeli numer ten nie zostanie podany, albo jeżeli będzie inny niż najnowszy, to aktualizacja zostanie odrzucona.

Jeżeli dokument zostanie zmieniony na dwa różne sposoby na dwóch różnych serwerach\footnote{Albo w dwóch różnych bazach na tym samym serwerze.} to wykrywane zostaje to w czasie replikacji.
W przypadku wystąpienia kilku konkurujących ze sobą wersji dokumentu, w sposób deterministyczny jedna z nich zostaje oznaczona jako aktualna, a równocześnie wszystkie pozostałe wersje są zapisywane i dokument jest oznaczany jako pozostający w konflikcie.
Do czasu rozwiązania konfliktu przez użytkownika, jedynie aktualna wersja jest brana pod uwagę przez widoki.
Wszystkie skonfliktowane wersje są replikowane.

Rozwiązanie konfliktu polega na usunięciu niepoprawnych wersji i zaktualizowaniu dokumentu podając jego poprawną postać.

CouchDB nie przechowuje historycznych wersji dokumentów - nie są one przekazywane w czasie replikacji i są usuwane w czasie pakowania plików z danymi.

\subsection*{Wyszukiwanie}

CouchDB posiada unikalny mechanizm budowania indeksów drugiego poziomu przy wykorzystaniu MapReduce.
Posiada także funkcje pozwalające na transformację zapisanych dokumentów do innych formatów niż JSON.

\subsubsection*{Widoki}

Większość systemów NoSQL oferuje albo integrację z zewnętrznymi implementacjami MapReduce (tak jak Cassandra czy HBase integrują się z Hadoop), albo posiada własną implementację tego wzorca (tak jak Riak czy MongoDB).
Wspólną cechą wszystkich tych systemów jest jednak to, że wyniki operacji MapReduce są wyliczane ad hoc, a co za tym idzie, przy dużych zbiorach danych obliczenia te mogą trwać długo - zbyt długo aby możliwe było ich wyliczanie za każdym razem gdy są potrzebne do zaprezentowania użytkownikowi.
Z tego względu w systemach tych zazwyczaj wyniki operacji MapReduce są zapisywane w bazie skąd mogą one być odczytywane na potrzeby użytkowników, a obliczenia są ponawiane co pewien czas.

CouchDB różni się od tych systemów tym, że wyniki operacji Map i Reduce są zapisywane na dysku w strukturze $B^+-tree$.
Parę funkcji map i reduce zapisanych w bazie nazywamy widokiem.
Ponieważ wyszukiwanie po kluczu emitowanym przez funkcję map jest bardzo efektywne, a także możliwe jest wyszukiwanie używając zakresu tych kluczy, widoki są bardzo efektywnym mechanizmem wyszukiwania w CouchDB.

Ponadto funkcje reduce mogą być wykorzystywane do szybkiego wykonywania operacji grupujących, takich jak suma czy średnia.

Wyniki działania funkcji map i reduce są obliczane i zapisywane przy pierwszym odwołaniu do widoku.
Jeżeli pomiędzy wywołaniami widoku zajdą zmiany w bazie, to funkcje te zostaną przeliczone tylko dla tych dokumentów, dla których jest to niezbędne.

Widoki nie są replikowane (replikowany jest jedynie dokument zawierający definicje widoków), dlatego każdy serwer musi je wyliczać samodzielnie.

W przypadku BigCouch, widoki są wyliczane odrębnie dla każdego przedziału kluczy, a następnie strumieniowane po jednym z każdego przedziału do węzła koordynującego, który wykonuje merge-sort na tych danych.

\subsubsection*{Funkcje Show i Listy}

Jedną z zalet Apache CouchDB jest to, że możliwe jest wykorzystanie tego systemu w aplikacjach internetowych napisanych w języku JavaScript bez potrzeby wykorzystywania warstwy pośredniej między bazą a klientem w postaci serwera aplikacyjnego (albo przynajmniej z bardzo uproszczonym serwerem aplikacyjnym, który pełni głównie rolę serwera proxy).

Na potrzeby takich aplikacji w CouchDB dodane zostały funkcjonalności transformowania rekordów do innych formatów niż JSON.
Dostępne są dwa warianty tych funkcji: wyświetlające pojedyncze dokumenty, czyli funkcje Show oraz wyświetlające widoki, czyli Listy.
Podobnie jak widoki, Listy i funkcje Show są zapisane w bazie.

Oba typy funkcji transformujących używają mechanizmu Etags, który pozwala na buforowanie ich wyników przez serwery proxy.

\subsection*{Unikalne cechy}

W poprzednich sekcjach zostało już opisane kilka mechanizmów, które wyróżniają CouchDB, takie jak inkrementalne MapReduce, zaawansowane mechanizmy replikacji czy ułatwienia, które pozwalają na pominięcie serwera aplikacyjnego w architekturze aplikacji.
W tej sekcji przedstawiony zostanie mechanizm wewnętrznie używany przez replikację, ale przydatny także do wielu innych celów.

\subsubsection*{Lista Zmian (\_changes Feed)}

Opisana kilka sekcji wyżej funkcjonalność replikacji oferuje szereg ciekawych opcji, takich jak na przykład filtrowanie replikowanych dokumentów.
O ile jednak replikacja działa pomiędzy instancjami CouchDB, to Lista Zmian pozwala każdej aplikacji na otrzymywanie powiadomień o zmianach w bazie.

Lista Zmian znajduje się pod specjalnym adresem \verb+_changes+ (stąd nazwa).
Jako argumenty można przekazać do niej numer sekwencyjny pierwszej zmiany, która nas interesuje.

Istnieje także kilka trybów, w których lista ta może pracować.
Domyślnie zwracana jest lista zmian i połączenie jest zamykane - niezależnie czy jakieś zmiany nastąpiły, czy nie.
Możliwe jest także wyspecyfikowanie, że serwer ma czekać z odpowiedzią, aż zajdą jakieś zmiany, a wtedy je zwrócić i zamknąć połączenie.
Dodatkowo dostępny jest także tryb ciągły, czyli tryb, w którym serwer nigdy nie zamyka połączenia, a jedynie wysyła klientowi zmiany w takim tempie w jakim one zachodzą.

Zmiany zwracane przez listę zmian, mogą być filtrowane przy użyciu tych samych filtrów, co przy replikacji.

Lista zmian ma wiele zastosowań.
Jednym z najbardziej podstawowych jest umożliwienie zewnętrznym systemom na synchronizację danych z CouchDB.
Przykładem tego zastosowanie jest na przykład couchdb-lucene, które w ten sposób pobiera dokumenty, które następnie są indeksowane przy pomocy Apache Lucene, co pozwala na wyszukiwanie pełnotekstowe.
Innym zastosowaniem listy zmian, jest potraktowanie jej jako mechanizmu \emph{server-push} na potrzeby aplikacji internetowych (umożliwia to na przykład na napisanie aplikacji typu chat przy użyciu CouchDB).

\subsection*{Typowe zastosowania}

Najmocniejszą stroną CouchDB jest replikacja i możliwość tworzenia zdecentralizowanych aplikacji - głównie dlatego, że pod tym względem system ten nie posiada konkurencji.
Przykładem takiej aplikacji jest Ubuntu One - projekt pozwalający użytkownikom na synchronizację różnorakich ustawień pomiędzy instalacjami systemu operacyjnego.

Oparcie się o protokół HTTP i język JavaScript sprawia, że CouchDB jest także interesującym projektem z punktu widzenia twórców aplikacji internetowych, gdyż pozwala na zmniejszenie impedancji pomiędzy różnymi warstwami aplikacji (mniej różnych języków programowania, dokumentowy model danych bardziej przystaje do obiektowego niż model relacyjny).

\subsection*{Przeciwwskazania}

CouchDB, mimo niezłych wyników jeżeli chodzi o ogólną przepustowość systemu, nie należy do najszybszych jeżeli chodzi o opóźnienia pojedynczych zapytań.
Dlatego, baza ta najbardziej się nadaje do zastosowania z aplikacjami, które odczytują i zapisują co najwyżej po kilka dokumentów na stronę, lub z aplikacjami typu ,,gruby klient'', które wymagają mniejszej liczby interakcji z systemem bazodanowym.

Duże potrzeby dyskowe i brak wbudowanego mechanizmu partycjonowania danych sprawia, że CouchDB nie należy do systemów pozwalających na pracę z miliardami dokumentów równocześnie (chociaż wraz z dojrzewaniem rozwiązań partycjonujących bazę może się to zmienić).
Nie należy się spodziewać także, że CouchDB kiedykolwiek będzie systemem, w którym dynamiczne dodawanie i usuwanie węzłów w celu skalowania systemu w górę i w dół będzie proste - ze względu konieczność budowania widoków od nowa na każdym dodanym węźle\footnote{Oczywiście można także skorzystać z tego, że pliki z widokami/bazą mogą być po prostu kopiowane, ale nawet wtedy trzeba pamiętać że pliki te mogą osiągać rozmiary dziesiątek czy setek gigabajtów.}. 

\subsection*{Dokumentacja i wsparcie}

CouchDB należy do jednych z najlepiej udokumentowanych systemów spośród opisywanych w tej pracy.
Dostępnych jest bardzo wiele prezentacji wideo, artykułów i porównań z innymi systemami, a także kilka książek, z których jedna: ,,CouchDB: The Definitive Guide'' jest dostępna on-line i za darmo i stanowi doskonałe wprowadzenie w wiele z zagadnień dotyczących tego systemu.

CouchDB jest bardzo popularnym projektem, co widać po liczbie różnych bibliotek i narzędzi wykorzystujących i przeznaczonych dla tej bazy.

CouchDB jest komercyjnie wspierane przez firmę CouchOne.
Dostępny jest także hosting oferowany przez CouchOne i Cloudant.

\subsection*{Pomocne odnośniki}

Poniżej zamieszczono kilka odnośników do stron WWW związanych z CouchDB.

\begin{description}
 \item [http://couchdb.apache.org/] - strona domowa projektu
 \item [http://guide.couchdb.org/] - ,,CouchDB: The Definitive Guide'', świetne wprowadzenie do CouchDB
 \item [http://wiki.apache.org/couchdb/] - wiki CouchDB, podobnie jak w innych wiki projektów Apache struktura nie jest zbyt przejrzysta, ale znajduje się tu wiele ciekawych informacji
 \item [http://en.wikipedia.org/wiki/B-tree] - artykuł na Wikipedii na temat struktury danych $B-tree$, wykorzystywanej w CouchDB 
 \item [https://github.com/cloudant/bigcouch] - strona domowa projektu BigCouch
 \item [http://hg.toolness.com/browser-couch/raw-file/blog-post/index.html] - strona domowa projektu BrowserCouch
 \item [http://oreillynet.com/pub/e/1760] - prezentacja wideo na temat BigCouch
\end{description}

\section{MongoDB}
\label{sec:mongodb}

\subsection*{Wstęp} 

\subsection*{Protokół komunikacji}

\subsection*{Replikacja}

\subsection*{Partycjonowanie}

\subsection*{Persystencja}

\subsection*{Wersjonowanie}

\subsection*{Wyszukiwanie}

\subsection*{Unikalne cechy}

\subsection*{Typowe zastosowania}

\subsection*{Przeciwwskazania}

\subsection*{Dokumentacja i wsparcie}

\subsection*{Pomocne odnośniki}

\chapter{Bazy Grafowe}

\section*{Streszczenie}
W tym rozdziale opisane zostało Neo4J -- najbardziej znana i dojrzała grafowa baza danych.
System ten posiada niemalże monopol na rynku grafowych baz danych na licencji Open Source, ale mimo to popularnością znacznie ustępuje większości, jeżeli nie wszystkim opisanym wcześniej systemom NoSQL.
Prawdopodobnie wiąże się to z licencją tego produktu, która sprawia, że wykorzystanie Neo4J w typowej aplikacji internetowej wiąże się albo z koniecznością kupna licencji, albo z koniecznością udostępnienia kodu aplikacji.

\section{Neo4J}
\label{sec:neo4j}

\subsection*{Wstęp}

Neo4J jest najbardziej znaną i popularną grafową bazą danych.
Jest to także jeden z najstarszych systemów NoSQL - system ten jest stosowany w środowiskach produkcyjnych od 2003 roku.
W 2007 roku baza ta została udostępniona publicznie na licencji Open Source (Affero General Public License).

Ponieważ w odróżnieniu od wielu innych systemów Neo4J jest bazą wbudowaną, licencja AGPL sprawia, że oprogramowanie z niej korzystające także musi być udostępnione na tej licencji (chyba że jest to aplikacja do użytku wewnętrznego).
Dla firm, które chciałyby wykorzystać Neo4J do budowy swoich aplikacji, ale nie chcą udostępniać ich kodu źródłowego, istnieje możliwość wykupienia licencji komercyjnej\footnote{Dostępne są trzy płatne wersje, wszystkie płatne według modelu subskrypcyjnego za każdy serwer. Pierwsza, najtańsza wersja jest za darmo.}.

\subsection*{Protokół komunikacji}

Neo4J jest wbudowaną bazą danych, czyli nie wymaga komunikacji sieciowej pomiędzy klientem i serwerem, ponieważ baza jest częścią tego samego procesu, co aplikacja z niej korzystająca.
Takie rozwiązanie jest bardzo korzystne dla grafowych baz danych, ponieważ dzięki temu możliwe jest implementowanie różnego rodzaju algorytmów trawersujących, które wczytują z bazy wierzchołek po wierzchołku, co nie byłoby praktyczne, gdyby każda taka operacja wiązała się z opóźnieniem związanym z przesyłaniem danych przez sieć.

Mankamentem tego rozwiązania jest to, że aplikacje korzystające z Neo4J muszą być uruchamiane w Wirtualnej Maszynie Java (JVM).
Ponieważ na JVM możliwe jest programowanie w wielu innych językach niż Java (m.in. Groovy, Ruby, Python, JavaScript), to przywiązanie do tej platformy nie jest bardzo poważnym problemem, ale tym niemniej jest to pewne ograniczenie.

Poza trybem wbudowanym, dostępne jest też REST API po HTTP.
Pozwala ono wprawdzie korzystać z Neo4J aplikacjom napisanym na inne platformy niż JVM, ale jest to niewątpliwie mniej efektywne rozwiązanie.

\subsection*{Replikacja}

\myfigure{chapters/bazy_grafowe/neo4j-replikacja.png}{Neo4J Replikacja}{fig:neo4j-replikacja}

Od wersji 1.2 Neo4J posiada możliwość replikacji typu master-slave\footnote{Funkcjonalność ta jest dostępna tylko w wersji Open Source i najdroższej wersji komercyjnej.}.
Funkcjonalność ta przypomina bardzo Zbiory Replik w MongoDB\footnote{patrz strona \pageref{sec:mongodb-replication}.}, z tą różnicą że Neo4J wykorzystuje Apache Zookeeper\footnote{Jest to system Open Source bardzo przypominający Google Chubby, który został opisany na stronie \pageref{sec:google-chubby}.} do wyboru serwera master, podczas gdy MongoDB wykorzystuje mechanizm głosowania pomiędzy wszystkimi dostępnymi węzłami.

Neo4J jest wbudowaną bazą danych, w związku z tym każdy serwer aplikacyjny jest także repliką bazy danych.
Fakt że baza jest replikowana pozostaje niewidoczny z punktu widzenia aplikacji.
W przypadku zapisu, węzeł slave kontaktuje się z serwerem master i czeka aż potwierdzi on zapis danych, a następnie wprowadza modyfikacje lokalnie.
Przy każdym zapisie zanim operacja zwróci przetwarzanie do klienta, dane są zapisane na przynajmniej dwóch węzłach.
W ten sposób z punktu widzenia serwera aplikacyjnego zapewniona jest konsystencja na poziomie odczytu własnych zapisów (ang. \emph{ready-your-writes consistency}).

Taka architektura pozwala na skalowanie w zastosowaniach o znacznej przewadze odczytów do liczby zapisów.
Warto jednak pamiętać, że konfiguracja ta wymaga nieparzystej liczby instancji Zookeeper, czyli zalecane jest wykorzystywanie jej w konfiguracjach od trzech węzłów wzwyż.

\subsection*{Partycjonowanie}

Bardzo trudno jest partycjonować grafowe bazy danych ze względu na ich wysoki stopień powiązania.
Nawet znalezienie klastrów wysokopowiązanych między sobą wierzchołków, które nadają się do rozproszenia na różne maszyny nie gwarantuje, że nowo utworzone połączenia i wierzchołki nie zaburzą tego stanu, wymuszając tym samym migrację danych między serwerami.

W chwili obecnej Neo4J nie posiada wbudowanego mechanizmu partycjonowania danych.
Według zapowiedzi twórców tego systemu funkcjonalność taka ma się pojawić w wersji 2.0, której data wydania nie jest jeszcze znana.

\subsection*{Persystencja}

Dokumentacja Neo4J nie zagłębia się w szczegóły persystencji w tym systemie.
Na podstawie faktu, że baza ta jest transakcyjna i że przerwane przez awarię transakcje są odtwarzane na podstawie dziennika transakcji, można wnioskować że najprawdopodobniej zmiany w bazie są trwale zapisywane na dysku po zakończeniu transakcji.

\subsection*{Wersjonowanie}

W Neo4J wszystkie zmiany w grafie muszą być objęte transakcją.
Domyślnie, poziom izolacji transakcji zapewnia, że zmiany dokonane w jej obrębie nie będą widoczne dla innych równoległych transakcji aż zostanie ona poprawnie zakończona.
Oznacza to, że długotrwała operacja trawersowania grafu może obserwować zmiany w grafie, które zostały dokonane po jej rozpoczęciu.
Istnieje możliwość ustawienia także wyższego poziomu izolacji, ale jest to kosztowne.

Neo4J nie obsługuje wersjonowania węzłów ani krawędzi.

\subsection*{Wyszukiwanie}

Neo4J jako baza grafowa koncentruje się przede wszystkim na ułatwieniu operacji trawersowania grafu i realizacji algorytmów grafowych, ale aby ułatwić korzystanie z tej bazy jej twórcy dodali także pewne udogodnienia pozwalające na wyszukiwanie wierzchołków i krawędzi na podstawie ich właściwości.

\subsubsection*{Model Danych}

Neo4J służy do przechowywania skierowanych grafów, gdzie każdy wierzchołek i każda krawędź mogą być opisane zestawem właściwości.
Zarówno węzły, jak i krawędzie mają unikalne identyfikatory, które mogą być użyte do ich wczytania z bazy.
Modelowanie symetrycznych relacji odbywa się poprzez dodanie dwóch przeciwnie skierowanych krawędzi między wierzchołkami.

\subsubsection*{Traversal API}

Podstawowym mechanizmem wyszukiwania w grafie jest API Trawersowania (ang. \emph{Traversal API}).
Umożliwia ono deklaratywne określanie parametrów wyszukiwania w grafie, takich jak w jaki sposób wybierać krawędzie i węzły do przejścia, które węzły zwrócić, do jakiej głębokości przeszukiwać graf i czy przechodzić po nim używając algorytmu DFS czy BFS\footnote{\emph{Depth First Search} czy \emph{Breadth First Search}.}.

Takie trawersowanie grafu jest bardzo szybkie, bo wczytuje właściwości węzłów tylko wtedy, gdy jest to konieczne.
Ponadto, użytkownikowi zwracany jest iterator, a nie kompletna lista wyników, co zmniejsza wymagania pamięciowe takiego przeszukiwania.

\subsubsection*{Indexer}

Dla ułatwienia wyszukiwania węzłów i krawędzi, których właściwości spełniają pewne kryteria, Neo4J zostało zintegrowane z Apache Lucene.
Dzięki temu, możliwe jest nie tylko wyszukiwanie pełnotekstowe, ale także wyszukiwanie po wartościach właściwości z wykorzystaniem operatorów logicznych (NOT, AND, OR) oraz podając zakresy wartości.

\subsubsection*{Neo4J Spatial}

Struktura danych składająca się z wierzchołków, skierowanych krawędzi i właściwości nadaje się do reprezentowania bardzo wielu różnych struktur danych.
Przykładem tego jest dodatek do Neo4J pozwalający na przechowywanie w grafie indeksu geograficznego, pozwalającego między innymi na wyszukiwanie wierzchołków i krawędzi w określonej odległości od punktu o podanych współrzędnych oraz na wyszukiwanie najbliższych sąsiadów.

\subsubsection*{Standardowe języki zapytań}

Poza API specyficznym dla Neo4J, istnieją jeszcze inne możliwe sposoby pracy z tą bazą.
Jednym z nich jest zestaw narzędzi przeznaczonych dla grafowych baz danych TinkerPop.
Do tego zestawu należy między innymi Blueprints API, czyli uniwersalny interfejs pozwalający na tworzenie struktury grafu obsługujący kilka różnych baz grafowych.
Innym projektem należącym do tej grupy jest Gremlin, określany czasem jako ,,PERL dla grafowych baz danych''.
Jest to język programowania, oparty na języku Groovy, który pozwala na wykorzystywanie specjalnych wyrażeń (przypominających nieco XPath) do trawersowania grafu.

\subsubsection*{Inne ciekawe dodatki}

Dla wygody użytkownika, istnieje kilka algorytmów i struktur danych zaimplementowanych w oparciu o Neo4J.
Do algorytmów należą algorytmy wyszukiwania najkrótszej ścieżki w grafie ($A*$ i $Dijkstra$), a do struktur danych należy między innymi drzewo $B-tree$.

\subsection*{Unikalne cechy}

W odróżnieniu od innych systemów opisanych w tej pracy Neo4J implementuje interfejs pozwalający na masową zmianę struktury danych (migracje).
Jest to alternatywna dla stosowanego w innych systemach podejścia polegającego na obsłudze wielu wersji danych w kodzie aplikacji.

\subsection*{Typowe zastosowania}

Neo4J to najbardziej dojrzała i popularna grafowa baza danych.
Bardzo dużą jej zaletą jest łatwość modelowania nawet bardzo skomplikowanych relacji domenowych, szczególnie w porównaniu z relacyjnymi bazami danych, a także łatwość odwzorowania wielu struktur danych.
Dodatkową zaletą tej bazy jest prostota wyszukiwania danych przy pomocy dynamicznie tworzonych zapytań zapewniana przez integrację z Lucene.

Neo4J nadaje się bardzo dobrze do zastosowań wymagających rozwiązywania problemów grafowych, takich jak silniki rekomendacji, narzędzia do wykrywania oszustw i wzorców w sieciach, czy też na potrzeby sieci społecznościowych.
Baza ta może stanowić dobrą alternatywę dla relacyjnych baz danych w zastosowaniach gdzie domena problemu jest skomplikowana, często się zmienia, a zapytania do bazy często łączą więcej niż dwie tabele na raz.

\subsection*{Przeciwwskazania}

Brak partycjonowania w Neo4J sprawia, że choć baza ta skaluje się do miliardów węzłów i krawędzi, to jednak może to nie być wystarczająco dużo dla rozbudowanych grafów.
Podobnie też nie jest to najlepsze rozwiązanie w przypadku domen, gdzie występują luźne i nieliczne powiązania.
W przypadku już istniejących systemów, które nie są uruchamiane na platformie Java jedynym rozwiązaniem jest zastosowanie interfejsu REST udostępnianego przez Neo4J, co wiąże się ze zmniejszoną wydajnością i mniejszą przeźroczystością warstwy persystencji.

\subsection*{Dokumentacja i wsparcie}

Neo4J jest wprawdzie dojrzałą bazą danych i prawdopodobnie najpopularniejszym systemem wśród grafowych baz danych, ale tym niemniej nie należy do najlepiej udokumentowanych systemów NoSQL.
Głównym źródłem wiedzy o Neo4J jest oficjalne wiki projektu, które opisuje głównie aspekty użytkowe bazy, czyli API i możliwości konfiguracji, ale nie poświęca wiele miejsca architekturze tej bazy i jej szczegółom implementacyjnym.
W internecie można znaleźć kilka prezentacji wideo na temat Neo4J, ale ich treść jest praktycznie identyczna.
Projekt ten też wydaje się rozwijać wolniej niż inne opisywane w tej pracy.
Prawdopodobnie jest to związane z jego komercyjną naturą.

Neo4J jest rozwijane i wspierane przez szwedzką firmę Neo Technology, podobnie jak inne formy tego typu oferuje ona szkolenia, wsparcie i pomoc w migracji z innych systemów bazodanowych.

\subsection*{Pomocne odnośniki}

Poniżej zamieszczono kilka odnośników do stron WWW związanych z Neo4J:

\begin{description}
 \item [http://neo4j.org/] - Strona domowa projektu
 \item [http://components.neo4j.org/] - Dodatkowe komponenty
 \item [http://www.tinkerpop.com/] - Zestaw narzędzi TinkerPop
 \item [http://neotechnology.com/] - Strona domowa Neo Technology 
 \item [http://vimeo.com/9852755 oraz http://vimeo.com/9859000] - Prezentacja wideo na temat Neo4J (najdłuższa, najbardziej rozbudowana z dostępnych wersji)
 \item [https://github.com/andreasronge/neo4j] - Biblioteka obudowująca Neo4J w JRuby
\end{description}

\chapter{Skalowalne Bazy Relacyjne}

\section*{Streszczenie}
W tym rozdziale opisana zostanie skalowalna bazy relacyjna: VoltDB.

\section{VoltDB}
\label{sec:voltdb}
\chapter{Wnioski}

\section*{Streszczenie}

W poprzednich rozdziałach opisane zostały różnego rodzaju nierelacyjne bazy danych, a także niektóre wzorce i systemy, które wpłynęły na architekturę tych baz.
W tej części podsumowane zostaną podobieństwa między tymi systemami oraz techniki, które w nich zostały wykorzystane. 

\section{Replikacja}

W dzisiejszych czasach system bazodanowy nie ma prawa posługiwać się tą nazwą jeżeli nie wspiera replikacji, a zatem każdy z systemów opisanych w tej pracy wspiera tą funkcjonalność w jednej lub więcej z jej odmian.

Współczesne systemy rozproszone coraz częściej wykorzystują replikację aby zapewnić trwałość danych przy zachowaniu akceptowalnej wydajności.
Alternatywą dla takiego podejścia jest stosowanie macierzy RAID z pamięcią cache, która jest podtrzymywana przy pomocy baterii.
To drugie podejście wprawdzie pozwala na redundantne przechowywanie danych, ale w wypadku awarii serwera, w którym ta macierz jest umieszczona, system bazujący na tym rozwiązaniu utraci dostęp do tej części danych, a jej przywrócenie wymaga interwencji człowieka.

Wykorzystanie replikacji w celu zapewnienia trwałości danych pozwala na zachowanie dostępności systemu w razie awarii jednego z węzłów.
Oczywiście stopień trwałości danych w przypadku awarii dużej skali zależy od mechanizmu i konfiguracji persystencji - systemy, które zapisują dane na dysk co minutę, nigdy nie będą w stanie zapewnić pełnej trwałości danych\footnote{Zapewnienie pełnej trwałości danych jest problemem dużo bardziej skomplikowanym i zależy od wielu innych czynników poza tym, jak często dane są zapisywane na dysk}, ale system, który jest w stanie dokonywać replikacji pomiędzy centrami danych, na pewno będzie bardziej odporny na utratę danych niż system, który bazuje tylko na macierzach RAID.

Pomijając jednak to zastosowanie replikacji, jest ona stosowana przede wszystkim aby umożliwić rozłożenie obciążenia systemu między wiele węzłów.
Konfiguracje opierające się na architekturze Master-Slave są w stanie w ten sposób podzielić odczyty między wszystkie zaangażowane węzły, podczas gdy zapisy są obsługiwane tylko przez jeden z serwerów.
Konfiguracja Master-Master pozwala na rozłożenie zarówno odczytów, jak i zapisów na wiele serwerów, ale odbywa się to kosztem albo dostępności, albo konsystencji systemu.

\subsection*{Replikacja Master-Slave}

Najbardziej prymitywnym wariantem replikacji spośród napotkanych w opisanych w tej pracy systemach jest prosta replikacja Master-Slave.
Przez pojęcie ,,prosta'' rozumiana jest tu funkcjonalność replikacji, która nie umożliwia konfiguracji tego, czy odbywa się ona synchronicznie czy asynchronicznie (a w szczególności na ilu węzłach zapis musi się powieść aby został uznany za udany), a w wypadku awarii węzła Master, system nie jest w stanie samodzielnie podjąć decyzji o wyborze nowego węzła master.

Systemem, który implementuje replikację w tym wariancie jest Redis, który w obecnej wersji jest wciąż jeszcze systemem praktycznie pozbawionym możliwości skalowania horyzontalnego.

MongoDB posiada wprawdzie od niedawna funkcjonalność Zbiorów Replik, ale ponieważ aktualnie nie można tej funkcjonalności wykorzystywać w połączeniu z autentykacją, to większość użytkowników wykorzystuje starszą funkcjonalność, która nie umożliwia automatycznego wyboru serwera master.
System ten oferuje jednak szerokie możliwości konfiguracji replikacji, w tym określanie liczby węzłów na których musi zostać wykonany zapis oraz opóźnienie replikacji, które pozwala na tworzenie kopii zapasowych.

\subsection{Replikacja Master-Slave z automatycznym wyborem serwera Master}

Administracja systemem rozproszonym na wiele węzłów może być kosztowna.
Z tego względu większość systemów, które reklamują się jako horyzontalnie skalowalne, implementuje funkcjonalności, które mają za zadanie ułatwienia życia ich administratorom.
Przykładem takiej funkcjonalności jest wybór nowego węzła Master w przypadku awarii jego poprzednika.

Systemy opisane w tej pracy realizują tą funkcjonalność na różne sposoby.
Podstawową różnicą w tym przypadku jest to, czy system dokonuje wyboru nowego wyróżnionego węzła samodzielnie, czy opiera się w tym celu na rozproszonym systemie blokad.

Systemem, w którym wybór węzła master polega na głosowaniu przeprowadzanym pośród wszystkich dostępnych węzłów, jest MongoDB.
W systemie tym, węzeł master jest wybierany z pośród wszystkich węzłów należących do zbioru replik.
Jeżeli decyzja nie może być podjęta, gdyż dostępna jest tylko połowa lub mniej węzłów, to dokonywanie zapisów staje się niemożliwe, ale system pozostaje dalej dostępny dla operacji odczytu.
W przypadku gdy w MongoDB jest wykorzystywana funkcjonalność automatycznego partycjonowania danych, za metadane o systemie odpowiadają tak zwane węzły (a właściwie procesy) konfiguracyjne.
Aby jakakolwiek zmiana w metadanych została wykonana, wszystkie te węzły muszą być aktywne.

Do systemów wybierających węzeł master automatycznie przy pomocy systemu blokad Apache Zookeeper należą HBase i Neo4J.
W tym drugim systemie, każdy zapis jest inicjowany przez węzeł lokalny dla aplikacji (ze względu na to, że baza ta jest bazą wbudowaną), a koordynacją wszystkich zapisów zajmuje się węzeł master, który musi potwierdzić każdy zapis aby się on powiódł.
Neo4J nie obsługuje partycjonowania danych, a zatem jedynym rodzajem wyróżnionego węzła w tym systemie jest węzeł master w replikacji.

W przypadku HBase, sytuacja wygląda odmiennie, ponieważ system ten rozdziela odpowiedzialność za serwowanie danych od odpowiedzialności za ich przechowywanie, a także posiada dodatkowy wyróżniony węzeł odpowiadający za metadane całości systemu.
Zarówno wyróżnione serwery odpowiadające za serwowanie przedziałów rekordów, jak i węzeł master odpowiadający za ogół metadanych systemu są wybierane przy pomocy Apache Zookeeper.
Na poziomie replikacji, która w HBase jest realizowana przez Hadoop File System, za wybór wyróżnionych węzłów odpowiadających za określony plik odpowiada węzeł NameNode.
W przypadku awarii tego węzła, cały system ulega awarii.

\subsection*{Replikacja Master-Master}

Wykrycie awarii węzła master i wybór nowego, jest operacją, która może potrwać nawet kilkadziesiąt sekund.
W wypadku replikacji, oznacza to że system (a przynajmniej ta część danych, za którą odpowiadał ten węzeł) staje się niedostępny do zapisu przez ten czas.
W niektórych zastosowaniach nawet minutowy przestój systemu może się wiązać ze znacznymi konsekwencjami finansowymi.

W takich przypadkach alternatywą są systemy, gdzie wiele węzłów równocześnie może obsłużyć tą samą operację zapisu, a równocześnie operacje zapisu nie są obsługiwane za każdym razem przez wszystkie węzły\footnote{Jeżeli wszystkie węzły muszą potwierdzić zapis aby się on powiódł, mamy do czynienia z opisywanym już mechanizmem 2PC (ang. \emph{2 Phase Commit}) i niedostępność jednego węzła powoduje niedostępność systemu.}.
Dzięki takiej redundancji, możliwe jest, aby awaria jednego z węzłów nie miała wpływu na dostępność systemu.

Niestety konsekwencją takiej architektury jest to, że konfliktujące zapisy (np. usunięcie rekordu i jego zmiana) mogą zostać obsłużone równocześnie przez dwa różne węzły,  a dopiero późniejsza synchronizacja doprowadza do wykrycia takiego konfliktu.
Ponieważ w ogólnym przypadku rozwiązanie takiego konfliktu jest niemożliwe, to zadanie to jest zazwyczaj w takim przypadku przekazywane klientowi bazy.

Wszystkie systemy opisane w tej pracy, które implementują replikację tego typu, nie rozróżniają węzłów na master i slave - wszystkie węzły są w stanie przyjąć operacje zapisu.
Ten wariant architektury jest czasem nazywany ,,pozbawionym węzła master'' (ang. \emph{masterless design}).
Oznacza to po prostu, że w systemie nie ma węzłów wyróżnionych.
Taka architektura systemu stanowi znaczne ułatwienie dla jego administracji - dodając węzeł do systemu wystarczy uruchomić pojedynczy proces i wskazać mu adres jednego z już działających węzłów, a system zajmie się resztą.
Dla porównania MongoDB posiada pięć rodzajów procesów, które muszą być wystartowane z odpowiednimi parametrami i na właściwych maszynach aby poprawnie skonfigurować system z replikacją i partycjonowaniem.

Systemy opisane w tej pracy, które implementują ten wariant replikacji to Riak, Cassandra i CouchDB.
Pierwsze dwa z nich są inspirowane Amazon Dynamo, podobnie jak implementujący partycjonowanie wariant CouchDB - BigCouch.
Z tej grupy jedynie Apache Cassandra nie pozwala na rozwiązywanie konfliktów przez klientów systemu, a jedynie opiera się na znacznikach czasowych.

\section{Partycjonowanie}

Część opisywanych w tej pracy systemów została stworzona z myślą o przechowywaniu ogromnych ilości danych, większych niż można pomieścić na pojedynczym węźle.
W innych przypadkach ilość danych nie musi być wcale taka duża, ale po prostu ze względów wydajnościowych korzystne jest, aby dane w całości mieściły się w pamięci RAM, więc aby system był w stanie poradzić sobie z typowymi zbiorami danych, konieczne jest aby był on w stanie dzielić te zbiory pomiędzy węzły.

Najprostszym mechanizmem podziału jest określenie liczby partycji z góry, wraz z adresami serwerów za nie odpowiedzialnych.
Przy zapisie bądź odczycie rekordu, na podstawie jego klucza przy pomocy funkcji mieszającej lub operacji modulo ustalany jest serwer który odpowiada za ten klucz.

Ten prymitywny mechanizm nie tylko nie uwzględnia replikacji, ale także utrudnia dodawanie bądź usuwanie węzłów z systemu.
Jedynym systemem spośród opisanych w tej pracy, który wykorzystuje tego typu mechanizm jest Redis, ale nie jest to funkcja tego systemu, tylko coś co jest zaimplementowane w ramach biblioteki do komunikacji z tą bazą.
W przyszłości Redis ma wspierać partycjonowanie danych w ramach Redis Cluster\footnote{patrz strona \pageref{sec:redis-cluster}.}.

Poza Redisem, jedynym systemem który nie obsługuje partycjonowania danych jest Neo4J, choć i on ma obsługiwać tą funkcjonalność w przyszłości.

Pozostałe systemy można podzielić na inspirowane algorytmem Consistent Hashing znanym z Amazon Dynamo i inspirowanie podziałem na fragmenty (tablety) znanym z Google BigTable.

\subsection*{Consistent Hashing}

Opisany wcześniej\footnote{patrz strona \pageref{sec:dynamo-consistent-hashing}.} algorytm Consistent Hashing występuje w kilku wariantach.
Jego prostą wersję implementuje Apache Cassandra, z tą tylko różnicą, że system ten przechowuje klucze w postaci posortowanej, a zatem dodanie czy usunięcie węzła nie jest tak problematyczne.

W wersji rozbudowanej, gdzie cała przestrzeń kluczy jest podzielona na określoną liczbę równej wielkości przedziałów, algorytm ten jest zaimplementowany przez Riak i BigCouch.

Zaletą tego algorytmu jest to, że rekordy są rozdzielane w miarę równomiernie między węzłami systemu.
Wadą jest to, że dane powiązane ze sobą mogą być rozrzucone po całym systemie, a zatem np. obliczenia MapReduce mogą być zmuszone do przesłania większej ilości danych niż potrzeba byłoby, gdyby przechowywane byłyby one na jednym węźle.

\subsection*{Podział na Fragmenty}

Inną odmianą partycjonowania danych, jest zaczerpnięty z Google BigTable podział na fragmenty (zwane w terminologii BigTable tabletami).
W tej wersji całość zbioru danych dzielona jest na przedziały (fragmenty).
Przedziały te są zdefiniowane przy pomocy najmniejszego i największego klucza, który się w nich znajduje i zawierają rekordy posortowane po kluczu partycjonowania.
Kiedy przedział urośnie na tyle aby przekroczyć pewną skonfigurowaną wartość w megabajtach, to dzielony jest on na połowy.
Podobnie jeżeli przedział wystarczająco zmaleje, to jest on łączony ze swoim sąsiadem.

Z opisanych w tej pracy systemów podejście takie stosują HBase i MongoDB.

Zaletą tego podejścia jest możliwość lokowania danych nawzajem powiązanych w tych samych przedziałach, co przyspiesza przetwarzanie przedziałów powiązanych danych (odczyt jednej załaduje do pamięci RAM także kilka innych sąsiadujących).
W porównaniu do Consistent Hashing, ten algorytm podziału danych ma wiele zalet.
Po pierwsze, przedziały są w przybliżeniu tej samej wielkości, więc ich równomierna dystrybucja rzeczywiście oznacza, że wszystkie węzły odpowiadają za taką samą ilość danych (przynajmniej liczoną w megabajtach).
Ponadto liczba fragmentów łatwo się dostosowuje do ilości danych przechowywanych przez system - potencjalnie system, który zaczyna działanie od kilku węzłów, może być rozbudowywany o kolejne aż osiągnie rozmiar rzędu tysięcy węzłów.
W przypadku algorytmu inspirowanego przez Amazon Dynamo górną granicą jest skonfigurowana liczba przedziałów, która z kolei w systemach takich jak BigCouch, gdy ustawiona jest na zbyt wysoką może ograniczyć wydajność systemu.

Wadą natomiast jest to, że poprzez wybranie niewłaściwego klucza partycjonowania (np. czasu utworzenia rekordu), możemy sprawić, że większość zapisów będzie trafiać do tego samego przedziału obciążając tym samym nadmiernie jeden węzeł, co z łatwością może doprowadzić do awarii tego węzła i niedostępności systemu.

\section{Wersjonowanie}

Jedną ze wspólnych cech systemów NoSQL jest brak wsparcia dla transakcji obejmujących wiele rekordów.
Od tej reguły są jednak wyjątki - największym z nich jest Neo4J, które nie tylko obsługuje transakcje ale wymaga używania ich.
Drugim z wyjątków jest Redis, który wprawdzie nie implementuje prawdziwych transakcji, ale posiada mechanizm grupowania wielu poleceń po stronie klienta, które zostają później wykonane atomowo po stronie serwera.
Pozwala to wprawdzie na grupowe wykonanie kilku operacji zapisu, ale wykonanie operacji odczytu jest niemożliwe w ramach takiej ,,transakcji'', co do pewnego stopnia ogranicza jej przydatność.

Jednym ze sposobów na ominięcie problemu braku transakcji jest zapewnienie istnienia operacji pozwalających na atomową zmianę klucza.
Redis implementuje struktury danych, które mogą być w taki sposób modyfikowane a także pozwala na zgrupowanie kilku dowolnych modyfikujących operacji w jedną.
MongoDB także posiada szereg funkcji pozwalających na dokonywanie prostych modyfikacji dokumentów, takich jak dodanie pola czy dodanie elementu do tablicy.
W przypadku kolumnowych baz danych (Cassandra, HBase), możliwe jest wykonywanie operacji na pojedynczych kolumnach, które na poziomie pojedynczego rekordu są aplikowane atomowo.

Systemy, które oferują wysoką dostępność, czyli Riak, Cassandra i CouchDB, muszą także radzić sobie z konfliktami spowodowanymi zmianami wykonanymi na różnych węzłach systemu.
Riak w tym celu posługuje się mechanizmem zegarów wektorowych, które pozwalają na śledzenie historii zmian rekordu, z dużym prawdopodobieństwem pozwalając na określenie, które różnice są konfliktami, a które oznaczają, że jeden z węzłów ma nieaktualne informacje.
Nieco inny mechanizm wykorzystuje CouchDB, które dla każdego rekordu przechowuje listę poprzednich numerów wersji, a w wypadku konfliktu w sposób deterministyczny i bez potrzeby udziału więcej niż jednego węzła wyznacza jedną z konfliktujących wersji jako aktualną, umożliwiając też klientowi systemu rozwiązanie konfliktu.
Najbardziej prymitywny system jest stosowany przez Apache Cassandra, który to system używa znaczników czasowych w celu porównania konfliktujących wersji i zawsze wybiera nowszą pozbawiając klienta możliwości rozwiązania konfliktu.
Należy jednak zwrócić uwagę, że Cassandra wersjonuje każdą kolumnę rekordu z osobna, więc stosowane w tym systemie rozwiązanie jest w wielu wypadkach zadowalające.

\section{Wyszukiwanie rekordów}

Wyszukiwanie rekordów na podstawie ich właściwości w zbiorze danych, który składa się z wielu milionów a nawet miliardów rekordów nie jest prostym zadaniem.
Większość systemów opisanych w tej pracy pozwala użytkownikowi na wyszukiwanie rekordów jedynie po ich kluczu głównym, pozostawiając programiście stworzenie indeksów drugiego poziomu i zarządzanie nimi, ale praktycznie każdy z nich posiada pewne ułatwienia związane z wyszukiwaniem.

Redis posiada dużą liczbę struktur danych, które pozwalają na grupowanie rekordów w kolekcje, a operacje na zbiorach pozwalają także na tworzenie prostych indeksów.

Riak bazuje na frameworku MapReduce w celu wyszukiwania węzłów, ale dla dużych zbiorów danych tego typu wyszukiwanie może być zbyt wolne, aby można było wykorzystywać je do wykonywania zapytań on-line.
Zbudowany na podstawie tego systemu Riak Search pozwala na wyszukiwanie rekordów używając składni zapytań biblioteki Lucene.

Kolumnowe bazy danych nie pozwalają na wyszukiwanie rekordów po dowolnej kolumnie, ale posiadają bogate API do wyszukiwania kolumn wewnątrz jednego wiersza.
W połączeni z faktem, że kolumny są przechowywane posortowane, pozwala to na tworzenie prostych indeksów drugiego poziomu.
W najnowszej wersji Apache Cassandra nie jest już nawet konieczne ręczne tworzenie i utrzymywanie tych struktur, gdyż można to zlecić systemowi.

Do budowy bardziej złożonych indeksów w kolumnowych bazach danych możliwe jest wykorzystanie Apache Hadoop, z którym oba opisane w tej pracy systemy są dobrze zintegrowane.

Dokumentowe bazy danych posiadają najbardziej rozbudowane możliwości wyszukiwania rekordów.
CouchDB posługuje się w tym celu tak zwanymi Widokami, czyli implementacją MapReduce, która zapisuje wyniki działania tych funkcji w drzewie $B-tree$, co pozwala na wyszukiwanie przy pomocy tak zbudowanych indeksów w czasie logarytmicznym względem liczby dokumentów w bazie, a także na szybkie wykonywanie operacji podsumowujących, takich jak zliczanie dokumentów spełniających pewne kryteria.
Wadą tego podejścia jest to, że wymaga ono stworzenia widoków dla każdego potrzebnego aplikacji rodzaju zapytań, a dodanie takiej funkcji, czy nawet jej zaktualizowanie wymaga dużej ilości czasu na przeliczenie indeksu.

MongoDB posiada zarówno możliwość tworzenia indeksów drugiego poziomu, wyszukiwanie po atrybutach na których nie ma założonego indeksu, a także tworzenie złożonych zapytań.
Dzięki temu system ten oferuje najwygodniejsze API wyszukiwania spośród wszystkich omawianych.
Do wad tego rozwiązania należy to, że wszystkie indeksy muszą mieścić się w pamięci operacyjnej, co ogranicza liczbę indeksów jakie można stworzyć.
Takie ograniczenie, w połączeniu z faktem, że zapytania mogą odwoływać się do atrybutów, które nie posiadają indeksu sprawia, że dla dużych zbiorów danych wyszukiwanie może być wolne.

Neo4J - jedyna opisana w tej pracy baza grafowa, posiada bardzo duże możliwości jeżeli chodzi o odwzorowywanie struktur danych używanych do tworzenia indeksów.
Dzięki temu, możliwe jest użycie implementacji $B-tree$ czy indeksu geograficznego do wyszukiwania węzłów i krawędzi na podstawie ich właściwości, a do tego wykorzystywanie takich algorytmów jak BFS, DFS czy A*, do wyszukiwania w grafie na podstawie jego struktury.
Dla wygody użytkownika system ten pozwala także na wykorzystanie Apache Lucene do indeksowania węzłów i krawędzi na podstawie ich właściwości.

\section{Wybór między konsystencją a dostępnością}

Teoria CAP mówi, że system rozproszony może równocześnie zapewniać tylko dwie z trzech gwarancji: konsystencji (ang. \emph{Consistency}), dostępności (ang. \emph{Availability}) i odporności na podział sieci (ang. \emph{Partition tolerance}), ale jak praktyka pokazuje czasem nie są spełnione nawet dwie z nich.

Odporność na podziały sieci jest cechą niezbędną dla dużych systemów rozproszonych.
W przypadku małych systemów, jej brak oznacza, że system uznaje każdy podział (brak możliwości komunikacji z jednym lub więcej węzłami) za awarię tych węzłów.
Biorąc pod uwagę, że w systemach rozproszonych na kilku zlokalizowanych w tej samej szafie węzłach ryzyko wystąpienia podziału sieci jest nikłe, to brak odporności na podziały może być akceptowalnym poświęceniem.
Oczywiście jeżeli podział sieci nastąpi i obie części systemu będą kontynuować działanie przyjmując zapytania odczytu i zapisu, nie tylko zostanie utracona gwarancja konsystencji, ale także po połączeniu rozdzielonych fragmentów sieci system ulegnie awarii i będzie wymagał interwencji administratora aby zaczął działać ponownie.

Spośród opisanych w tej pracy systemów nie ma ani jednego, który nie cechowałby się odpornością na podział sieci.

W przypadku systemu Redis brak funkcjonalności automatycznego wyboru serwera master sprawia, że w wypadku awarii tego węzła system staje się niedostępny dla zapisów.
W wypadku podziału sieci, węzeł master kontynuuje pracę, zaś węzły slave, które są od niego oddzielone serwują nieaktualne dane.
Po zlikwidowaniu podziału replikacja jest wznawiana.
Nawet w sytuacji gdy nie ma jakichkolwiek awarii dane są replikowane asynchronicznie, a zatem system nie zapewnia konsystencji.
Jak widać, Redis jest systemem odpornym na podziały sieci, ale nie zapewnia ani konsystencji ani dostępności.

Kolejny z opisywanych w tej pracy systemów - Riak, czerpie wiele technik z Amazon Dynamo.
Wykorzystanie mechanizmów takich jak Hinted Handoff sprawia, że system ten zachowuje dostępność do zapisu nawet w przypadku podziałów sieci i awarii wielu węzłów naraz.
W przypadku podziału sieci, każdy z fragmentów jest w stanie kontynuować pracę niezależnie, a po ponownym połączeniu mechanizmy zegarów wektorowych, Read Repair i zapobieganie entropii przy pomocy Merkle Tree pozwala na wykrycie konfliktów.
W sytuacji gdy sieć nie jest podzielona, system ten można skonfigurować tak aby zapewniał gwarancje konsystencji i dostępności równocześnie, także w przypadku awarii pojedynczych węzłów.
Riak należy do grupy systemów dostępnych i odpornych na podziały sieci (AP).

Apache Cassandra, tak samo jak Riak jest systemem wzorowanym na Amazon Dynamo i wykorzystuje te same techniki (z wyjątkiem zegarów wektorowych), a co za tym idzie także należy do grupy AP.

HBase pod bardzo wieloma względami przypomina Google BigTable.
W przypadku podziału sieci, jedynie ta partycja która zawiera większość węzłów Apache Zookeeper jest w stanie kontynuować działanie.
Węzły należące do wszystkich pozostałych części pozostają niedostępne zarówno dla zapisów, jak i dla odczytów aż do zlikwidowania podziału.
Każdy zapis czeka, aż dane zostaną zreplikowane na wszystkie węzły na których dane te mają się znaleźć.
W wypadku awarii pojedynczych węzłów ich odpowiedzialność jest automatycznie przekazywana, jedynie awaria węzła NameNode, odpowiedzialnego m.in. za metadane systemu plików HDFS powoduje awarię systemu.
Powyższe cechy klasyfikują HBase jako system CP: odporny na podziały i konsystentny.

CouchDB jest systemem stworzonym do działania w środowisku gdzie połączenia między węzłami mogą być tracone regularnie.
W wypadku podziału sieci każdy węzeł zachowuje dostępność do odczytu i zapisu, a po zlikwidowaniu podziału replikacja jest wznawiana i konflikty w danych są wykrywane.
W CouchDB nie ma możliwości zapewnienia konsystencji danych między węzłami, gdyż replikacja przebiega zawsze w sposób asynchroniczny.
CouchDB jest zatem systemem z grupy AP.

Implementacja CouchDB wspierająca partycjonowanie danych - BigCouch, podobnie jak CouchDB radzi sobie dobrze z podziałami sieci (chociaż gorzej niż systemy bardziej zbliżone do Amazon Dynamo, ponieważ one implementują takie techniki jaki Hinted Handoff i redukcja entropii, a BigCouch nie).
W tym systemie w przypadku braku podziałów sieci możliwe jest zapewnienie konsystencji i dostępności w obliczu awarii pojedynczych węzłów.
Jest to możliwe ponieważ BigCouch pozwala na wykorzystanie parametrów R, W i N do konfiguracji zachowania systemu.
BigCouch także należy do grupy AP.

MongoDB posiada wiele możliwości konfiguracji, ale w tym kontekście rozważana jest tylko konfiguracja oparta o Zbiory Replik połączona z automatycznym partycjonowaniem danych.
W tej konfiguracji MongoDB posługuje się mechanizmem głosowania do wyboru węzła master w każdym zbiorze replik.
Podział sieci powoduje, że część zbiorów replik znajdzie się po jednej ze stron podziału kontynuując normalne działanie, a część zostanie przedzielona.
Przedzielone zbiory zachowają dostępność do odczytu i zapisu po stronie gdzie jest więcej węzłów, oraz tylko do odczytu po stronie gdzie jest mniej węzłów\footnote{Wbrew pozorom nie oznacza to, że system ten należy do grupy AP. Podział sieci sprawia w tym przypadku, że klienci systemu są w stanie dokonywać jedynie zapisów dotyczących tej części danych, których węzły master należą do tej samej partycji sieci co dany klient.}.
W przypadku braku podziałów sieci, można skonfigurować MongoDB tak, aby zapewnić konsystencję danych, ale w wypadku awarii pojedynczych węzłów system stanie się niedostępny dla zapisów.
MongoDB posiada zatem tylko jedną cechę, czyli odporność na podział sieci.

Neo4J, które nie obsługuje partycjonowania danych, nie jest systemem stworzonym do bycia rozpraszanym na bardzo wiele węzłów.
Wykorzystanie Apache Zookeeper sprawia, że w wypadku podziału sieci część, która nie zawiera większości węzłów Zookeeper stanie się niedostępna dla zapisów\footnote{Dokumentacja Neo4J nie specyfikuje co stanie się jeżeli węzeł straci połączenie z Apache Zookeeper. Wydaje się jednak prawdopodobne, że w takiej sytuacji system zachowuje się tak samo jak HBase i Google BigTable (które także wykorzystują rozproszoną usługę blokad) w analogicznej sytuacji i przestaje być dostępny zarówno dla odczytów jak i zapisów.}, przywrócenie połączenia spowoduje replikację zmian i kontynuację działania systemu.
Ponieważ Neo4J wykorzystuje asynchroniczną replikację, możliwe jest dokonanie odczytu nieaktualnych danych, ale węzeł który dokonał zmiany zawsze może ją odczytać.
Cechy te plasują Neo4J w grupie systemów odpornych na partycje i choć nie jest to system silnie konsystentny, to gwarancje konsystencji, które on oferuje są większe niż w przypadku innych nie zapewniających konsystencji systemów.

\chapter{Zakończenie}

Celem niniejszej pracy było przybliżenie czytelnikowi najbardziej znanych, nierelacyjnych systemów bazodanowych.
Zcharakteryzowane zostało siedem systemów, z których każdy znacznie różni się od wszystkich pozostałych.
Przedstawione zostały także teorie i wzorce, które wpłynęły na architekturę tych systemów i dokonano porównania tego jak opisane systemy realizują te wzorce, jakie mają cechy wspólne i czym się różnią.

Dostępność nowych, nierelacyjnych systemów baz danych, zebranych pod wspólną nazwą ,,Systemów NoSQL'', zachęca twórców aplikacji do porzucania sprawdzonych przez bardzo wiele lat rozwiązań, którymi są relacyjne bazy danych, na rzecz nowych systemów bazodanowych, które kuszą wizją ,,skalowalności'', ukrywając równocześnie trudności i problemy jakie wiążą się z ich zastosowaniem.

Podjęcie wyboru między relacyjną bazą danych a jednym z systemów NoSQL, a szczególnie wybór jednego z wielu systemów nierelacyjnych jest wyborem bardzo trudnym i wymagającym dobrego zrozumienia tematyki rozproszonych systemów bazodanowych i wymagań biznesowych aplikacji na rzecz której ten wybór jest dokonywany.
Poniżej zamieszczone zostały przykładowe pytania jakie powinien sobie w takiej sytuacji zadać architekt systemu. 

\section*{Przykładowe kryteria wyboru}

\begin{itemize}
 
 \item Jaka będzie maksymalna ilość danych przechowywanych w systemie?

 Niektóre systemy, takie jak Riak, HBase i Apache Cassandra, umożliwiają przechowywanie ogromnych ilości danych, ale wiąże się to z utrudnieniami związanymi z ograniczeniami w wyszukiwaniu i modelowaniu danych.
 Z kolei inne systemy (Redis, Neo4J) oferują znacznie bogatszy model danych, ale kosztem ograniczonej skalowalności.

 Częstym powodem porzucania relacyjnych baz danych na rzecz systemów NoSQL jest to, że zmiana schematu dużej bazy danych może trwać wiele godzin a czasem nawet dni, co ogranicza możliwości rozwoju aplikacji korzystającej z takiej bazy.
 Jeżeli jednak dla rozpatrywanej aplikacji czas takich migracji byłby liczony tylko w minutach, warto się dobrze zastanowić, czy ryzyko związane z wyborem nietypowego rozwiązania jest warte potencjalnych zysków.

 \item Jakiego rodzaju koszty wiążą się z niedostępnością aplikacji?

 Dla niektórych aplikacji, każda minuta niedostępności może wiązać się z poważnymi konsekwencjami finansowymi.
 Do systemów takich należą na przykład systemy wspierające produkcję czy systemy płatności internetowych.

 W innych przypadkach czasowe ograniczenie dostępności systemu nie wiąże się z wielkimi kosztami, dlatego bardziej opłaca się minimalizować koszt wytworzenia aplikacji, która byłaby bardziej skomplikowana (a zatem droższa), jeżeli wykorzystywałaby system stawiający na wysoką dostępność (Riak, Cassandra, CouchDB).

 \item Czy aplikacja potrzebuje specyficznych funkcjonalności zapewnianych przez system?

 Niektóre z omawianych w tej pracy systemów oferują unikalne możliwości dla twórców aplikacji.
 Na przykład CouchDB pozwala na tworzenie aplikacji, gdzie każdy klient posiada własną kopię (części) bazy danych, co pozwala mu pracować z aplikacją nawet przy braku połączenia z centralnym serwerem.
 Redis z kolei posiada implementacje wielu struktur danych, takich jak zbiory, listy i tablice mieszające.
 Jeszcze bardziej unikalny zestaw funkcjonalności posiada Neo4J, która to baza oferuje niespotykane nigdzie indziej możliwości modelowania domeny aplikacji (graf jest bardzo naturalnym sposobem odwzorowania rzeczywistości), a także umożliwia bardzo efektywną realizację algorytmów grafowych.

 Specyficzne potrzeby aplikacji są najlepszym powodem aby wykorzystać bazę danych najbardziej przystającą do wymagań, ale ponieważ każda taka funkcjonalność została dodana pewnym kosztem, to warto się upewnić, że nie ponosi się kosztów zbędnych funkcjonalności systemu.

\end{itemize}

Kryteriów takich jak te powyżej jest bardzo wiele: 
Jak bardzo liczy się opóźnienie? 
Jaki jest stosunek zapisów do odczytów? 
Czy aplikacja będzie skalowana dynamicznie, czy węzły będą dodawane i usuwane z klastra bardzo rzadko?

Odpowiedź na nie musi być istotną częścią procesu decyzyjnego -- zbyt wiele zespołów tworzących aplikacje internetowe ślepo podążając za modą wybiera jako główne narzędzie persystencji systemy NoSQL takie jak MongoDB, będąc całkowicie pozbawionymi świadomości wad tych systemów i możliwych alternatyw.
Celem autora było przedstawienie tych alternatyw, ich wad i zalet, oraz informacji, które pozwolą czytelnikowi na łatwiejsze zrozumienie wad i zalet innych systemów, dla których zabrakło tu miejsca.

\bibliography{bibliografia}

%-------------------------------------------------------------------------------
% Załączniki
%-------------------------------------------------------------------------------

\printindex
\listoffigures
\listoftables

% na potrzeby pisania pracy
\newpage
\listoftodos

\end{document}
