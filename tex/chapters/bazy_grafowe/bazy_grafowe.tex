\chapter{Bazy Grafowe}

\section*{Streszczenie}
W tym rozdziale opisane zostało Neo4J -- najbardziej znana i dojrzała grafowa baza danych.
System ten posiada niemalże monopol na rynku grafowych baz danych na licencji Open Source, ale mimo to popularnością znacznie ustępuje większości, jeżeli nie wszystkim opisanym wcześniej systemom NoSQL.
Prawdopodobnie wiąże się to z licencją tego produktu, która sprawia, że wykorzystanie Neo4J w typowej aplikacji internetowej wiąże się albo z koniecznością kupna licencji, albo z koniecznością udostępnienia kodu aplikacji.

\section{Neo4J}
\label{sec:neo4j}

\subsection*{Wstęp}

Neo4J jest najbardziej znaną i popularną grafową bazą danych.
Jest to także jeden z najstarszych systemów NoSQL - system ten jest stosowany w środowiskach produkcyjnych od 2003 roku.
W 2007 roku baza ta została udostępniona publicznie na licencji Open Source (Affero General Public License).

Ponieważ w odróżnieniu od wielu innych systemów Neo4J jest bazą wbudowaną, licencja AGPL sprawia, że oprogramowanie z niej korzystające także musi być udostępnione na tej licencji (chyba że jest to aplikacja do użytku wewnętrznego).
Dla firm, które chciałyby wykorzystać Neo4J do budowy swoich aplikacji, ale nie chcą udostępniać ich kodu źródłowego, istnieje możliwość wykupienia licencji komercyjnej\footnote{Dostępne są trzy płatne wersje, wszystkie płatne według modelu subskrypcyjnego za każdy serwer. Pierwsza, najtańsza wersja jest za darmo.}.

\subsection*{Protokół komunikacji}

Neo4J jest wbudowaną bazą danych, czyli nie wymaga komunikacji sieciowej pomiędzy klientem i serwerem, ponieważ baza jest częścią tego samego procesu, co aplikacja z niej korzystająca.
Takie rozwiązanie jest bardzo korzystne dla grafowych baz danych, ponieważ dzięki temu możliwe jest implementowanie różnego rodzaju algorytmów trawersujących, które wczytują z bazy wierzchołek po wierzchołku, co nie byłoby praktyczne, gdyby każda taka operacja wiązała się z opóźnieniem związanym z przesyłaniem danych przez sieć.

Mankamentem tego rozwiązania jest to, że aplikacje korzystające z Neo4J muszą być uruchamiane w Wirtualnej Maszynie Java (JVM).
Ponieważ na JVM możliwe jest programowanie w wielu innych językach niż Java (m.in. Groovy, Ruby, Python, JavaScript), to przywiązanie do tej platformy nie jest bardzo poważnym problemem, ale tym niemniej jest to pewne ograniczenie.

Poza trybem wbudowanym, dostępne jest też REST API po HTTP.
Pozwala ono wprawdzie korzystać z Neo4J aplikacjom napisanym na inne platformy niż JVM, ale jest to niewątpliwie mniej efektywne rozwiązanie.

\subsection*{Replikacja}

\myfigure{chapters/bazy_grafowe/neo4j-replikacja.png}{Neo4J Replikacja}{fig:neo4j-replikacja}

Od wersji 1.2 Neo4J posiada możliwość replikacji typu master-slave\footnote{Funkcjonalność ta jest dostępna tylko w wersji Open Source i najdroższej wersji komercyjnej.}.
Funkcjonalność ta przypomina bardzo Zbiory Replik w MongoDB\footnote{patrz strona \pageref{sec:mongodb-replication}.}, z tą różnicą że Neo4J wykorzystuje Apache Zookeeper\footnote{Jest to system Open Source bardzo przypominający Google Chubby, który został opisany na stronie \pageref{sec:google-chubby}.} do wyboru serwera master, podczas gdy MongoDB wykorzystuje mechanizm głosowania pomiędzy wszystkimi dostępnymi węzłami.

Neo4J jest wbudowaną bazą danych, w związku z tym każdy serwer aplikacyjny jest także repliką bazy danych.
Fakt że baza jest replikowana pozostaje niewidoczny z punktu widzenia aplikacji.
W przypadku zapisu, węzeł slave kontaktuje się z serwerem master i czeka aż potwierdzi on zapis danych, a następnie wprowadza modyfikacje lokalnie.
Przy każdym zapisie zanim operacja zwróci przetwarzanie do klienta, dane są zapisane na przynajmniej dwóch węzłach.
W ten sposób z punktu widzenia serwera aplikacyjnego zapewniona jest konsystencja na poziomie odczytu własnych zapisów (ang. \emph{ready-your-writes consistency}).

Taka architektura pozwala na skalowanie w zastosowaniach o znacznej przewadze odczytów nad liczbą zapisów.
Warto jednak pamiętać, że konfiguracja ta wymaga nieparzystej liczby instancji ZooKeeper, czyli zalecane jest wykorzystywanie jej w konfiguracjach od trzech węzłów wzwyż.

\subsection*{Partycjonowanie}

Bardzo trudno jest partycjonować grafowe bazy danych ze względu na ich wysoki stopień powiązania.
Nawet znalezienie klastrów wysokopowiązanych między sobą wierzchołków, które nadają się do rozproszenia na różne maszyny nie gwarantuje, że nowo utworzone połączenia i wierzchołki nie zaburzą tego stanu, wymuszając tym samym migrację danych między serwerami.

W chwili obecnej Neo4J nie posiada wbudowanego mechanizmu partycjonowania danych.
Według zapowiedzi twórców tego systemu funkcjonalność taka ma się pojawić w wersji 2.0, której data wydania nie jest jeszcze znana.

\subsection*{Persystencja}

Dokumentacja Neo4J nie zagłębia się w szczegóły persystencji w tym systemie.
Na podstawie faktu, że baza ta jest transakcyjna i że przerwane przez awarię transakcje są odtwarzane na podstawie dziennika transakcji, można wnioskować że najprawdopodobniej zmiany w bazie są trwale zapisywane na dysku po zakończeniu transakcji.

\subsection*{Wersjonowanie}

W Neo4J wszystkie zmiany w grafie muszą być objęte transakcją.
Domyślnie, poziom izolacji transakcji zapewnia, że zmiany dokonane w jej obrębie nie będą widoczne dla innych równoległych transakcji aż zostanie ona poprawnie zakończona.
Oznacza to, że długotrwała operacja trawersowania grafu może obserwować zmiany w grafie, które zostały dokonane po jej rozpoczęciu.
Istnieje możliwość ustawienia także wyższego poziomu izolacji, ale jest to kosztowne.

Neo4J nie obsługuje wersjonowania węzłów ani krawędzi.

\subsection*{Wyszukiwanie}

Neo4J jako baza grafowa koncentruje się przede wszystkim na ułatwieniu operacji trawersowania grafu i realizacji algorytmów grafowych, ale aby ułatwić korzystanie z tej bazy jej twórcy dodali także pewne udogodnienia pozwalające na wyszukiwanie wierzchołków i krawędzi na podstawie ich właściwości.

\subsubsection*{Model Danych}

Neo4J służy do przechowywania skierowanych grafów, gdzie każdy wierzchołek i każda krawędź mogą być opisane zestawem właściwości.
Zarówno węzły, jak i krawędzie mają unikalne identyfikatory, które mogą być użyte do ich wczytania z bazy.
Modelowanie symetrycznych relacji odbywa się poprzez dodanie dwóch przeciwnie skierowanych krawędzi między wierzchołkami.

\subsubsection*{Traversal API}

Podstawowym mechanizmem wyszukiwania w grafie jest API Trawersowania (ang. \emph{Traversal API}).
Umożliwia ono deklaratywne określanie parametrów wyszukiwania w grafie, takich jak w jaki sposób wybierać krawędzie i węzły do przejścia, które węzły zwrócić, do jakiej głębokości przeszukiwać graf i czy przechodzić po nim używając algorytmu DFS czy BFS\footnote{\emph{Depth First Search} czy \emph{Breadth First Search}.}.

Takie trawersowanie grafu jest bardzo szybkie, bo wczytuje właściwości węzłów tylko wtedy, gdy jest to konieczne.
Ponadto, użytkownikowi zwracany jest iterator, a nie kompletna lista wyników, co zmniejsza wymagania pamięciowe takiego przeszukiwania.

\subsubsection*{Indexer}

Dla ułatwienia wyszukiwania węzłów i krawędzi, których właściwości spełniają pewne kryteria, Neo4J zostało zintegrowane z Apache Lucene.
Dzięki temu, możliwe jest nie tylko wyszukiwanie pełnotekstowe, ale także wyszukiwanie po wartościach właściwości z wykorzystaniem operatorów logicznych (NOT, AND, OR) oraz podając zakresy wartości.

\subsubsection*{Neo4J Spatial}

Struktura danych składająca się z wierzchołków, skierowanych krawędzi i właściwości nadaje się do reprezentowania bardzo wielu różnych struktur danych.
Przykładem tego jest dodatek do Neo4J pozwalający na przechowywanie w grafie indeksu geograficznego, pozwalającego między innymi na wyszukiwanie wierzchołków i krawędzi w określonej odległości od punktu o podanych współrzędnych oraz na wyszukiwanie najbliższych sąsiadów.

\subsubsection*{Standardowe języki zapytań}

Poza API specyficznym dla Neo4J, istnieją jeszcze inne możliwe sposoby pracy z tą bazą.
Jednym z nich jest zestaw narzędzi przeznaczonych dla grafowych baz danych TinkerPop.
Do tego zestawu należy między innymi Blueprints API, czyli uniwersalny interfejs pozwalający na tworzenie struktury grafu obsługujący kilka różnych baz grafowych.
Innym projektem należącym do tej grupy jest Gremlin, określany czasem jako ,,PERL dla grafowych baz danych''.
Jest to język programowania, oparty na języku Groovy, który pozwala na wykorzystywanie specjalnych wyrażeń (przypominających nieco XPath) do trawersowania grafu.

\subsubsection*{Inne ciekawe dodatki}

Dla wygody użytkownika, istnieje kilka algorytmów i struktur danych zaimplementowanych w oparciu o Neo4J.
Do algorytmów należą algorytmy wyszukiwania najkrótszej ścieżki w grafie ($A*$ i $Dijkstra$), a do struktur danych należy między innymi drzewo $B-tree$.

\subsection*{Unikalne cechy}

W odróżnieniu od innych systemów opisanych w tej pracy Neo4J implementuje interfejs pozwalający na masową zmianę struktury danych (migracje).
Jest to alternatywna dla stosowanego w innych systemach podejścia polegającego na obsłudze wielu wersji danych w kodzie aplikacji.

\subsection*{Typowe zastosowania}

Neo4J to najbardziej dojrzała i popularna grafowa baza danych.
Bardzo dużą jej zaletą jest łatwość modelowania nawet bardzo skomplikowanych relacji domenowych, szczególnie w porównaniu z relacyjnymi bazami danych, a także łatwość odwzorowania wielu struktur danych.
Dodatkową zaletą tej bazy jest prostota wyszukiwania danych przy pomocy dynamicznie tworzonych zapytań zapewniana przez integrację z Lucene.

Neo4J nadaje się bardzo dobrze do zastosowań wymagających rozwiązywania problemów grafowych, takich jak silniki rekomendacji, narzędzia do wykrywania oszustw i wzorców w sieciach, czy też na potrzeby sieci społecznościowych.
Baza ta może stanowić dobrą alternatywę dla relacyjnych baz danych w zastosowaniach gdzie domena problemu jest skomplikowana, często się zmienia, a zapytania do bazy często łączą więcej niż dwie tabele na raz.

\subsection*{Przeciwwskazania}

Brak partycjonowania w Neo4J sprawia, że choć baza ta skaluje się do miliardów węzłów i krawędzi, to jednak może to nie być wystarczająco dużo dla rozbudowanych grafów.
Podobnie też nie jest to najlepsze rozwiązanie w przypadku domen, gdzie występują luźne i nieliczne powiązania.
W przypadku już istniejących systemów, które nie są uruchamiane na platformie Java jedynym rozwiązaniem jest zastosowanie interfejsu REST udostępnianego przez Neo4J, co wiąże się ze zmniejszoną wydajnością i mniejszą przeźroczystością warstwy persystencji.

\subsection*{Dokumentacja i wsparcie}

Neo4J jest wprawdzie dojrzałą bazą danych i prawdopodobnie najpopularniejszym systemem wśród grafowych baz danych, ale tym niemniej nie należy do najlepiej udokumentowanych systemów NoSQL.
Głównym źródłem wiedzy o Neo4J jest oficjalne wiki projektu, które opisuje głównie aspekty użytkowe bazy, czyli API i możliwości konfiguracji, ale nie poświęca wiele miejsca architekturze tej bazy i jej szczegółom implementacyjnym.
W internecie można znaleźć kilka prezentacji wideo na temat Neo4J, ale ich treść jest praktycznie identyczna.
Projekt ten też wydaje się rozwijać wolniej niż inne opisywane w tej pracy.
Prawdopodobnie jest to związane z jego komercyjną naturą.

Neo4J jest rozwijane i wspierane przez szwedzką firmę Neo Technology, podobnie jak inne formy tego typu oferuje ona szkolenia, wsparcie i pomoc w migracji z innych systemów bazodanowych.

\subsection*{Pomocne odnośniki}

Poniżej zamieszczono kilka odnośników do stron WWW związanych z Neo4J:

\begin{description}
 \item [http://neo4j.org/] - Strona domowa projektu
 \item [http://components.neo4j.org/] - Dodatkowe komponenty
 \item [http://www.tinkerpop.com/] - Zestaw narzędzi TinkerPop
 \item [http://neotechnology.com/] - Strona domowa Neo Technology 
 \item [http://vimeo.com/9852755 oraz http://vimeo.com/9859000] - Prezentacja wideo na temat Neo4J (najdłuższa, najbardziej rozbudowana z dostępnych wersji)
 \item [https://github.com/andreasronge/neo4j] - Biblioteka obudowująca Neo4J w JRuby
\end{description}
