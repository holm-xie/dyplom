\chapter{Wstęp}

Ostatnie lata przyniosły znaczny wzrost popularności nierelacyjnych baz danych.
Nowe systemy odznaczające się wysoką wydajnością i skalowalnością coraz częściej zastępują lub uzupełniają relacyjne bazy danych w systemach produkcyjnych.

Zyskujący na popularności ruch NoSQL (ang. \emph{Not only SQL}) postuluje aby osoby podejmujące decyzje o architekturze systemów informatycznych nie wybierały zawsze relacyjnych baz danych jako mechanizmu persystencji, tylko świadomie dokonywały analizy wad i zalet poszczególnych rozwiązań na potrzeby ich zastosowania.
Ze względu na bardzo dużą liczbę konkurujących ze sobą produktów, bardzo trudne jest wybranie systemu, który najlepiej spełni stawiane przed nim wymagania.

Niniejsza praca ma na celu przybliżenie najpopularniejszych obecnie nierelacyjnych baz danych, ułatwienie zrozumienia stosowanych w tych systemach technik oraz sformułowanie wad, zalet i charakterystyk zastosowań opisanych rozwiązań.

\section{Struktura pracy}

Niniejsza praca składa się z czterech części:

\begin{description}
 \item[Definicja Problemu]
 W rozdziale tym opisane zostały: tradycyjne podejście do skalowania aplikacji internetowych w oparciu o relacyjną bazę danych, problemy z którymi mierzą się systemy NoSQL, oraz historia i znaczenie ruchu NoSQL.
 \item[Studium Problemu]
 W kolejnym rozdziale przedstawione zostały, na podstawie dostępnej literatury, zagadnienia, które miały największe znaczenie w rozwoju systemów NoSQL.
 Opisane w tym rozdziale zostały zarówno prawa rządzące systemami rozproszonymi (\emph{Teoria CAP}), wzorce projektowe dotyczące replikacji (\emph{BASE, Eventual Consistency}) jak i funkcjonujące systemy internetowych gigantów takich jak Amazon i Google, które doczekały się dostępnych na licencji Open Source implementacji oraz wpłynęły w mniejszym lub większym stopniu na wiele baz NoSQL.
 \item[Opis dostępnych rozwiązań]
 Na tą część pracy składa się szereg rozdziałów zaczynając od rozdziału ,,Klasyfikacja Rozwiązań'', a kończąc na rozdziale opisującym ,,Bazy Grafowe''.
 Przybliża ona cechy, wady i zalety najpopularniejszych systemów NoSQL.
 Ponadto opisuje ona podział systemów NoSQL ze względu na takie parametry jak model danych czy sposób replikacji.
 \item[Wnioski]
 Ostatnia część pracy zajmuje się porównaniem możliwych zastosowań opisanych systemów i próbuje odpowiedzieć na pytania kiedy lepiej jest zastosować nierelacyjną bazę danych, a kiedy lepiej pozostać przy relacyjnej.
\end{description}

\section{Cel pracy}

Celem niniejszej pracy jest wprowadzenie czytelnika w świat nierelacyjnych, skalowalnych baz danych.
Przedstawione zostały stosowane w tych systemach techniki oraz cechy, które wyróżniają i łączą poszczególne rozwiązania.
Opisane zostały popularne obecnie systemy oraz zalecenia dotyczące wyboru mechanizmu persystencji dla różnych typów aplikacji.