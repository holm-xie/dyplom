\chapter{Bazy Klucz-Wartość}

\section*{Streszczenie}
W tym rozdziale opisane zostały dwie bazy typu klucz-wartość: Redis i Riak.
Pierwsza z nich, ciesząca się bardzo dużą i ciągle wzrastającą popularnością wychodzi poza prosty model klucz-wartość dodając operacje na różnego typu strukturach danych (listach, zbiorach, tablicach mieszających) oraz oferuje bardzo dużą wydajność, ale - przynajmniej w obecnej wersji - nie oferuje żadnych mechanizmów automatycznego partycjonowania danych.
Druga z nich dzięki architekturze, w której żaden węzeł nie jest wyróżniony, oraz zaawansowanym mechanizmom wersjonowania rekordów (zaczerpniętym z Amazon Dynamo), cechuje się bardzo wysoką dostępnością oraz możliwością dodawania i usuwania węzłów w miarę potrzeby.

\section{Redis}
\label{sec:redis}

\subsection*{Wstęp} 

Redis jest jedną z najbardziej popularnych\footnote{Popularność baz danych jest bardzo trudna do zmierzenia. Stwierdzenia dotyczące popularności poszczególnych systemów są subiektywną oceną autora, popartą obserwacją częstotliwości z jaką systemy te są wymieniane na blogach związanych z tematyką NoSQL, takich jak \cite{mynosql}.} baz NoSQL.
Swoją popularność zawdzięcza on niezwykle bogatemu jak na bazę typu klucz-wartość API, oraz bardzo dużej wydajności\footnote{Wraz z instalacją systemu dostępne jest narzędzie redis-benchmark, które to dla komputera klasy PC pokazało wydajność ponad 140 tysięcy operacji GET i SET na sekundę.}.
Bardzo często system ten jest wykorzystywany w aplikacjach internetowych równolegle z relacyjnymi bazami danych, najczęściej jako kolejka zadań albo w zastępstwie Memcached jako cache oraz baza, w której przechowywana jest sesja użytkownika.
System ten jest dostępny za darmo na licencji open source (New BSD License).

Mimo, że jest on klasyfikowany jako baza typu klucz-wartość, Redis nie jest horyzontalnie skalowalny.
Nie jest to zazwyczaj bardzo dużym problemem, ponieważ Redis jest w większości przypadków wystarczająco wydajny aby jeden węzeł był w stanie obsłużyć wszystkie zapytania generowane przez aplikację.

\subsection*{Protokół komunikacji}

Redis używa bardzo prostego, tekstowego protokołu komunikacji.
Istnieje możliwość autentykacji, ale tylko przy pomocy hasła zapisanego w konfiguracji serwera.
Nie ma możliwości tworzenia wielu użytkowników i określania ich uprawnień.
Ponieważ hasło jest przesyłane otwartym tekstem, a połączenia nie są w żaden sposób szyfrowane, Redis nie należy do najbezpieczniejszych rozwiązań na rynku.

Dzięki prostemu protokołowi, biblioteki do komunikacji z tym systemem są dostępne praktycznie dla każdego języka programowania.
Dla niektórych języków programowania dostępne są także biblioteki wyższego poziomu, które pełnią rolę bardzo podobną do bibliotek ORM (Object-Relational Mapping).
Biblioteki takie jak Ohm dla języka Ruby pozwalają na mapowanie obiektów na strukturę kluczy w Redis, włącznie z indeksowaniem wartości niektórych pól, co umożliwia wyszukiwanie rekordów po czym innym niż wartość klucza. 

\subsection*{Replikacja}

Redis posiada mechanizm replikacji w trybie master-slave.
Replikacja jest asynchroniczna, co oznacza, że dane odczytane z serwerów slave mogą być czasem nieaktualne.
Często Redis jest konfigurowany w taki sposób, aby serwer master w ogóle nie dokonywał zapisów na dysk twardy, tylko pozostawiał tą rolę serwerowi lub serwerom slave.
Rozwiązanie takie pozwala na zwiększenie wydajności.

\subsection*{Partycjonowanie}

Redis nie obsługuje partycjonowania w wersji 2.0.
Niektóre biblioteki pozwalają wprawdzie na implementację partycjonowania po stronie klienta, ale są to bardzo prymitywne rozwiązania, które nie mogą w żaden sposób konkurować z partycjonowaniem po stronie serwera, jakie jest zaimplementowane np. w omawianym w kolejnym rozdziale systemie Riak.

W przypadku systemów typu klucz-wartość samodzielna implementacja partycjonowania i replikacji kluczy jest zazwyczaj najprostsza z pośród wszystkich typów nierelacyjnych baz danych.
Redis jednak obsługuje złożone typy danych, których wielkość dla pojedynczego klucza może potencjalnie przekroczyć wielkość dostępnej pamięci RAM, a których nie da się w żaden prosty sposób podzielić na wiele węzłów.
Dlatego dopóki nie zostanie zaimplementowany \emph{Redis Cluster}, systemowi temu będzie daleko do konkurencji pod względem skalowalności.

\subsubsection*{Redis Cluster}
\label{sec:redis-cluster}

W planach jest implementacja tak zwanego \emph{Redis Cluster}, który pozwoli na horyzontalne skalowanie systemu poprzez replikację i partycjonowanie danych.
Redis Cluster będzie systemem typu CA (Consistent-Available)\footnote{Dokładniejszy opis tego rozwiązania można znaleźć w \cite{antirez-redis-cluster}.}.
W systemie będą występować cztery rodzaje węzłów:

\begin{description}
 \item[Data Node] (węzeł danych) - węzły przechowujące dane.
 Skalowanie systemu odbywa się poprzez zwiększanie liczby tych węzłów.

 \item[Configuration Node] (węzeł konfiguracyjny) - węzeł przechowujący metadane o całym systemie, takie jak listy \emph{Data Node} i \emph{Proxy Node}.
 W razie awarii tego węzła, system wprawdzie nie zaprzestaje normalnej pracy, ale nie jest w stanie obsłużyć sytuacji wyjątkowych takich jak awaria innego węzła.
 Ten węzeł tylko przechowuje dane konfiguracyjne, węzłem odpowiedzialnym za ich obsługę jest \emph{Handling Node}.

 \item[Proxy Node] (węzeł proxy) - węzły odpowiedzialne za koordynację zapytań w systemie.
 Zapytania w systemie są zawsze kierowane do węzłów Proxy, które przekazują je następnie do jednego (w przypadku odczytów) lub wszystkich (w przypadku operacji zmieniających dane) węzłów, na których odpowiednie rekordy są zapisane.
 Awarie są wykrywane przez \emph{Proxy Node} i informacja o nich jest zapisywana w węźle konfiguracyjnym.

 \item[Handling Node] (węzeł zarządzający) - to klient, który obsługuje dane zapisane w węźle konfiguracyjnym.
 Zajmuje się on zmianą przydziału zakresów kluczy do węzłów w razie dodania lub usunięcia węzła, lub jego awarii.
 Najczęściej \emph{Handling Node} i \emph{Configuration Node} powinny być umieszczone na tym samym fizycznym węźle.
\end{description}

\subsection*{Persystencja}

\subsubsection*{Snapshotting}

Redis zawdzięcza swoją szybkość temu, że zarówno operacje zapisu, jak i operacje odczytu nie muszą operować na danych zapisanych na dysku twardym, tylko na danych w pamięci RAM.
Domyślnie Redis pracuje w trybie zapisywania zrzutów aktualnego stanu (ang. \emph{snapshotting}) co pewien czas w zależności od liczby wykonanych operacji.
Typową konfiguracją w tej opcji jest zapis danych na dysk co 60 sekund jeżeli zostało dokonanych przynajmniej 100 zmian, albo do 1000 sekund, jeżeli została wykonana co najmniej jedna zmiana.

\subsubsection*{Append Only File}

Alternatywą dla periodycznego zrzucania stanu bazy na dysk jest \emph{Append Only File} (plik tylko do dopisywania).
Technika ta polega na dopisywaniu zmian w bazie na koniec pliku, przypominającego dziennik transakcji.
Przy starcie systemu plik ten jest odczytywany i zmiany w nim zawarte są aplikowane po kolei aby przywrócić stan bazy.
Oczywiście plik ten rośnie z czasem, co powoduje także wydłużenie czasu startu systemu, dlatego dostępna jest komenda służąca do przepisania pliku z postaci dziennika do postaci zrzutu bazy - pozwala to zaoszczędzić miejsce na dysku i znacznie przyspieszyć start systemu.

W przypadku \emph{Append Only File} domyślnie dane są zapisywane na dysk co sekundę przy pomocy wywołania systemowego \verb+fsync()+, które gwarantuje że dane zostaną zapisane na dysku.
Alternatywnie możliwe jest wywołanie \verb+fsync()+ po każdej modyfikującej operacji, ale obija się to tak bardzo negatywnie na wydajności, że nie jest to zalecana konfiguracja.
Można także skonfigurować Redis tak, aby nigdy sam nie dokonywał operacji \verb+fsync()+ i powierzył odpowiedzialność za regularne zapisywanie danych na dysk systemowi operacyjnemu, co zwiększa wydajność kosztem bezpieczeństwa danych.

\subsection*{Wersjonowanie}

Redis nie posiada mechanizmu wersjonowania rekordów - dla klucza jest przechowywana tylko jego aktualna wartość i nie ma identyfikatorów wersji.
System ten jednak obsługuje transakcje (operacja \verb+MULTI+) oraz komendę \verb+WATCH+, które pozwalają na uniknięcie przypadkowego nadpisania rekordu.

\subsubsection*{Zmiana rekordu bez użycia MULTI i WATCH}

\begin{enumerate}
 \item Klienci A oraz B odczytują wartość klucza (\verb+GET nazwa_klucza+)
 \item Klienci A oraz B przetwarzają otrzymaną wartość klucza i zmienia ją
 \item Klient A zapisuje nową wartość dla klucza (\verb+SET nazwa_klucza nowa_wartosc1+)
 \item Klient B zapisuje nową wartość dla klucza (\verb+SET nazwa_klucza nowa_wartosc2+) nadpisując tym samym zmiany dokonane przez klienta A.
\end{enumerate}

\subsubsection*{Zmiana rekordu z użyciem transakcji}

\begin{enumerate}
 \item Klienci A oraz B oznaczają klucz jako obserwowany (\verb+WATCH nazwa_klucza+)
 \item Klienci A oraz B odczytują wartość klucza (\verb+GET nazwa_klucza+)
 \item Klienci A oraz B przetwarzają otrzymaną wartość klucza i zmienia ją
 \item Klienci A oraz B rozpoczynają transakcję w celu zapisania nowej wartości (\verb+MULTI+)
 \item Klient A zapisuje nową wartość dla klucza (\verb+SET nazwa_klucza nowa_wartosc1+)
 \item Klient B zapisuje nową wartość dla klucza (\verb+SET nazwa_klucza nowa_wartosc2+)
 \item Klient A kończy transakcję (\verb+EXEC+) - transakcja kończy się powodzeniem
 \item Klient B kończy transakcję (\verb+EXEC+) - transakcja kończy się niepowodzeniem, ponieważ klient A zmodyfikował obserwowany klucz.
 Zmiana dokonana przez klienta B jest wycofana.
 \item Klient B musi wykonać całą operację od początku, odczytując tym samym wartość zapisaną przez klienta A.
\end{enumerate}

\subsection*{Wyszukiwanie}

Redis implementuje operacje na kilku różnych strukturach danych.
Wyróżnia go to spośród innych systemów typu klucz-wartość, sprawiając że jest on tym samym o wiele łatwiejszy w użyciu i wymaga mniej pracy od programisty.

Podobnie jak inne systemy typu klucz-wartość Redis nie udostępnia możliwości indeksowania zapisanych wartości, ale samodzielne tworzenie takich indeksów jest nieco ułatwione dzięki istnieniu takich struktur danych jak zbiór czy lista.

\subsubsection*{Obsługiwane struktury danych}

\begin{description}
 \item[Ciąg znaków] - jest to podstawowa struktura danych.
 Dostępne są między innymi operacje odczytujące i zapisujące wartość dla klucza, oraz (nietypowo) operacje pozwalające na dopisanie ciągu znaków na koniec klucza oraz na odczytanie jedynie jego fragmentu.
 Możliwe jest także określenie czasu życia klucza, po którym może on zostać przez system usunięty.

 \item[Lista] - pozwala na zapisanie sekwencji wartości.
 Możliwe jest atomowe dodanie wartości na początek bądź koniec listy oraz atomowe pobranie wartości z początku lub końca.
 Dostępna jest także blokująca wersja pobrania wartości.
 Listy są bardzo przydatne do implementacji kolejek zadań.
 Stosuje się je także do łączenia wielu rekordów w kolekcje (zapisując ich klucze na liście), ale warto pamiętać, że stronicowanie takiego zbioru ma złożoność liniową do wielkości listy, więc powinno się go raczej unikać.

 \item[Zbiór] - umożliwia wykonywania operacji na zbiorach (sprawdzenie członkostwa elementu, suma, przecięcie, różnica).
 Możliwe jest pobranie jednego, losowego elementu, albo wszystkich elementów zbioru.

 \item[Posortowany Zbiór] - umożliwia zapisanie posortowanej sekwencji wartości.
 Możliwe jest wykonanie przecięcia i sumy, ale nie różnicy posortowanych zbiorów.
 Możliwe jest także, podobnie jak dla list, stronicowanie zbioru.

 \item[Tablica mieszająca] - często rekordy w Redisie są zapisywane w taki sposób, że kluczowi identyfikującemu rekord przypisujemy rekord zserializowany, np. do formatu JSON.
 Jest to akceptowalne, jeżeli zawsze chcemy odczytywać rekordy w całości, ale jeżeli rekord jest duży, a potrzebne jest nam tylko kilka jego pól, to lepiej każde z nich zapisać pod osobnym kluczem.
 Tablica mieszająca jest strukturą, która optymalizuje właśnie to zastosowanie.
\end{description}

\subsection*{Unikalne cechy}

Redis posiada wiele funkcjonalności, które wyróżniają go na tle innych systemów typu klucz-wartość.
Wśród nich poczesne miejsce zajmuje na pewno opisana wcześniej obsługa różnych typów danych, ale można do nich zaliczyć także mechanizm obsługi zdarzeń (Publish-Subscribe) oraz implementacja pamięci wirtualnej wewnątrz systemu, bez polegania na funkcjonalności systemu operacyjnego.

\subsubsection*{Publish-Subscribe}

W ostatnich latach aplikacje internetowe coraz częściej uaktualniają zawartość strony bez potrzeby odświeżenia jej przez użytkownika.
Często potrzebne jest aby takie aktualizacje były dostarczane do klienta w czasie niemal rzeczywistym (na przykład Google Wave pokazuje użytkownikowi tekst wpisywany przez innych użytkowników litera po literze).
W takich przypadkach mechanizm obsługi zdarzeń ma największe zastosowanie.
Mechanizm ten, nazywany również wzorcem Publish-Subscribe, polega na tym, że klienci mogą zapisywać się na otrzymywanie zdarzeń publikowanych w określonych kanałach.
Dzięki istnieniu takich kanałów, zapisani na zdarzenia nie muszą wiedzieć o publikujących i vice versa.
Natomiast dzięki temu, że zapisani na zdarzenia są o ich zajściu informowani bezpośrednio po fakcie, nie jest konieczne regularne odpytywanie serwera, co zmniejsza jego obciążenie, oraz skraca okres czasu pomiędzy zajściem zdarzenia a pokazaniem jego efektów użytkownikowi.

\subsubsection*{Pamięć Wirtualna}

Redis jest przeznaczony do pracy ze zbiorami danych, które w całości mieszczą się w pamięci operacyjnej serwera.
Ponieważ jednak w rzeczywistości często występują zbiory danych, których nie da się zmieścić w pamięci RAM, ale których niektóre fragmenty są odczytywane i zapisywane dużo częściej niż inne, Redis implementuje własną wersję pamięci wirtualnej.

Bardzo często pada pytanie dlaczego Redis implementuje coś, co jest już zaimplementowane w systemie operacyjnym.
Szeroko odpowiada na to artykuł na blogu autora tej bazy \cite{antirez-redis-vm}.
Jedną z zalet własnej implementacji jest to, że pozwala to na stronicowanie na poziomie rekordów, podczas gdy system operacyjny operuje na poziomie stron pamięci, które mogą zawierać zarówno często, jak i rzadko używane rekordy.
Inną zaletą jest to, że rekord zapisany na dysku nie musi mieć identycznej postaci jak w pamięci.
Poprzez nie zapisywanie metadanych i różnych wskaźników, zapisywane rekordy mogą zajmować na dysku 10 razy mniej miejsca niż w pamięci operacyjnej.
Zaletą nie wymienioną przez autora jest to, że kiedy korzystamy z wirtualnej pamięci, ustalony zostaje maksymalny rozmiar pamięci zajmowanej przez system\footnote{Opcja ta może być włączona także bez włączania pamięci wirtualnej, ale włączenie jej w takiej sytuacji może prowadzić do zrywania połączeń z klientami a nawet utraty danych.}.
Dzięki temu Redis może koegzystować na jednej maszynie z innymi aplikacjami takimi jak serwer aplikacyjny i nie konkurować z nimi o RAM.

\subsection*{Typowe zastosowania}

Dzięki temu, że Redis posiada obsługę list z atomową i blokującą operacją pobrania pierwszego lub ostatniego elementu listy, system ten jest powszechnie wykorzystywany do przechowywania kolejek zadań.
Wykorzystując posortowane zbiory i mechanizm opisany w sekcji ,,Wersjonowanie'', możliwa jest także implementacja kolejek priorytetowych.

Ze względu na bardzo dużą wydajność oraz możliwość określania czasu życia kluczy Redis jest także wykorzystywany często jako cache dla aplikacji internetowych.

Redis nie jest często wykorzystywany jako podstawowa baza danych, ale zdobył on popularność wśród twórców gier internetowych, gdzie jego stosunkowo prosty model danych nie jest przeszkodą, a wydajność jest bardzo dużą zaletą.

\subsection*{Przeciwwskazania}

Redis został stworzony z myślą o zbiorach danych, które można pomieścić w pamięci operacyjnej serwera, dlatego (mimo istnienia pamięci wirtualnej) nie jest zalecane stosowanie tego systemu w połączeniu z bardzo dużymi zbiorami danych.
Jedną z podstawowych zalet tego systemu jest duża wydajność, którą traci on całkowicie jeżeli zastosowany zostanie Append Only File z opcją zapisu na dysk po każdej transakcji.
W związku z tym, jeżeli ryzyko utraty danych z ostatniej sekundy przed awarią jest nie do przyjęcia, Redis nie jest najlepszym rozwiązaniem w takim przypadku.

\subsection*{Dokumentacja i wsparcie}

Dokumentacja Redisa jest na bardzo wysokim poziomie.
Każda komenda jest dobrze opisana (włączając w to opis złożoności obliczeniowej), a ponadto rozwiązania takie jak transakcje, Publish-Subscribe czy pamięć wirtualna są opisane w osobnych artykułach, najczęściej z przykładami.
W internecie można znaleźć wiele przykładów zastosowania Redisa i bibliotek na licencji open source, które z niego korzystają, co ułatwia lepsze poznanie tego systemu.

Redis został stworzony i jest rozwijany przez Salvatore Sanfilippo, a od niedawna jego rozwój jest sponsorowany przez firmę VMWare.

\subsection*{Pomocne odnośniki}
 
Poniżej zamieszczono kilka odnośników do stron WWW związanych z Redisem.

\begin{description}
 \item [http://code.google.com/p/redis/] - strona domowa projektu
 \item [http://code.google.com/p/redis/wiki/CommandReference] - spis wszystkich komend
 \item [http://retwis.antirez.com/] - klon Twittera korzystający z Redisa
 \item [http://ohm.keyvalue.org/] - strona domowa biblioteki Ohm (odpowiednik ORM dla Redisa)
 \item [http://www.rediscookbook.org/] - Redis Cookbook, opis rozwiązań niektórych typowych problemów
 \item [https://github.com/defunkt/resque] - popularna implementacja kolejki zadań stworzona na potrzeby serwisu GitHub
\end{description}

\section{Riak}
\label{sec:riak}

\subsection*{Wstęp} 

Riak jest napisaną w Erlangu, dopiero zdobywającą popularność bazą typu klucz-wartość, która mocno bazuje na opisanych w artykule o Amazon Dynamo technikach (patrz strona \pageref{sec:dynamo-techniki}).
System ten jest dostępny w dwóch wersjach: darmowej na licencji open source (Apache License 2.0) oraz rozszerzonej, płatnej wersji o nazwie Riak Enterprise DS.

Niniejszy rozdział nie zagłębia się w opisane już techniki, tylko zamiast tego opisuje szczegóły, które nie występują w Dynamo, a zostały zaimplementowane na potrzeby Riaka.

Informacje w tym rozdziale zostały zaczerpnięte głównie z wiki projektu, które stanowi bardzo dobre źródło informacji na jego temat \cite{riak-wiki}.

\subsection*{Protokół komunikacji}

Riak obsługuje trzy protokoły komunikacji - natywny w Erlangu\footnote{Erlang jest językiem, który bardzo dobrze obsługuje komunikację międzyprocesową w środowisku rozproszonym (praktycznie bez różnicy czy procesy są częścią tego samego procesu systemu operacyjnego, czy znajdują się na różnych maszynach) - ten klient wykorzystuje właśnie ten protokół komunikacji.}, interfejs REST po HTTP (wewnętrznie używający klienta w Erlangu) oraz interfejs w technologii Protocol Buffers\footnote{Protocol Buffers jest technologią serializacji i definicji interfejsu RPC opublikowaną przez Google w 2008 roku i używaną wewnętrznie w wielu projektach tej firmy. Więcej szczegółów jest dostępnych na stronie domowej projektu http://code.google.com/p/protobuf/}.
Większość bibliotek wykorzystuje PBC (Protocol Buffers Client) do komunikacji z Riakiem, ze względu na przewagę jaką ten protokół ma pod względem wydajności nad HTTP.

Riak nie obsługuje żadnych mechanizmów autentykacji ani autoryzacji użytkowników.
W opinii autorów systemu funkcjonalność taka nie jest potrzebna, gdyż z założenia Riak jest przeznaczony do pracy w sieci lokalnej.
W przypadku interfejsu HTTP łatwo można użyć tak zwanego reverse-proxy, aby dodać tą funkcjonalność (polega to na umieszczeniu przed bazą dodatkowych serwerów z którymi komunikują się klienci i które przekazują odpowiednio zautentykowane połączenia dalej).

\subsection*{Replikacja}

W Riaku każdy rekord jest identyfikowany przez parę (kubełek\footnote{ang. \emph{bucket}}, klucz).
Konfiguracji replikacji dokonuje się na poziomie kubełka określając (tak jak w Amazon Dynamo) liczbę węzłów N, na które klucz ma być replikowany.
Węzły te są wybierane zgodnie z algorytmem Consistent Hashing opisanym we wcześniejszym rozdziale (strona \pageref{sec:dynamo-consistent-hashing}).
W przypadku awarii wykorzystywany jest opisany na stronie \pageref{sec:dynamo-hinted-handoff} mechanizm Hinted Handoff.

Przy odczytach i zapisach stosowane są opisane już parametry R i W, które oznaczają odpowiednio ile węzłów (z N przechowujących wartość) musi odpowiedzieć zanim operacja zwróci rezultat.
W przypadku odczytów jest zastosowany mechanizm Read Repair, który polega na tym, że jeżeli któryś z węzłów zwróci nieaktualną wartość (ale nie wartość będącą w konflikcie z inną zwróconą - patrz zegary wektorowe strona \pageref{sec:dynamo-vector-clocks}), to wartość na tym węźle zostanie automatycznie uaktualniona, a klientowi zostanie zwrócona tylko jedna, najnowsza wersja. 

\subsection*{Partycjonowanie}

Partycjonowanie w Riaku wykorzystuje dokładnie ten sam mechanizm co Amazon Dynamo - Consistent Hashing.
Dzięki temu łatwo można do klastra dodawać oraz usuwać z niego węzły.

\subsection*{Persystencja}

Riak obsługuje możliwość wymiany mechanizmu zapisu danych (ang. \emph{Pluggable Backends}).
Polega to na tym, że w konfiguracji możemy zapisać nazwę i parametry konfiguracyjne \emph{backendu}, które obowiązują dla wszystkich kluczy przechowywanych przez ten węzeł.
Przy wykorzystaniu specjalnego \emph{backendu} o nazwie \verb+riak_kv_multi_backend+, możliwe jest ustalenie osobnego \emph{backendu} dla każdego kubełka.

Dostępne \emph{backendy} dzielimy na dwie podstawowe kategorie: zapisujące dane na dysk (trwałe) i zapisujące dane w pamięci operacyjnej (nietrwałe).
Przy zapisie oprócz podania parametru W oznaczającego ile węzłów musi potwierdzić zapis, możliwe jest także podanie parametru DW\footnote{\emph{Durable Writes}, z ang. \emph{durable} = trwały}, które określa ile węzłów musi dokonać zapisu do trwałego \emph{backendu} aby operacja zwróciła wynik.

Riak oferuje kilka wbudowanych \emph{backendów}, ale dostępne są też inne, niestandardowe.
Wbudowane backendy to: DETS (domyślny), Bitcask (zapis do plików typu \emph{append-only}), ETS (w pamięci), gb\_trees (drzewo w pamięci), fs (każdy klucz w osobnym pliku na dysku), cache (dostępnych kila wariantów) oraz multi\_backend, który pozwala na konfigurację backendu dla kubełków. 

Niestety nie ma sposobu na zapewnienie replikacji klucza na określoną liczbę węzłów zapisujących na dysk albo w pamięci (z artykułu o Amazon Dynamo można wywnioskować, że mechanizm taki występuje tam, podobnie jak zapewnienie, że jedna z replik znajduje się w innym centrum obliczeniowym).

\subsection*{Wersjonowanie}

Riak wykorzystuje zegary wektorowe - ten sam mechanizm wersjonowania, który został już opisany w rozdziale na temat Amazon Dynamo (patrz strona \pageref{sec:dynamo-vector-clocks}).
Do dyspozycji użytkownika pozostają dwie opcje konfiguracyjne, obie ustawiane jako właściwości dla każdego kubełka z osobna: \verb+allow_mult+ oraz \verb+last_write_wins+.
Pierwszy z nich kontroluje możliwość tworzenia wielu wersji rekordu (włączany jest jeżeli rekord może być modyfikowany przez wielu użytkowników systemu), drugi zaś upraszcza mechanizm rozwiązywania konfliktów - w przypadku wystąpienia kilku różnych wersji, użytkownikowi zaprezentowana zostanie najnowsza wersja.

\subsection*{Wyszukiwanie}

W odróżnieniu od Redisa, Riak nie obsługuje różnorodnych struktur danych.
Poza podstawowym dla baz typu klucz-wartość interfejsem dostępu do danych (zapisz rekord o danym kluczu, odczytaj rekord o danym kluczu) Riak obsługuje kilka mechanizmów pozwalających na stosunkowo łatwe wyszukiwanie rekordów: MapReduce, Link Walking oraz aplikację Riak Search.

\subsubsection*{MapReduce}

Map-Reduce w Riaku istotnie się różni od Google MapReduce opisanego w rozdziale \ref{sec:google-map-reduce}.
Powodem istnienia tych różnic jest przede wszystkim to, że MapReduce w Riaku jest stosowane przede wszystkim jako mechanizm wyszukiwania i wykonywania różnego rodzaju operacji podsumowujących, podczas gdy framework Google jest stosowany przede wszystkim do przetwarzania ogromnych ilości danych.

Największą różnicą między tymi dwoma implementacjami jest to, że w Riaku operacja Reduce jest wykonywana na węźle, który zainicjował wykonanie operacji MapReduce, podczas gdy framework Google umożliwia określenie liczby węzłów, które mają wykonać operację Reduce, dzięki czemu wynikiem tej operacji może być zbiór danych większy niż można zmieścić na pojedynczym węźle.

W przypadku MapReduce traktowanego jako mechanizm wyszukiwania w bazie nie jest to wielkim problemem: typowe zapytanie wykorzysta jedną lub więcej faz Map aby wyszukać pożądane rekordy, a następnie w fazie Reduce posortuje je (albo po prostu utworzy z nich listę) i zwróci wynik do klienta.
W przypadku takich zapytań liczba zwróconych rekordów zazwyczaj jest wystarczająco mała aby zmieścić się w pamięci operacyjnej węzła zlecającego operację.
Dla porównania, sztandarowy przykład aplikacji MapReduce - zliczanie liczby wystąpień słów, ze względu na wymagania pamięciowe jest niepraktyczny do wykonania w Riaku.
W tym przypadku dla każdego rekordu w systemie zostałaby wygenerowana bardzo duża liczba kluczy pośrednich, które z kolei musiałyby zostać przesłane przez sieć do węzła zlecającego operację.
Sam narzut komunikacyjny jest w takim przypadku bardzo duży, ale dodatkowo wynikiem tej operacji może być zbiór danych większy niż dostępna pamięć operacyjna czy nawet przestrzeń dyskowa pojedynczego węzła.
W planach jest zniesienie tego ograniczenia, ale w momencie pisania tych słów nie wiadomo jeszcze kiedy to nastąpi\footnote{http://blog.basho.com/2010/07/27/webinar-recap---mapreduce-querying-in-riak/}.

Kolejną różnicą jest to, że w Riaku operacja MapReduce składa się z dowolnej, większej od zera liczby faz.
Każda faza może być jedną z: Map, Link oraz Reduce.
Fazy mogą występować w dowolnej kolejności, a każdy kolejny krok otrzymuje wynik poprzedniego.
Funkcje Map i Reduce mogą być specyfikowane w dwóch językach programowania: Erlang i JavaScript i mogą albo być podane jako argument zapytania, albo być zapisane na serwerze.
Faza Link jest specjalnym rodzajem operacji Reduce, która umożliwia trawersowanie grafu rekordów przy pomocy linków (patrz niżej).

\subsubsection*{Link Walking}

W Riaku rekordy mogą być powiązane między sobą krawędziami (zwanymi linkami) tworząc w ten sposób graf skierowany.
Każda krawędź jest opisana etykietą, dzięki czemu różne krawędzie mogą mieć różne ,,znaczenie''.

W odróżnieniu od baz grafowych, krawędzie nie mają właściwości, oraz nie ma możliwości szybkiego (w czasie O(1)) znalezienia rekordów, których krawędzie wskazują na określony rekord - graf można trawersować tylko i wyłącznie zgodnie z kierunkiem wyznaczonym przez krawędź.
Kolejną różnicą jest brak API, które umożliwiłoby wyszukiwanie w grafie przy pomocy algorytmu BFS albo DFS.

Wyszukiwanie powiązanych przy pomocy krawędzi rekordów jest możliwe albo jako faza Link w MapReduce albo na poziomie API (co w praktyce jest równoważne operacji MapReduce składającej się tylko z jednej fazy Link).

\subsubsection*{Riak Search}

Riak Search to silnik wyszukiwania pełnotekstowego, który bazuje na Riaku. 
Pozwala on na wyszukiwanie rekordów używając składni zapytań zapożyczonej z Apache Lucene oraz przy pomocy interfejsu REST który jest kompatybilny z Apache Solr.
Jest to nowy produkt, dostępny na licencji open source Apache 2.0, wciąż jeszcze w wersji beta, ale jest już używany w systemach produkcyjnych.

Ponieważ Riak KV\footnote{Riak KV to pełna nazwa opisywanej w tym rozdziale bazy.} jest częścią Riak Search, nic nie przeszkadza w wykorzystywaniu tych samych węzłów do przechowywania indeksu silnika wyszukiwania i danych.
Riak Search posiada wbudowaną obsługę formatu JSON (który jest najczęściej wykorzystywanym formatem serializacji danych w Riaku), dzięki czemu indeksowanie rekordów przechowywanych w bazie jest bardzo proste - sprowadza się do określenia tzw. \emph{hooka} (odpowiednik znanych z relacyjnych baz danych triggerów) dla kubełka, który zleci indeksowanie rekordu przy jego zapisie.

Język zapytań wykorzystywany w tej aplikacji jest wystarczająco rozbudowany aby zastąpić proste zapytania MapReduce.
Możliwe jest wyszukiwanie rekordów, które mają pole o określonej wartości (lub w przedziale wartości), warunki mogą być łączone przy pomocy operatorów logicznych, możliwe jest nawet wyszukiwanie wykorzystujące w warunku metrykę Levenshteina.
Ponieważ zapytania Riak Search wykorzystują indeks, wymagają one znaczniej mniej czasu aby zwrócić wynik niż zapytania wykorzystujące MapReduce.

\subsection*{Unikalne cechy}

Większość baz dostępnych aktualnie na rynku należy albo do kategorii CP (Consistent-Partition Tolerant) albo CA (Consistent-Available).
Riak jest jednym z bardzo niewielu produktów, który przynależy do kategorii AP (Available-Partition Tolerant).
Bardzo interesującą właściwością tego systemu (podobnie jak Amazon Dynamo) jest to, że jest on w stanie przyjmować zapisy praktycznie zawsze, o ile choć jeden węzeł nie uległ awarii.
Niestety, jedynie komercyjna wersja Riaka posiada możliwość replikacji między centrami danych, a ponieważ awarie całych centrów obliczeniowych nie są niespotykane, jeżeli komuś zależy na wysokiej dostępności, to musi on za nią zapłacić\footnote{Basho oferuje wersję komercyjną Riaka po promocyjnych cenach dla start-upów (według modelu \emph{,,pay what you can''}), stąd nie jest to koniecznie dużą przeszkodą dla nowych firm.}.

\subsection*{Typowe zastosowania}

W ostatnich latach termin Cloud Computing (w wolnym tłumaczeniu - obliczenia w chmurze) zyskał znacznie na popularności.
Cloud Computing polega na tym, że firmy zamiast tworzyć i zarządzać własnymi centrami obliczeniowymi, wynajmują wirtualne maszyny od wyspecjalizowanych dostawców, dzięki czemu firmy te nie muszą się przejmować takimi problemami jak awarie sprzętu, czy jego administracją.
Dużą zaletą tego modelu jest także to, że płaci się w tych rozwiązaniach z dokładnością do godziny, dzięki czemu można wynajmować więcej maszyn w godzinach szczytu (lub w sezonie większego obciążenia), a mniej na przykład w nocy.

Niestety Cloud Computing ma też wady.
Na Amazon Elastic Computing Cloud (Amazon EC2 - najpopularniejszy dostawca Cloud Computing) należy liczyć się z tym, że wynajęte maszyny mogą być regularnie wyłączane i włączane, co w przypadku takich baz jak Redis czy MongoDB może grozić regularną utratą danych.

Dwie cechy Riaka czynią go idealnym kandydatem do wykorzystania ,,w chmurze'': nie posiada on wyróżnionych węzłów, więc restart dowolnego z nich nigdy nie spowoduje chwilowej niedostępności całego systemu, a ponadto posiada bardzo dobrą obsługę dodawania oraz usuwania węzłów, dzięki czemu można go łatwo skalować w górę lub w dół w zależności od obciążenia, czy pory dnia.

W przypadku systemów mniejszej skali, Riak może być bardzo przydatny w przypadku aplikacji, gdzie niedostępność systemu nawet przez kilka-kilkadziesiąt sekund może spowodować znaczne straty.

\subsection*{Przeciwwskazania}

Riak posiada wiele zalet, ale w praktyce wszystkie z nich nabierają znaczenia wraz ze wzrostem liczby węzłów systemu.
Jeżeli aplikacja nie wymaga przechowywania bardzo dużych ilości danych, albo prawie stuprocentowej dostępności (a zdecydowana większość aplikacji ich nie wymaga), to zastosowanie takiej bazy jak Riak jest raczej niepotrzebne.

\subsection*{Dokumentacja i wsparcie}

Riak posiada dość dobrą oficjalną dokumentację w postaci wiki, dostępne są też w internecie nagrania wideo z prezentacji na różnych konferencjach na jego temat.
Wiele ciekawych informacji można także znaleźć na blogu firmy Basho, gdzie opisywane są nie tylko niektóre techniki zastosowane w Riaku, ale także pewne szczegóły implementacyjne.
System ten zdobywa stopniowo popularność, dzięki czemu można znaleźć coraz więcej informacji na jego temat na różnych blogach oraz forach dyskusyjnych.

Riak jest tworzony przez firmę Basho, która oferuje także komercyjną wersję tej bazy (w cenie tej oferty są także szkolenia, wsparcie oraz porady dotyczące architektury i implementacji).

\subsection*{Pomocne odnośniki}
 
Poniżej zamieszczono kilka odnośników do stron WWW związanych z Riakiem.

\begin{description}
 \item [http://wiki.basho.com/] - strona wiki z dokumentacją projektu
 \item [http://www.basho.com/] - strona domowa projektu
 \item [http://blog.basho.com/] - oficjalny blog, na którym można znaleźć wiele ciekawych informacji na temat Riaka
 \item [http://seancribbs.github.com/ripple/] - biblioteka dla języka Ruby implementująca zarówno niskopoziomowego klienta Riaka, ale także bibliotekę mapującą obiekty na rekordy w Riaku (odpowiednik ORM).
\end{description}
