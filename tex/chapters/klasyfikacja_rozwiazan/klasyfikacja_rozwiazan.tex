\chapter{Klasyfikacja rozwiązań}

\section*{Streszczenie}

Jednym z założeń ruchu NoSQL jest zwrócenie uwagi użytkowników systemów bazodanowych na fakt, że dla różnych aplikacji i do różnych zastosowań mogą pasować zupełnie różne systemy.
Ogromna liczba nierelacyjnych baz danych\footnote{Strona http://nosql-database.org/ wymienia ponad stu takich baz.} sprawia, że niemożliwym jest opisanie wszystkich z nich w niniejszej pracy.
Z tego względu przedstawionych zostanie jedynie kilka najlepiej znanych i najbardziej popularnych rozwiązań.

Istnieje wiele możliwych sposobów klasyfikacji systemów NoSQL.
Na potrzeby tej pracy przyjęto najczęściej stosowany system, który dzieli bazy ze względu na wykorzystywany model danych.

\section{Podział ze względu na model danych}

\begin{description}
 \item[Bazy typu klucz-wartość]
 Systemy tego typu modelują dane jako tablicę asocjacyjną.
 W takiej tablicy unikalnemu kluczowi jest przyporządkowana wartość.
 Zazwyczaj ani klucz, ani jego wartość nie są interpretowane przez system.
 Bazy tego typu udostępniają zazwyczaj jedynie prosty interfejs pozwalający na odczytanie lub ustawienie wartości klucza bądź jego usunięcie; bardziej skomplikowane zapytania są zazwyczaj niemożliwe.
 
 \item[Bazy kolumnowe]
 Systemy te wzorują swój model danych na Google BigTable (patrz strona \pageref{google-bigtable-model-danych}).
 W bazie kolumnowej unikalnemu kluczowi odpowiada wiele kolumn.
 Kolumny są grupowane w rodziny - zazwyczaj wszystkie kolumny należące do tej samej rodziny są jednego typu.
 W przypadku Apache Cassandra kilka rodzin kolumn może być zebranych w super-rodzinę.
 Kolumny przynależące do jednej rodziny z reguły zawierają dane jednego typu, gdyż są one kompresowane razem.
 Bazy te zazwyczaj wymagają zdefiniowania struktury bazy z dokładnością do rodzin, czy super-rodzin kolumn.
 Jedynie kolumny, którym jest przypisana jakaś wartość dla danego klucza zajmują miejsce w pamięci, dzięki temu różnych kolumn może być bardzo wiele i często służą na przykład do tworzenia list wartości.
 Bazy kolumnowe pozwalają jedynie na wykonywanie zapytań o pojedynczy klucz lub ich zakres, ale za to oferują szeroki zakres możliwości limitowania zwracanych kolumn.

 \item[Bazy dokumentowe]
 W systemie dokumentowym unikalnemu kluczowi przypisany jest tak zwany dokument.
 Dokument jest to drzewo, którego liśćmi są wartości prymitywne (np. ciąg znaków, liczba, wartość logiczna, pusty obiekt NULL, itp.) a węzłami wartości złożone (tablica, tablica asocjacyjna).
 Dokumenty są zazwyczaj zgrupowane w kolekcje, po których można iterować.
 Dokumenty należące do tej samej kolekcji nie muszą mieć tych samych pól czy struktury.
 W odróżnieniu od systemu typu klucz-wartość przechowującego dane w formacie JSON, systemy dokumentowe są w stanie interpretować wartości zawarte w dokumentach i umożliwiają użytkownikowi efektywne wyszukiwanie rekordów na podstawie tych wartości przy pomocy różnych rodzajów indeksów.
 W bazie dokumentowej nie występują zazwyczaj relacje między dokumentami, ale możliwe jest zagnieżdżanie dokumentów, co ułatwia modelowanie relacji jeden-do-wielu.
 
 \item[Bazy grafowe]
 Głównym wyróżnikiem grafowych baz danych jest to, że baza grafowa pozwala w złożoności O(1) uzyskać informacje o relacjach danego węzła z innymi węzłami.
 W takim systemie można uzyskać informacje zarówno o krawędziach wchodzących jak i wychodzących.
 Zarówno węzły, jak i krawędzie grafu są opisane właściwościami.
 Podobnie jak w przypadku baz dokumentowych węzły czy krawędzie nie mają z góry narzuconej struktury i mogą się różnić liczbą, typem i nazwami właściwości.
 Podstawowym mechanizmem zapytań w bazie grafowej są rożne mechanizmy trawersowania grafu biorąc pod uwagę właściwości węzłów i krawędzi i poruszając się po krawędziach wchodzących i wychodzących.
 Drugorzędnym mechanizmem jest wyszukiwanie węzłów i krawędzi grafu na podstawie ich właściwości.
 Możliwość wykonywania takich zapytań nie jest jednak konieczna w bazie grafowej i niektóre produkty pozostawiają obsługę takich zapytań innym systemom. 

 \item[Skalowalne bazy relacyjne]
 Do tej grupy należą systemy, które zachowują relacyjny model danych, oraz najczęściej SQL jako język zapytań, ale w odróżnieniu od typowych relacyjnych baz danych są zaprojektowane z myślą o horyzontalnej skalowalności.
\end{description}

\section{Opisane aspekty rozwiązań}

Ponieważ dokładne przedstawienie dowolnego systemu bazodanowego na kilku stronach A4 jest zadaniem nie do wykonania, zamierzeniem autora tej pracy było wybranie najbardziej istotnych i cennych dla czytelnika informacji.
Pominięte zostały przede wszystkim szczegóły dotyczące API poszczególnych systemów oraz łatwo dostępne informacje, takie jak instrukcja instalacji systemu.
Opisane zostały natomiast istotne cechy systemów, które pozwalają czytelnikowi lepiej zrozumieć mechanizm działania aplikacji.
Każda baza została opisana według następującego schematu:

\begin{description}
 \item[Wstęp] 
 Każdą sekcję poprzedzono wstępem przybliżającym w kilku zdaniach opisywaną bazę. 
 
 \item[Protokół komunikacji]
 Opisuje dostępne protokoły komunikacji z bazą i wsparcie dla różnych języków programowania.

 \item[Replikacja]
 Opisuje dostępne mechanizmy replikacji systemu i możliwości ich konfiguracji.

 \item[Partycjonowanie]
 W przypadku baz obsługujących horyzontalne partycjonowanie danych opisany zostanie ten mechanizm i jego możliwości konfiguracji.
 W przypadku baz nie dających takiej możliwości opisane zostaną narzędzia ją zapewniające.

 \item[Persystencja]
 Opisany zostanie mechanizm trwałego zapisu danych oraz różne opcje konfiguracji systemu z tym związane i w przypadku systemów, które obsługują wymienne moduły persystencji, zostaną one opisane.

 \item[Wersjonowanie]
 W przypadku systemów które oferują taką opcję, zostanie opisany mechanizm wersjonowania rekordów i rozwiązywania konfliktów.

 \item[Wyszukiwanie]
 Opisane zostaną ogólnie opcje tworzenia zapytań do systemu i indeksowania danych.

 \item[Unikalne cechy]
 Przedstawione zostaną cechy systemu, które wyróżniają go na tle konkurencji.

 \item[Typowe zastosowania]
 Opisane zostaną typowe zastosowania i znane wdrożenia danego systemu.

 \item[Przeciwwskazania]
 Opisuje typy zastosowań, z którym system może mieć problemy, szczególnie takie, które mogą nie występować przy prostej ewaluacji.

 \item[Dokumentacja i wsparcie]
 Opisany zostanie poziom aktywności społeczności związanej z daną bazą, możliwości komercyjnego wsparcia, subiektywna ocena jakości dokumentacji oraz w przypadku produktów rozwijanych przez konkretnych dostawców, zostaną oni przedstawieni.

 \item[Pomocne odnośniki]
 Każdy opis zostanie zakończony garścią odnośników pomocnych w zapoznawaniu się z systemem.

\end{description} 